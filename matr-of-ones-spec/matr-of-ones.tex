\documentclass{article}

\usepackage{amsmath}
\usepackage{amsfonts}
% \usepackage[utf8]{inputenc}

\usepackage{amsthm}
\newtheorem{propos}{Proposition}

\providecommand{\Matr}{\mathit{Matr}}
\providecommand{\NN}{\mathbb{N}}

\title{Spectra of matrix of ones}
\author{rn}

\begin{document}
\maketitle
\tableofcontents
\newpage

\section{Intro}
Let $\mathcal A \in \Matr_{N\times N}$ be matrix consisting of all ones
$$\mathcal A =
\begin{pmatrix}
    1      & 1      & \cdots & 1 \\
    1      & 1      & \cdots & 1 \\
    \vdots & \vdots & \ddots & \vdots \\
    1      & 1      & \cdots & 1
\end{pmatrix}$$

Note
$$\mathcal A^2 =
\begin{pmatrix}
    N      & \cdots & N \\
    \vdots & \ddots & \vdots \\
    N      & \cdots & N
\end{pmatrix} = N \mathcal A$$

And
$\mathcal A^k = N^{k-1} \mathcal A$

It follows then, that polynomial $p$ given by formula
$$p(\lambda) = \lambda^2 - N \lambda$$
is annihilating polynomial of $\mathcal A$,
since $\mathcal A^2 - N\mathcal A = N\mathcal A - N\mathcal A = 0$.

From Hamilton-Cayley theorem and things (TODO: fixme, i.e. make it more specific)
one can see that spectrum of $A$
is a subset of set of roots of $p(\lambda)$
$$\sigma(A) \subset \{ \lambda; \quad p(\lambda) = 0 \}$$

Now let's try to find corresponding eigenvectors.
$N$ is an eigenvalue, since $A x = N x$ for $x = \left(1, \cdots, 1\right)^T$.

As for zero, its eignspace consists exactly of these vectors,
for which the sum of the coordinates equals zero.
Choose the basis of $E(A; 0)$ as
$$
g_1 = \begin{pmatrix}1 \\ -1 \\ 0 \\ \vdots \\ 0\end{pmatrix}
g_2 = \begin{pmatrix}0 \\ 1 \\ -1 \\ 0 \\ \vdots \\ 0\end{pmatrix}
g_{n-1} = \begin{pmatrix}0 \\ \vdots \\ 0 \\ 1 \\ -1 \end{pmatrix}$$

We can orthonormalize it then.

$$e_1 = g_1 / \|g_1\| =
\begin{pmatrix}\frac{1}{\sqrt2} \\ - \frac{1}{\sqrt2} \\ \vdots \end{pmatrix}$$

$$f_2 = g_2 - \left(g_2|e_1\right)e_1 =
\begin{pmatrix}0\\ 1\\ -1\\ \vdots\end{pmatrix}
-
\frac{-1}{\sqrt2} \begin{pmatrix}\frac{1}{\sqrt2} \\ - \frac{1}{\sqrt2} \\ \vdots \end{pmatrix}
= \begin{pmatrix} 1/2 \\ 1/2 \\ -1 \\ \vdots \end{pmatrix}$$

$$e_2 = f_2/\|f_2\| =
f_2 \frac{1}{\sqrt{4/3}} =
\begin{pmatrix} \frac1{\sqrt6} \\ \frac1{\sqrt6} \\ - \frac2{\sqrt6} \\ \vdots \end{pmatrix}$$

Same way

$$f_3 =
g_3 - \underbrace{\left(g_3|e_1\right)}_{=0}e_1
    - \underbrace{\left(g_3|e_2\right)}_{=-\sqrt{\frac23}}e_2
= \begin{pmatrix}0 \\ 0 \\ 1 \\ -1 \\ \vdots\end{pmatrix}
    - (-\sqrt{\frac23}) \begin{pmatrix} \frac1{\sqrt6} \\ \frac1{\sqrt6} \\ - \sqrt{\frac23} \\ \vdots \end{pmatrix}
= \begin{pmatrix} 1/3 \\ 1/3 \\ 1/3 \\ -1 \\ \vdots \end{pmatrix}$$

    $$e_3 = f_3 / \|f_3\| = f_3 \frac{1}{\sqrt{1/3 + 1}} =
\begin{pmatrix}
    \frac{1}{2\sqrt3} \\ \frac{1}{2\sqrt3} \\
    \frac{1}{2\sqrt3} \\ - \frac{3}{2\sqrt3} \\
    \vdots
\end{pmatrix}$$

So suppose that first $m$ vectors of this orthonormal system
are of the form

$$ e_k =
\begin{pmatrix}
    a_k \\ a_k \\ \vdots \\ a_k \\ b_k \\ 0 \\ \vdots \\ 0
\end{pmatrix}$$
i.e. first $k$ elements of the vector all equal the same $a_k$
and $(k+1)$'st is the $b_k = -ka_k$ closing the sequence.
This is true at least for first 3 vectors.
At the same time $g_{k+1}$ has all zeros
except $(k+1)$'st and $(k+2)$'nd entries which are $1$ and $-1$ respectively.
Hence $\left(g_k|e_{k-1}\right) = b_{k-1}$ and $\left(g_k|e_j\right) = 0$ for $j < k-1$.
That gives

$$f_k = g_k - \left(g_k|e_{k-1}\right) e_{k-1} =
\begin{pmatrix}
    - a_{k-1} b_{k-1} \\
    - a_{k-1} b_{k-1} \\
    \vdots \\
    - a_{k-1} b_{k-1} \\
    1 - b_{k-1}^2     \\
    -1 \\
    \vdots
\end{pmatrix}$$

One can also note (TODO: prove) (TODO: REALLY) that
$1 - b_{k-1}^2 = - a_{k-1} b_{k-1}$ for $k>1$
and thus

$$f_k =
\begin{pmatrix}
    - a_{k-1} b_{k-1} \\
    \vdots \\
    - a_{k-1} b_{k-1} \\
    -1 \\
    \vdots
\end{pmatrix}$$

Norm of this vector can be calculated as

\providecommand{\fknorm}{\sqrt{k a_{k-1}^2 b_{k-1}^2 + 1}}
$$\|f_k\| = \fknorm$$

Now that $b_k = -(k-1)a_k$

\renewcommand{\fknorm}{\sqrt{k(k-1)^2 a_{k-1}^4 + 1}}
$$\|f_k\| = \fknorm$$

If $a_k = 1/c_k$ then the closing entry is $b_k = -k/c_k$.
Also $b_k$ is always $-1/\|f_k\| = \frac{-k}{k\|f_k\|}$.
That gives us immediately

$$e_k =
\frac{1}{k\fknorm}
\begin{pmatrix}1 \\ \vdots \\ 1 \\ -k \\ \vdots\end{pmatrix}$$
for $k=2,N-1$.

$$e_1
    = \begin{pmatrix}a_1 \\ -a_1 \\ \vdots \end{pmatrix}
    = \begin{pmatrix}\frac1{\sqrt2} \\ - \frac{1}{\sqrt2} \\ \vdots \end{pmatrix}$$

The change of basis transformation $U$ which diagonalizes $A$ is given by matrix

    $$\mathcal U =
    \begin{pmatrix}
         1/\sqrt2 & 1/\sqrt6  & 1/2\sqrt3  & \cdots & a_{n-1} & 1/\sqrt N \\
        -1/\sqrt2 & 1/\sqrt6  & 1/2\sqrt3  & \cdots & a_{n-1} & 1/\sqrt N \\
                  & -2/\sqrt6 & 1/2\sqrt3  & \cdots & a_{n-1} & 1/\sqrt N \\
                  &           & -3/2\sqrt3 & \cdots & a_{n-1} & 1/\sqrt N \\
                  &           &            & \ddots & a_{n-1} & 1/\sqrt N \\
                  &           &            &        & \vdots  & \vdots    \\
                  &           &            &        & a_{n-1} & 1/\sqrt N \\
                  &           &            &        & -(n-1)a_{n-1} & 1/\sqrt N
    \end{pmatrix}$$

\begin{propos}
    For every $N\in\NN$
    $$\det U = 1$$
    where $U$ is the change of basis operator
    that diagonalizes the matrix of ones

\end{propos}

\end{document}
