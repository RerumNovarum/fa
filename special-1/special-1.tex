\documentclass{article}

\usepackage{fontspec}
\setmainfont{CMU Serif}
\usepackage{polyglossia}
\setdefaultlanguage{russian}
\setotherlanguage{english}

\usepackage{amsmath}
\usepackage{amsfonts}
\usepackage{mathrsfs}
\usepackage{mathtools}

\newtheorem{thm}{Теорема. }

\title{О спектральных свойствах некоторого специального класса матриц}
\author{Sergey Kozlukov}

\begin{document}
\maketitle

Пусть \( H \) --- \( M \)-мерное евклидово пространство,
\( A: H\to H \) --- действующий на \( H \) обратимый линейный оператор,
с известными (не обязательно различными) собственными значениями
и соответствующими ортонорированными собственными векторами \( e_1, \ldots, e_M \).

Рассмотрим линейный оператор \( \mathbb{A} \),
действующий на \( MN \)-мерном евклидовом пространстве \( \mathcal{H} = H^N \)
с естественным индуцированным скалярным произведением,
заданный операторной матрицей:
\[
    \mathbb{A} = \begin{pmatrix}
        A & \cdots & A \\
        \vdots & \ddots & \vdots \\
        A & \cdots & A
    \end{pmatrix}.
    \]

Можно найти жорданову нормальную форму матрицы такого оператора.
Заметим, что ядро \( \mathcal{H}_0 = \mathtt{Ker} \mathbb{A} \) оператора \( \mathbb{A} \)
состоит в~точности из тех векторов \( x = ( x_1, \ldots, x_N ) \), для которых \( \sum_i x_i  = 0 \).
Так как \( \mathtt{rank}\mathbb{A} = \mathtt{rank} A = M \),
то в качестве базиса в \( \mathcal{H}_0 \) можно выбрать систему
из следующих \( MN - M \) векторов:
\[ h_1^1 = \begin{pmatrix} e_1 \\ -e_1 \\ 0 \\ \vdots \\ 0 \end{pmatrix}, \cdots,
    h_1^{N-1} = \begin{pmatrix} 0 \\ \vdots \\ 0 \\ e_1 \\ -e_1 \end{pmatrix}, \cdots,
    h_M^1 = \begin{pmatrix} e_M \\ -e_M \\ 0 \\ \vdots \\ 0 \end{pmatrix}, \cdots,
    h_M^{N-1} = \begin{pmatrix} 0 \\ \vdots \\ 0 \\ e_M \\ -e_M \end{pmatrix}. \]
Рассмотрим теперь ортогональное ядру подпространство
\[ \mathcal{H}_1 = \mathcal{H}_0^\perp = \{ x\in\mathcal{H};\quad (x,y)=0 \text{ для всех } y\in\mathcal{H}_1 \}, \]
так что \( \mathcal{H} = \mathcal{H}_0 \oplus \mathcal{H}_1 \).
Ясно, что \( x~=~(x_1,\ldots,x_N) \) лежит в \( \mathcal{H}_1 \) тогда и~только тогда, когда
\( 0 = (x, h_i^j) = (x_j - x_{j+1}, e_i) \) для всех \( i=\overline{1,M}, j=\overline{1,N-1} \),
т.е. вектор \( x_j - x_{j+1} \in H \) ортогонален всем \( e_1, \ldots, e_M \),
формирующим базис в~\( H \).
Отсюда следует, что  \( x_j - x_{j+1} = 0 \),
или, что~то~же~самое:
Это~означает,~что
\[ \mathcal{H}_1 = \{ %\left\{
    x = \begin{pmatrix} x_0 \\ \vdots \\ x_0 \end{pmatrix}; \quad
    x_0 \in H
    \}, \]
    %\right\}. \]
а в качестве базиса в \( \mathcal{H}_1 \) можно выбрать (собственные) векторы
\( h_k^0 = (e_k, \ldots, e_k), k=\overline{1,M} \).

К векторам \( h_k^j \) применим процесс ортогонализации Грамма-Шмидта и получим
ортонормированный базис в \( \mathcal{H} \):
\[ f_k^0 = \frac{1}{\sqrt{N}} \begin{pmatrix} e_j \\ \vdots \\ e_j \end{pmatrix}, k=\overline{1,M}, \]
\[
    f_k^j = \frac{1}{\sqrt{k^2+k}}
    \begin{pmatrix} e_j \\ \vdots \\ e_j \\ -ke_j \\ 0 \\ \vdots \\ 0 \end{pmatrix}, k=\overline{1,M}, j=\overline{1,N-1} \]
где в \( \|f_k^j\| f_k^j, j=\overline{1,N-1} \) на первых \( k \) местах стоят единицы, на \((k+1)\)-ом --- \( -k \),
а на остальных нули.

Ясно, что векторы \( f_k^j, k\in\overline{1,M}, j\in\overline{0,N-1} \)
являются собственными для оператора \( \mathbb{A} \):
векторам \( f_k^0 \) соответствует собственное значение \( N \lambda_k \),
а остальным -- ноль:
\[ \sigma(\mathbb{A}) = N \sigma(A)\cup\{0\} = \{ 0, N\lambda_1, \ldots, N\lambda_M. \]

\begin{thm}[О жордановой нормальной форме матрицы  оператора \( \mathbb{A} \)]
    ...Матрица имеет вид... Унитарная матрица преобразования имеет вид...
\end{thm}

% Будем рассматривать линейные операторы \( A_{2N}\in L(\mathcal{H}) \), задаваемые матрицами \( \mathcal{A}_{2N} \in \mathtt{Matr}_{2N}\mathbb{C} \) размера \( 2N\times 2N \) вида
% \[ \mathcal{A}_{2N} =
%     \begin{pmatrix}
%     0 & 1 & 0 & 1 & \cdots &  1 \\
%     1 & 0 & 1 & 0 & \cdots &  0 \\
%     0 & 1 & 0 & 1 & \cdots &  1 \\
%     \vdots & \vdots & \vdots & \vdots & \ddots & \vdots \\
%     1 & 0 & 1 & 0 & \cdots & 0
%     \end{pmatrix}. \]
% Их можно записать в виде матрицы, составленной из \( N\times N \) блоков
% \[ \mathcal{A}_{2N} =
%     \begin{pmatrix}
%         \mathcal{A}_2 & \cdots & \mathcal{A}_2 \\
%         \vdots        & \ddots & \vdots \\
%         \mathcal{A}_2 & \cdots & \mathcal{A}_2.
%         \end{pmatrix}, \]
% \[ \mathcal{A}_2 = \begin{pmatrix} 0 & 1 \\ 1 & 0 \end{pmatrix}. \]
% 
% Легко проверяется, что собственными значениями матрицы \( \mathcal{A}_2 \)
% являются \( 1 \) и \( -1 \), а соответствующими собственными векторами --- 
% \( e_1 = \begin{pmatrix}1 \\ 1\end{pmatrix} \),
% \( e_2 = \begin{pmatrix}1 \\ -1\end{pmatrix} \).
% 
% Найдём спектр и собственный базис матрицы \( \mathcal{A}_{2N} \).
% Для этого введём в рассмотрение векторы
% \[ h_1^1 = \begin{pmatrix} e_1 \\ -e_1 \\ 0 \\ \vdots \\ 0 \end{pmatrix}, \cdots, h_1^{N-1} = \begin{pmatrix} 0 \\ \vdots \\ 0 \\ e_1 \\ -e_1 \end{pmatrix},
%    h_2^1 = \begin{pmatrix} e_2 \\ -e_2 \\ 0 \\ \vdots \\ 0 \end{pmatrix}, \cdots, h_2^{N-1} = \begin{pmatrix} 0 \\ \vdots \\ 0 \\ e_2 \\ -e_2 \end{pmatrix}, \]
% очевидно лежащие в ядре рассматриваемого оператора.
% Также рассмотрим генерируемое ими подпространство
% \[ \mathcal{H}_0 = \mathtt{span}(h_1^1,\ldots,h_1^{N-1},h_2^1,\ldots,h_2^{N-1}), \]
% размерности \( \dim\mathcal{H}_0 = 2N - 2. \)
% Так как, очевидно, \( \mathtt{rank}{\mathcal{A}_{2N}} = 2 \),
% то \( \mathcal{H}_0 \) есть в точности ядро этого оператора: \( \mathcal{H}_0 = \mathtt{Ker}(A_{2N}) \).
% 
% Рассмотрим теперь ортогональное дополнение \( \mathcal{H}_1 = \mathcal{H}_0^{\perp} \) к подпространству \( \mathcal{H}_0 \)
% --- такое линейное подпространство \( \mathcal{H}_1 \subset \mathcal{H} \),
% что \( \mathcal{H} = \mathcal{H}_0 \oplus \mathcal{H}_1 \)
% и для всех \( x\in\mathcal{H}_1, y\in\mathcal{H}_0 \) имеет место \( \langle x, y \rangle = 0 \).
% Последнее эквивалентно условию
% \[ \langle x,h_i^j \rangle = 0, \quad \text{ для всех } x\in\mathcal{H}_1, i=\overline{1,2}, j=\overline{1,N-1}. \]
% Заметим:
% \[ \langle x, h_i^j \rangle = (x_i, e_j) + (x_{i+1}, -e_j) = (x_i - x_{i+1}, e_j). \]
% Отсюда для всех \( x = (x_1, \ldots, x_N) \in \mathcal{H}_1 \), где \( x_j = ( x_j^1, x_j^2 )\in H, j = \overline{1,N} \):
% \begin{align*}
% &   \left\{ \begin{aligned}
%     & (x_i - x_{i+1}, \begin{pmatrix} 1 \\1 \end{pmatrix} )  = x_i^1 - x_{i+1}^1 + x_i^2 - x_{i+1}^2 = 0, \\
%     & (x_i - x_{i+1}, \begin{pmatrix} 1 \\ -1\end{pmatrix} ) = x_i^1 - x_{i+1}^1 - x_i^2 + x_{i+1}^2 = 0,
%     \end{aligned} \right. \\
% &   i = \overline{1,N-1}. \end{align*}
% Решая эту систему линейных уравнений, получаем \( x_i = x_{i+1}, i = \overline{1,N} \).
% Короче, все векторы \( x\in \mathcal{H}_1 \) имеют вид \( x = (x_0, \ldots, x_0), \) где \( x_0\in H \).
% В качестве базиса в \( \mathcal{H}_1 \) естественно выбрать \( h_0^i = (e_i, \ldots, e_i), i=\overline{1,2} \).
% Непосредственно проверяется, что эти векторы являются собственными, с соответствующими собственными числами \( N \) и \( -N \).
% 
% В построенном базисе \( h_0^1, h_0^2, h_1^1, \ldots, h_1^{N-1} , h_2^1, h_2^{N-1} \)
% матрица рассматриваемого оператора примет вид жордановой нормальной формы:
% \[ \begin{pmatrix}
%     N & 0  & 0 & \cdots & 0 \\
%     0 & -N & 0 & \cdots & 0 \\
%     0 & 0  & 0 & \cdots & 0 \\
%     \vdots & \vdots & \vdots & \ddots & \vdots \\
%     0 & 0  & 0 & \cdots & 0
%     \end{pmatrix} \]
% % 
% % Далее будем рассматривать матрицы вида
% % \[ \mathcal{A}_{2N} - \mathcal{B}_{2N}, \]
% % где \( \mathcal{B}_{2N} \) -- некоторое возмущение.
 \end{document}
