\section{Введение}

Здесь и, если не уточнено иное, далее $X$ --- конечномерное линейное пространство над полем $\KK$,
$\dim X = n \geq 3$,
$A : X\to X$ --- линейный оператор ранга $\rank A = \dim\Img A = 3$.
$A_0 = A|\Img A : \Img A\to \Img A$ --- сужение оператора $A$ на подпространство $\Img A$.
$\mathcal A_0 \in \Matr_n$ --- матрица оператора $A_0$.

\begin{dfn}{Спектр}
  Говорят, что число $\lambda\in\KK$ принадлежит спектру $\spec A$ оператора $A$,
  если оператор $B = A - \lambda I$ не обратим:
  $\Ker B \neq \{0\} \lor \Img B \neq X$, т.е.
  $$ \spec A = \{ \lambda\in\KK ;\qquad \Ker (A-\lambda I) \neq \{0\} \lor \Img (A-\lambda I) \neq X \}$$

  В конечномерном пространстве эти два условия эквивалентны, т.к. $X = \Ker B \oplus \Img B$
\end{dfn}

Задача состоит в том, чтобы найти явные формулы для собственных значений оператора ранга три.

\subsection{Структура оператора}
Для начала уточним, как ``работает'' линейный оператор $A: X\to X$ ранга три.
Пусть $e_1, e_2, e_3\in X$ --- базис в $\Img A$.
Для любого $x\in X$,
$A x = \xi_1(x) e_1 + \xi_2(x) e_2 + \xi_3(x) e_3$,
где $xi_j: X\to \KK$ --- некоторые функционалы.
Несложно показать, что эти функционалы линейны:
$$ A(x_1 + x_2) = \sum_{j=1}^3 \xi_j(x_1 + x_2) e_j $$
$$ A(x_1 + x_2) = A x_1 + A x_2 = \sum_{j=1}^3 (\xi_j(x_1) + x_j(x+2)) e_j $$
$$ A(\alpha x) = \sum_{j=1}^3 \xi_j(\alpha x) e_j $$
$$ A(\alpha x) = \alpha A x = \sum_{j=1}^3 \alpha \xi_j(x) e_j $$
Где $(e_j)$ линейно-независимы, и поэтому для любого $x\in X$
$$\xi_j(x_1 + x_2) = \xi_j(x_1) + \xi_j(x_2)$$
$$\xi_j(\alpha x) = \alpha\xi_j(x)$$
Кроме того, $(xi_j)$ линейно-независимы:
$$\xi_1 = \alpha\xi_2 + \beta\xi_3
\implies
A x = \xi_2(x) (\alpha e_1 + e_2) + \xi_3(x) (\beta e_1 + e_3)
\implies
\dim\Img A \leq 2$$
В то время как $\dim\Img A = 3$ и мы пришли к противоречию.

Таким образом, доказана
\begin{thm}
  Пусть $A: X\to X$,
  $\rank A = 3$.
  $e_0, e_1, e_2$ --- базис в $\Img A$.
  Тогда
  \begin{equation*}\begin{aligned}
      & \exists \xi_0, \xi_1, \xi_2: X\to \KK \text{ --- линейно-независимые линейные функционалы} \\
      & \forall x\in X \qquad A x = \xi_0(x) e_0 + \xi_1(x) e_1 + \xi_2(x) e_2
    \end{aligned}\end{equation*}
\end{thm}

\subsection{Спектр оператора ранга три}
Для начала заметим:
\begin{thm}
  $$\left\{\begin{aligned}
    & \dim X > 3 \implies 0\in \spec A \\
    & \dim X = 3 \implies 0\notin \spec A\end{aligned}\right.$$
\end{thm}
\begin{proof}
  Утверждение очевидным образом
  следует из $3 < n = \dim X = \dim\Ker A + \dim\Img A$.

  В первом случае $\dim\Ker A = \dim X - \dim\Img A = n - 3 \geq 1$.
  I.e., существует ненулевой  $x\in X$, для которого $A x = 0 x$, а значит $0\in\spec A$.

  Если же $\dim X = 3$, то $\dim X = \rank A$, $A$ обратим и потому $0\notin\spec A$
\end{proof}

Введём теперь $A_0 = A|\Img A \in \LB(\Img A)$
--- сужение оператора $A$ на подпространство $\Img A\subset X$.

$\mathcal A_0 = \left(a_{ij}\right) \in \KK^{3\times 3}$ --- матрица оператора $A_0$.

Её столбцы --- это векторы $A_0 e_j  = A e_j$, то есть $a_{ij} = \xi_i(e_j)$:

$$\mathcal A_0 =
\begin{pmatrix}
\xi_1(e_1) & \xi_1(e_2) & \xi_1(e_3) \\
\xi_2(e_1) & \xi_2(e_2) & \xi_2(e_3) \\
\xi_3(e_1) & \xi_3(e_2) & \xi_3(e_3)
\end{pmatrix}$$

Если $\dim X = 3$, то $A = A_0$ и $\spec A = \spec A_0$.
Пусть $\dim X > 3$.

Из предыдущей теоремы следует,
что $0\in\spec A$ и $0\notin\spec A_0$.

Если $0\neq\lambda\in\spec A$, то $\exists x\neq 0$,
такой что $A x = \lambda x \in \Img A$.

Тогда $x = \frac{1}{\lambda} A x\in\Img A$, i.e. $x\in \Img A$,
то есть $A_0$ определён на $x$,
и $A_0 x = \lambda x$, что и означает $\lambda\in\spec A_0$.

Обратно, если $\lambda\in\spec A_0$,
то $\exists x\in\Img A\subset X$,
для которого $A x = A_0 x = \lambda x$,
i.e. $\lambda\in\spec A$.

Фактически, доказана
\begin{thm}{О связи спектра оператора ранга три и спектра его сужения}
  \begin{equation*}
    \left\{\begin{aligned}
      & \dim X = 3 & \implies & \spec A = \spec A_0 \\
      & \dim X > 3 & \implies & \spec A = \spec A_0 \cup \{0\}
    \end{aligned}\right.\end{equation*}
\end{thm}

\begin{thm}{О структуре спектра оператора ранга три}\*
  \label{thm:specstr}
  $$
  \dim X = 3 \implies
  (\spec A = \emptyset)
  \lor (\exists \lambda_1 \neq 0 \quad
  \spec A = \{\lambda_1,
    \frac{1}{2} (\tr\mathcal A_0 - \lambda_1
    \pm\sqrt{ {(\tr\mathcal A_0 - \lambda_1)}^2 - 4\frac{1}{\lambda_1} \det\mathcal A_0 })
  \})$$
  $$
  \dim X > 3 \implies
  (\spec A = \{0\})
  \lor (\exists \lambda_1 \neq 0 \quad
  \spec A = \{0, \lambda_1,
    \frac{1}{2} (\tr\mathcal A_0 - \lambda_1
    \pm\sqrt{ {(\tr\mathcal A_0 - \lambda_1)}^2 - 4\frac{1}{\lambda_1} \det\mathcal A_0 })
  \})$$
\end{thm}
\begin{proof}
  Пусть имеется $\lambda_1\in\spec A_0$.
  Из предыдущих теорем известно, что $\lambda_1\neq 0$.
  Характеристический многочлен $A_0$ имеет вид
  $$\chi(\lambda) = -\lambda^3 + \tr\mathcal A_0 \lambda^2 + k\lambda + \det\mathcal A_0$$
  NB: $k$ легко вывести из определения определителя, но в этом нет нужды.

  В то же время, $\lambda_1$ является корнем многочлена $\chi$, поэтому
  $$\chi(\lambda) = (\lambda - \lambda)(c_1 \lambda^2 + c_2 \lambda + c_3)$$

  Приравнивая эти представления, получаем:
  $$\left\{\begin{aligned}
    & c_1 = -1 \\
    & c_2 = \tr\mathcal A_0 - \lambda_1 \\
    & c_3 = - \frac{1}{\lambda_1} \det\mathcal A_0
  \end{aligned}\right.$$

  $$\chi(\lambda) = -(\lambda - \lambda)(\lambda^2 + (\lambda_1 - \tr\mathcal A_0) \lambda + \frac{1}{\lambda_1}\det\mathcal A_0)$$

  Значит числа вида
  $$\lambda =
  \frac{1}{2} (\tr\mathcal A_0 - \lambda_1
  \pm\sqrt{ {(\tr\mathcal A_0 - \lambda_1)}^2 - 4\frac{1}{\lambda_1} \det\mathcal A_0 }) $$

  также являются корнями $\chi$.

  $$\spec A_0 = \{\lambda_1,
    \frac{1}{2} (\tr\mathcal A_0 - \lambda_1
    \pm\sqrt{ {(\tr\mathcal A_0 - \lambda_1)}^2 - 4\frac{1}{\lambda_1} \det\mathcal A_0 })
  \}$$

  Наконец, $\spec A = \spec A_0$ если $\dim X = 3$, иначе $\spec A = \spec A_0 \cup \{ 0 \}$
\end{proof}

\subsection{Спектральное разложение}
Снова $X$ --- линейное пространство, $\dim X = n$.
\begin{dfn}{Оператор простой структуры}
  Говорят, что оператор $A\in\LB(X)$ --- простой структуры,
  если у него есть $n=\dim X$ линейно-независимых собственных векторов,
  то есть, в $X$ существует базис $(v_1, \ldots, v_n)$, такой что
  $$A v_k = \lambda_k v_k, \quad k=\overline{1,n}, \lambda_k\in\spec A$$
\end{dfn}

Известен следующий результат линейной алгебры:
\begin{thm}{Интерполяционная формула Сильвестра}
  Пусть $A\in\LB(X)$ --- простой структуры,
  $(v_1, \ldots, v_n)$ --- линейно-независимы
  и $A v_k = \lambda v_k$.
  Тогда имеет место разложение:
  $$A = \sum_{k=1}^n \lambda_k P_k$$

  Где
  $$P_k = \prod_{j=1,j\neq k} \frac{A - \lambda_j}{\lambda_k - \lambda_j}$$
\end{thm}

Далее будем рассматривать операторы ранга три простой структуры.
Это означает, что в $\Img A$ существует базис, состоящий из собственных векторов.
В частности, это верно, когда спектр оператора $A|\Img A$ состоит из трёх различных точек.

Кроме того, как и прежде, будем считать одно собственное значение $\lambda_1\in\spec A$ известным.
Последняя теорема утверждает, что
$$A = \sout{0 P_0} + \lambda_1 P_1 + \lambda_2 P_2 + \lambda_3 P_3$$

Где $\lambda_1$ --- известное собственное значение,
$\lambda_2, \lambda_3$ определяются формулами из Thm.~\ref{thm:specstr},
а $P_k$ --- проекторы, вида
$$\begin{aligned}
  & P_1 = \frac{A - \lambda_2 I}{\lambda_1 - \lambda_2}\frac{A - \lambda_3 I}{\lambda_1 - \lambda_3}\frac{A}{\lambda_1} \\
  & P_2 = \frac{A - \lambda_1 I}{\lambda_2 - \lambda_1}\frac{A - \lambda_3 I}{\lambda_2 - \lambda_3}\frac{A}{\lambda_2} \\
  & P_3 = \frac{A - \lambda_1 I}{\lambda_3 - \lambda_1}\frac{A - \lambda_2 I}{\lambda_3 - \lambda_2}\frac{A}{\lambda_3} \\
  & P_0 = - \frac{A - \lambda_1 I}{\lambda_1}\frac{A - \lambda_2 I}{\lambda_2}\frac{A - \lambda_3}{\lambda_3} = I - P_1 - P_2 - P_3 \\
\end{aligned}$$

NB: Если $\dim X = 3$, то $P_0 = 0$.

\subsection{Экспонента}

\begin{dfn}{Операторная экспонента}
  Пусть $A\in\LB(X)$
  $$e^A \defeq \sum_{k=0}^\infty \frac{1}{k!} A^k \in\LB(X)$$
\end{dfn}

\begin{thm}
  Пусть $\dim X = n \geq 3$, $A\in\LB(X)$ --- оператор простой структуры, $\rank A = 3$,
  а $\lambda_0$ --- известное собственное значение.

  Тогда
  $$e^A = P_0 + e^{\lambda_1} P_1 + e^{\lambda_2} P_2 + e^{\lambda_3} P_3$$
\end{thm}
\begin{proof}
  \begin{align*}
  e^A & = \sum_{k=0}^\infty \frac{1}{k!} A^k 
    = \sum_{k=0}^\infty \frac{1}{k!} (\sum_{j=1}^3 \lambda_j P_j)^k
    = \sum_{k=0}^\infty \frac{1}{k!} (\sum_{j=1}^3 \lambda_j^k P_j) = \\
  & = I - P_1 - P_2 - P_3 + e^{\lambda_1} P_1 + e^{\lambda_2} P_2 + e^{\lambda_3} P_3 = \\
  & = P_0 +                 e^{\lambda_1} P_1 + e^{\lambda_2} P_2 + e^{\lambda_3} P_3\end{align*}
\end{proof}
