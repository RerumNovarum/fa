\documentclass{article}

\usepackage[cm]{fullpage}
\usepackage{hyperref}

\usepackage{fontspec}
\setmainfont{CMU Serif}

\usepackage{amsmath,amsthm,amssymb,amsfonts,mathtools,ulem}

\theoremstyle{plain}
\newtheorem{thm}{Th.}
\newtheorem{lemma}{Lemma.}
\newtheorem{corollary}{Corollary.}
\newtheorem*{NB}{NB}
\theoremstyle{definition}
\newtheorem{dfn}{Def.}

\providecommand{\LB}{\mathrm{L}}
\providecommand{\KK}{\mathbb{K}}
\providecommand{\NN}{\mathbb{N}}
\providecommand{\ZZ}{\mathbb{Z}}
\providecommand{\RR}{\mathbb{R}}
\providecommand{\CC}{\mathbb{C}}
\providecommand{\rank}{\mathrm{rank}}
\providecommand{\spec}{\mathrm{spec}}
\providecommand{\Ker}{\mathrm{Ker}}
\providecommand{\Img}{\mathrm{Im}}
\providecommand{\Matr}{\mathrm{Matr}}
\providecommand{\tr}{\mathrm{tr}}
\providecommand{\defeq}{\vcentcolon=}
\providecommand{\eqdef}{=\vcentcolon}

\title{DRAFT: Спектральный анализ операторов ранга три с одним известным собственным значением}
\date{\today}
\author{RN}

\begin{document}
  \maketitle
  \newpage

  \tableofcontents
  \newpage

  Пусть \( X \) --- \( M \)-мерное ( \( M < \infty \) ) линейное пространство,
\( A: X\to X \) --- обратимый линейный оператор простой структуры:
\( A \) имеет ровно \( M \) линейно-независимых собственных векторов \( e_1,~\ldots,~e_M \),
которым отвечают, вообще говоря не все различные, ненулевые собственные значения
\( \lambda_1,~\ldots,~\lambda_M \) (см. \cite{baskakov-algebra}).

Рассмотрим оператор
\( \mathbb{A}: X^N\to X^N \), действующий на пространстве \( X^N \) по формуле
\[ \mathbb{A}x =
    \begin{pmatrix}
        A & \cdots & A \\
        \vdots & \ddots & \vdots \\
        A & \cdots & A
    \end{pmatrix}
    \begin{pmatrix}
        x_1 \\
        \vdots \\
        x_N
    \end{pmatrix}
    = \begin{pmatrix}
        A \sum_{i=1}^N x_i \\
        \vdots \\
        A \sum_{i=1}^N x_i
    \end{pmatrix},
    \quad x=(x_1,\ldots,x_N) \in X^N. \]

В статье даны спектр и жорданов базис для операторов такого вида,
а также уточнения для случая блочных матриц, составленных из самосопряжённых блоков.


\end{document}
