Ненулевые собственные значения оператора $A$
удовлетворяют характеристическому многочлену оператора $A_0$,
так как
$$\left\{\begin{aligned}
& \dim X = 3 \implies \spec A = \spec A_0 \\
& \dim X > 3 \implies \spec A = \spec A_0 \cup \{0\}
\end{aligned}\right.$$

Для оператора $A_0$ в трёхмерном пространстве имеет место
$$\chi(\lambda) = -\lambda^3 + \tr\mathcal A_0 \lambda^2 + k\lambda + \det\mathcal A_0$$
где $\mathcal A_0$ --- матрица оператора $A_0$,
а $k$ несложно выписать пользуясь определением определителя (перебрав перестановки включающие $\lambda$).

Пусть известно некоторое
$\lambda_0\in\spec A_0$.
Тогда
$$\chi(\lambda) = (\lambda - \lambda_0) (c_0\lambda^2 + c_1\lambda + c_2)$$

Коэффициенты получим из равенства
$$\chi(\lambda) = -\lambda^3 + \tr\mathcal A_0 \lambda^2 + k\lambda + \det\mathcal A_0
  = c_0\lambda^3 + (c_1 - \lambda_0 c_0)\lambda^2 + (c_2 - \lambda_0 c_1)\lambda - \lambda_0 c_2$$

\begin{align*}
& c_0 = -1 \\
& c_1 = \tr\mathcal A_0 - \lambda_0 \\
& -\lambda_0 c_2 = \det\mathcal A_0
\end{align*}

Из последнего уравнения следует $\lambda_0\neq 0$,
так как $\rg A = 3$, $\det\mathcal A_0 \neq 0$
(TODO: переписать, результат можно было получить сразу, из того что в этом случае $\Ker A = \{0\}$).

Получается, если известно одно собственное значение $\lambda_0$ оператора $A_0$,
то оно ненулевое и
$$\chi(\lambda) = (\lambda - \lambda_0) (-\lambda^2 + (\tr\mathcal A_0 - \lambda_0)\lambda - \frac{1}{\lambda_0}\det\mathcal A_0)$$

Корни последнего многочлена можно выписать в явном виде:
$$\{\lambda_0,
     \frac{1}{2} (\tr\mathcal A_0 - \lambda_0
     \pm\sqrt{ {(\tr\mathcal A_0 - \lambda_0)}^2 - 4\frac{1}{\lambda_0} \det\mathcal A_0 }
  \}$$

Если
 $(\tr\mathcal A_0 - \lambda_0)^2 = 4\frac{1}{\lambda_0}\det\mathcal A_0$,
то характеристический многочлен имеет лишь два корня: $\lambda_0$ и $\tr\mathcal A_0 - \lambda_0$.

Наконец, если $\dim X > 3$, остается учесть собственное значение $0$


