\section{Возмущённый случай}

Рассмотрим возмущённую матрицу единиц
\( \almostallones{M}{N} \) размерности \( N \),
состоящую из \( N^2 - M \) единиц и \( M \) нулей.

Её можно представить в виде
\[
    \almostallones{M}{N} = \allones{N} - \perturbmatrix{M}{N}
    \]
Где \( \allones{N} \) --- матрица единиц,
а \( \perturbmatrix{M}{N} \) -- матрица-возмущение,
имеющая единицы в точности в тех местах,
на которых в \( \almostallones{M}{N} \) стоят нули.

\begin{prop}
    Подобные матрицы имеют одинаковый спектр
\end{prop}

Применяя полученную в \eqref{eq:diagtransform} матрицу преобразования,
получаем, что \( \almostallones{M}{N} \) подобна (а значит разделяет с ней общий спектр)
матрице
\[ A - B \]
Где
\[
    A = \transformmatrix{U}{\allones{N}}
    = \begin{pmatrix}
      N &   &        & \\
        & 0 &        & \\
        &   & \ddots & \\
        &   &        & 0
        \end{pmatrix}
        \]
\[
    B = \transformmatrix{U}{\perturbmatrix{M}{N}}
    \]
Снова, пропущенные элементы матрицы -- нулевые.

\subsection{Общая схема метода подобных операторов}
Следуя общей схеме, мы будем искать матрицу, подобную \( A - B \)
в виде
\[ A - J X \]
с матрицей подобия вида
\[ E + \Gamma X ,\]
где
\[ J, \Gamma \in \End{\mathfrak U} \]
\[ \mathfrak{U}\subset\matr{N}{} \]
--- линейные операторы (трансформаторы)
\footnote{\( \End{\mathfrak U} \) --- пространство непрерывных линейных операторов,
действующих на подпространстве \( \mathfrak U \) пространства матриц порядка \( N \)},
действующие на некотором \( \mathfrak U \),
называемом пространством допустимых возмущений,
и подбираемые в ходе решения так, чтобы матрица \( A - JX \)
имела как можно более простую структуру.

Иначе говоря, будем искать решение \( X \in\matr{N}{} \)
уравнения подобия
\begin{equation}\label{eq:similarity-orig}
    (A - B)(E+\Gamma X) = (E+\Gamma X) (A - JX).
\end{equation}

В подпространстве \( \mathfrak U \subset \matr{N}{} \)
мы будем запускать итерационный процесс,
и, как будет видно ниже, потребуется \( B \in \mathfrak U \)
и замыкания образов возникающих далее операторов также
должны содержаться в
пространстве допустимых возмущенний.
В виду способа построения матрицы \( B \), сказать что-либо о её структуре затруднительно,
поэтому мы вынуждены выбрать \( \mathfrak U = \matr{N}{}\).

Пространство \( \mathscr{X} \) представим в виде
декартова произведения
\( \mathscr{X} = \mathscr{X}_1~\times~\mathscr{X}_2 \)
одномерного пространства \( \mathscr{X}_1=\RR \)
и \((n-1)\)-мерного пространства \( \mathscr{X}_2 \),
элементы этого пространства в виде
\( x=\begin{pmatrix}x_1\\x_2\end{pmatrix}, \)
\( x_i\in\mathscr{X}_i \).
Матрицы из \( \matr{N}{} \) будем записывать
в блочном виде:
\[
    X = \begin{pmatrix}
    X_{11} & X_{12} \\
    X_{21} & X_{22}
    \end{pmatrix},
    \]
где \( X_{11}\in\RR \) --- число,
    \( X_{12} \) -- строка длины \( N-1 \),
    \( X_{21} \) -- столбец,
    \( X_{22} \in \matr{N-1}{} \).
При этом произведению матриц соответствует произведение блочных матриц.

Вернёмся к задаче.
Оператор \( \Gamma X \) определим как решение уравнения
\[
    A\Gamma X - (\Gamma X) A = X - JX.
    \marginnote{
        \footnotemark}
    \]
\footnotetext{в левой части стоит \( \ad{A}{\Gamma X} \)
--- оператор коммутирования с \(A\), применённый к \(\Gamma X\): \( \ad{Y}{Z} = YZ - ZY \)}
Пусть
\[ \Gamma X =
\begin{pmatrix}
    \Gamma_{11}(X) & \Gamma_{12}(X) \\
    \Gamma_{21}(X) & \Gamma_{22}(X)
\end{pmatrix}, \]
тогда
\[
    A \Gamma X - (\Gamma X) A =
    \frac{1}{N}
\begin{pmatrix}
    0               & \Gamma_{12}(X) \\
    -\Gamma_{21}(X) & 0
\end{pmatrix}
    . \]

Положим теперь
\[
    JX = \begin{pmatrix}
        X_{11} & 0 \\
        0      & X_{22}
    \end{pmatrix},
    \]
\[ \Gamma X =
    \begin{pmatrix}
        0       & X_{12} \\
        -X_{21} & 0
    \end{pmatrix}.
    \]
При этом матрица \( A - JX \) оказывается блочно-диагональной
и её спектр есть объединение спектров её диагональных блоков.
Заметим, естественнее было бы начать с выбора оператора \( J \) доставляющего структуру,
как можно более близкую к \( A \), а именно
оператор \( X\mapsto X_{11} \).
К сожалению, такой выбор \( J \) не позволяет решить
уже уравнение для \(\Gamma X\) во всём \(\mathfrak U = \matr{N}{}\),
и мы приходим к блочно-диагональному виду.

Развернём теперь уравнение \eqref{eq:similarity-orig}

\[
    A - B + A\Gamma X - B\Gamma X = A - JX + (\Gamma X) A - (\Gamma X) JX
    \]
\[
    A\Gamma X - (\Gamma X) A + JX = B \Gamma X + B - (\Gamma X) JX
    \]
\[
    X = B \Gamma X + B - (\Gamma X) JX
    \]
Применяя слева обратимый оператор \( J \) и подставляя результат обратно, получим уравнение
\begin{equation}\label{eq:similarity2}
    X = B \Gamma X + B - (\Gamma X) (J(B\Gamma X + B))
\end{equation}

Теперь мы докажем существование единственного решения уравнения \eqref{eq:similarity2}
и выведем оценки спектра,
показав, что в правой части этого уравнения стоит образ \( X \)
при сжимающем (в некотором шаре с центром в нуле) отображении \( \Phi \):
\[ \Phi(X) = B \Gamma X + B - (\Gamma X) (J(B\Gamma X + B)) \]

\subsection{Существование инвариантного множества}
Покажем сначала существование такого шара \( \{ X\in\matr{N}{}; \norm{X} \leq \varepsilon \} \),
который содержит свой образ при отображении \( \Phi \)
(другими словами, матрицы из этого шара
 не покидают его под действием \( \Phi \))

Будем считать, что алгебре матриц \( \matr{N}{} \)
введена некоторая суб-мультипликативная норма
\footnote{суб-мультипликативной называют норму, удовлетворяющую неравенству \( \norm{XY}\leq \norm{X}\norm{Y} \) при всех \( X, Y \) },
а в пространстве \( \End{\matr{N}{}} \) трансформаторов
используется обычная операторная норма%
\footnote{операторная норма
(матрицы или оператора) \( \mathcal A \)
вводится как точная нижняя грань таких констант \( C > 0 \),
что для всех \( x\in\mathscr X\) имеет место
(\( \norm{A x} \leq C \norm{x} \))}.
Тогда имеют место оценки

\begin{align*}
    \norm{\Phi(X)} &= \norm{B \Gamma X + B - (\Gamma X) (J(B\Gamma X + B))} \leq \\
%    &\leq \norm{B}\norm{\Gamma}\norm{X} + \norm{B} + \norm{\Gamma}\norm{X} (\norm{B}\norm{\Gamma}\norm{X}+\norm{B}) \leq \\
    &= \gamma^2\beta\norm{X}^2 + 2\gamma\beta\norm{X} + \beta,
\end{align*}
где \[ \beta=\norm{B}, \gamma=\norm{\Gamma} \]
Пока что отложим оценку этих величин.

\begin{lemma}
    Если
    \( \gamma\beta \leq \frac14, \)
    то \( \Phi \) отображает в себя шар
    \begin{equation}\label{def:omega}
        \Omega = \{ X\in\matr{N}{} ; \norm{X}\leq r_0 \},
    \end{equation}
    \[0 \leq r_0 = \frac{1 - 2\gamma\beta - \sqrt{1-4\gamma\beta}}{2\gamma^2\beta} < 4\beta, \]
\end{lemma}
\begin{proof}
    Обозначим \( r=\norm{X} \). Тогда
    \( \norm{\Phi(X)} \leq \norm{X} \), если
    \begin{equation}\label{eq:omega-invariance}
        \gamma^2\beta r^2 + (2\gamma\beta - 1) r + \beta \leq 0.
    \end{equation}
    Детерминант возникшего многочлена:
    \( \Delta = 1 - 4\gamma\beta\).
    Стало быть, если \( {\gamma\beta \leq \frac14} \),
    то определённое выше \( r_0 \) есть наименьшее \( r \),
    удовлетворяющее неравенству \eqref{eq:omega-invariance}

    Наконец, \( r_0 \geq 0 \):
    \begin{align*}
        1 - 2\gamma\beta - \sqrt{1-4\gamma\beta} > 0 \marginnote{\( 1-4\gamma\beta \geq 0 \)}\\
        \iff
        1-4\gamma\beta + 4\gamma^2\beta^2 > 1 - 4\gamma\beta \\
        \iff
        4\gamma^2\beta^2 \geq 0 \quad \text{что верно}
    \end{align*}
    И \( r_0 \leq 4\beta \):
    \begin{align*}
        r_0 \leq \varepsilon\beta \\
        \iff 1-2\gamma\beta-\sqrt{1-4\gamma\beta} \leq 2\varepsilon\gamma^2\beta^2 \\
        \iff 1 - 2\gamma\beta(1+\gamma\beta\varepsilon) \leq \sqrt{1-4\gamma\beta} \\
        \iff 1 - 4\gamma\beta(1+\gamma\beta\varepsilon)+4\gamma^2\beta^2(1+\gamma\beta\varepsilon)^2 \leq 1-4\gamma\beta \\
        \iff (1+\gamma\beta\varepsilon)^2\leq\varepsilon \marginnote{\(\gamma\beta\leq\frac14\)} \\
        \impliedby \frac{1}{16}\varepsilon^2 - \frac{1}{2} \varepsilon + 1 \leq 0 \\
        \iff \varepsilon=4 \\
    \end{align*}
\end{proof}

\subsection{Существование и единственность неподвижной точки}
Определим теперь, при каких условиях \( \Phi \) будет сжимающим в найдённом шаре.
\begin{lemma}
    Пусть \( \gamma\beta \leq \frac14, \)
    тогда \( \Phi \) есть сжимающее отображение \(\Omega\to\Omega\) шара \(\Omega\) в себя:
    \[ \norm{\Phi(X)-\Phi(Y)} \leq q \norm{X-Y}, \quad q<1\]
\end{lemma}
\begin{proof}
    Имеют место неравенства:
    \begin{align*}
        \Phi(X) - \Phi(Y) &= B\Gamma (X-Y) + \\
        &- \frac12 \Gamma (X + Y)J(B\Gamma(X - Y)) + \\
        &- \frac12 \Gamma (X - Y)J(B\Gamma(X + Y)),
        \marginnote{\footnotemark}
    \end{align*}
    \footnotetext{\(
        (\Gamma X)J(B\Gamma X) - (\Gamma Y)J(B\Gamma Y) =
        \frac12
        \left(
            (\Gamma X + \Gamma Y)(J(B\Gamma X) - J(B\Gamma Y)) +
            (\Gamma X - \Gamma Y)(J(B\Gamma X) + J(B\Gamma Y))
        \right) \)}
    \[
        \norm{\Phi(X) - \Phi(Y)} \leq (1 + \gamma\norm{X+Y})\beta\gamma \norm{X-Y} \leq q\norm{X-Y}.
        \]
    Здесь:
    \begin{equation}\label{eq:lipconst}
        q = (1+2\gamma r_0)\gamma\beta
        \leq (1+8\gamma\beta)\gamma\beta \leq \frac{3}{4}
        \marginnote{\footnotemark}
    \end{equation}
    \footnotetext{Здесь используются неравенства \(r_0\leq 4\beta\) и \(\gamma\beta \leq \frac14\)}
    Это и значит, что \( \Phi \) --- сжимающее отображение в найдённом шаре
\end{proof}

\begin{lemma}
    Пусть \( \norm{\cdot} \) --- какая-нибудь субмультипликативная норма в \( \matr{N}{} \),
    \( \beta=\norm{B}, \gamma=\sup_{\norm{X}=1}\norm{\Gamma X} \)
    и пусть
    \( \gamma\beta\leq\frac14. \)

    Тогда в \( \Omega \) существует единственное решение \( X^o \) уравнения
    \eqref{eq:similarity2} \( X=\Phi(X) \),
    при этом имеют место оценки:
    \[
        \norm{X^o} \leq r_0 \leq 4\beta
        \]
    \[
        \norm{X^k - X^o} \leq \frac{3^k}{4^{k-1}}\beta
        \]
\end{lemma}
\begin{proof}
    \( \Omega \) --- замкнутое подмножество банахова пространства \( \matr{N}{} \).
    \( \Phi: \Omega\to\Omega \) -- сжимающее отображение.
    По теореме Банаха о неподвижной точке, в этом шаре существует единственное решение \( X^o \)
    уравнения \( X = \Phi (X) \),
    доставляемое методом простых итераций (c нулём в качестве начального приближения),
    как предел сходящейся последовательности
    \[
        \{X_k = \Phi^k(0); k\in\NN\}
        \]
    Грубейшая верхняя оценка его нормы --- радиус шара \( \Omega \).
    При этом \[ \norm{\Phi^k(0) - X^o} \leq \frac{q^k}{1-q} \norm{\Phi(0) - 0} = \frac{q^k}{1-q}\norm{B} \]
\end{proof}

\subsection{Оценка спектра}
Итак, если \( \gamma\beta \leq \frac14 \),
то существует единственное решение \( X^o \) уравнения
\( (A-B)(I+\Gamma X) = (I+\Gamma X)(A-JX) \),
или, другими словами, существует такая матрица
\( X^o =
\begin{pmatrix}
    x_{11} & X_{12} \\
    X_{21} & X_{22}
\end{pmatrix}
\),
удовлетворяющая \( \norm{X^o} \leq r_0 \leq 4\norm{B} \),
что матрица \( A-B \) подобна блочно-диагональной матрице \( A - JX^o \):
\[
    A-B \sim
    \begin{pmatrix}
        N - x_{11} & 0 \\
        0          & -X_{22}^o
    \end{pmatrix},
    \]
При этом спектр матрицы \( A - JX^o \) есть объединение спектров
её диагональных блоков:
\[ \spec{A-B} = \left\{N - x_{11}\right\} \cup \spec{-X_{22}}. \]

\begin{lemma}
    Пусть \( \norm{\cdot} \) --- норма в \( \matr{N}{} \),
    согласованная с какой-нибудь нормой в пространстве \( \mathscr{X} \) векторов,
    \( \beta = \norm{B}, \gamma=\sup_{\norm{X}=1} \norm{\Gamma X} \)
    и пусть верно строгое неравенство \( \gamma\beta < \frac14 \).

    Тогда спектр \( A - B \)
    есть объединение неперсекающихся одноточечного вещественного множества \( \sigma_1 \)
    и множества \( \sigma_2 \),
    лежащих в шарах радиуса \( 4\beta \)
    с центрами в \( N \) и \( 0 \) соответственно:

    \[ \spec{A-B} = \sigma_1 + \sigma_2 \]
    \[ \sigma_1 = \left\{\lambda_1\right\} \subset \left\{\lambda\in\RR; \lvert \lambda - N\rvert < 4\beta\right\} \]
    \[ \sigma_2 \subset \left\{\lambda\in\CC; \lvert\lambda\rvert <4\beta \right\} \]
\end{lemma}
\begin{proof}
    Согласованная норма субмультипликативна, поэтому применима предыдущая лемма.
    Согласованность также означает, что для любой матрицы \( Z \)
    и любого её нормированного собственного вектора \( h_j \),
    отвечающего собственному значению \( \mu_j \),
    имеет место \({ \normex{v}{A h_j} \leq \norm{A} }\),
    т.е. \({ \lvert \mu_j \rvert \leq \norm{A} }\).

    Значит \( \lvert\lambda_1\rvert = \lvert N - x_{11} \rvert \leq r_0 < 4\beta \)
    и \( \spr{-X_{22}} \leq r_0 < 4\beta  \)
\end{proof}


Доказательство теоремы \ref{thm:almostallones-spectra}
состоит в выборе подходящих норм.
\begin{proof}
    Напомню, \( B = U^{-1} \perturbmatrix{M}{N} U \)
    получена из матрицы возмущения с \( M \) единицами применением преобразования \eqref{eq:diagtransform},
    диагонализирующего матрицу единиц,
    а \( \Gamma \) --- оператор из \( \End{\matr{N}{}} \),
    определённый формулой
    \( \Gamma X = \frac1N \begin{pmatrix}0 & X_{12} \\ -X_{21} & 0\end{pmatrix} \).

        Рассмотрим в \( \mathscr{X} \) обычную \(p\)-норму:
    \( \normex{v}{x} = \left(\sum_j \lvert x_j \rvert^p \right)^{1/p} \),
    и аналогично положим в пространстве \( \matr{N}{} \)
    \( \normex{m}{Z} = \left( \sum_j {\normex{v}{Z e_j}}^p \right)^{1/p} \),
    где \( \{ e_j \} \) --- канонический базис в \( \mathscr{X} \).
    Не трудно убедиться, что \( \normex{m}{\cdot} \) согласована с \( \normex{v}{\cdot} \).

    Тогда оказывается:
    \[ \gamma = \frac1N, \]
    \[  \beta = M^{1/p} ,\]
    где \( M \) --- число единиц в матрице \( \perturbmatrix{M}{N} \) возмущения
    (число нулей в исходной матрице \( \almostallones{M}{N} \)).

    Подробнее:
    \[ \gamma = \frac1N \sup\normex{v}{\begin{pmatrix}0 & X_{12} \\ -X_{21} & 0\end{pmatrix}} = \frac1N \]

    Остаётся заметить, что умножение на унитарную матрицу \( U \)
    (и \( U^{-1} \)) есть изометрия
    в \( \matr{N}{} \)
    и поэтому
    \[ \beta = \normex{m}{B} = \normex{m}{\perturbmatrix{M}{N}} = \left(\sum_j{\normex{v}{\perturbmatrix{M}{N}e_j}}^p\right)^{1/p}.\]
    Ясно, что \( {\normex{v}{\perturbmatrix{M}{N}e_j}}^p \) --- это суть число единиц в \( j \)-ом столбце
    матрицы \( \perturbmatrix{M}{N} \).
    Тогда \( \beta \) это суть корень степени \( p \) из числа \( M \) единиц в \( \perturbmatrix{M}{N} \).

    Значит, если
    \( M^{1/p} < \frac{N}{4} \), т.е. \( M < \left(N/4\right)^p \),
    то выполняются условия леммы,
    причём \( r_0 < 4M^{1/p} \).
\end{proof}
