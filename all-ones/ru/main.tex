\documentclass[12pt,a4paper]{article}
\usepackage{amsmath}
\usepackage{amsfonts}
\usepackage{mathrsfs}
\usepackage{mathtools}

\usepackage{fontspec}
\usepackage{unicode-math}
\setmainfont{CMU Serif}
\setmathfont{XITS Math}
\usepackage{polyglossia}
\setdefaultlanguage{russian}
\setotherlanguage{english}


\usepackage{amsthm}
\theoremstyle{definition}
\newtheorem{dfn}{Определение}
\theoremstyle{plain}
\newtheorem{thm}{Теорема}
\newtheorem*{thm*}{Теорема}
\newtheorem{lemma}{Лемма}
\theoremstyle{remark}
\newtheorem{prop}{Утверждение}
\newtheorem{corollary}{Следствие}
\newtheorem{conjecture}{Предположение}

% Mimic vestnik' style
\usepackage{indentfirst}
\oddsidemargin=-3mm
\renewcommand{\baselinestretch}{1.2}
%\evensidemargin=-3mm %
\textwidth 163mm %
\textheight220mm %
\topmargin-8mm %
\sloppy
\let\thefootnote\relax

\usepackage{marginnote}
\usepackage{hyperref}
\usepackage{tabularx}

\begin{document}
\setcounter{thm}{0}
\setcounter{lemma}{0}
\quad УДК 517.984.3 : 519.177

\begin{center}
% The whole point of LaTeX (as opposed to e.g. ConTeXt)
% is to not care about layout and typesetting in the document itself.
% Instead, macro's and environments are used,
% which hide all the details and let you enforce a consistent style
% throughout the whole document by simply modifying
% these environments' definitions.
%
% So the publisher may just ask authors
% ``please, put theorems in `thm` environments (which we'll provide)''.
% This way author does not suffer from useless verbosity
% and an editor can easily adapt an article for publication.
\vskip0.5cm
\textbf{СПЕКТР МАТРИЦЫ СМЕЖНОСТЕЙ ПОЧТИ ПОЛНОГО ОРГРАФА}\\
\vskip0.5cm
\textbf{
    \textbf{Козлуков С.В.\footnotemark} \\
    \textit{<rerumnovarum@openmailbox.org>} \\
    \textit{Воронежский Государственный Университет}
}
\footnotetext{
    \copyright Козлуков С.В.
}
\end{center}

\addcontentsline{tos}{subsection}{{\it{Козлуков С.в.}} Спектр матрицы смежностей почти полного орграфа}
\addcontentsline{toc}{subsection}{{\it{Sergey Kozlukov}} Spectra of an almost-all-ones matrix}

\begin{quote}
    \small{{\bf Аннотация.}
    В статье методом подобных операторов найдены оценки собственных значений
    матриц смежностей почти полных орграфов.
    }

    \textbf{Ключевые слова:}
    \small{метод подобных операторов, собственные значения, спектр графов}
\end{quote}

\begin{quote}
    \small{{\bf Abstract.}
    This article considers spectra of almost-complete digraphs.
    Their eigenvalues estimates are derived using The Method Of Similar Operators.
    }

    \textbf{Keywords:}
    \small{The Method Of Similar Operators, eigenvalues, graph spectra}
\end{quote}

\begin{center}{1. Введение и основной результат}\end{center}
Пусть \( X \) --- \( M \)-мерное ( \( M < \infty \) ) линейное пространство,
\( A: X\to X \) --- обратимый линейный оператор простой структуры:
\( A \) имеет ровно \( M \) линейно-независимых собственных векторов \( e_1,~\ldots,~e_M \),
которым отвечают, вообще говоря не все различные, ненулевые собственные значения
\( \lambda_1,~\ldots,~\lambda_M \) (см. \cite{baskakov-algebra}).

Рассмотрим оператор
\( \mathbb{A}: X^N\to X^N \), действующий на пространстве \( X^N \) по формуле
\[ \mathbb{A}x =
    \begin{pmatrix}
        A & \cdots & A \\
        \vdots & \ddots & \vdots \\
        A & \cdots & A
    \end{pmatrix}
    \begin{pmatrix}
        x_1 \\
        \vdots \\
        x_N
    \end{pmatrix}
    = \begin{pmatrix}
        A \sum_{i=1}^N x_i \\
        \vdots \\
        A \sum_{i=1}^N x_i
    \end{pmatrix},
    \quad x=(x_1,\ldots,x_N) \in X^N. \]

В статье даны спектр и жорданов базис для операторов такого вида,
а также уточнения для случая блочных матриц, составленных из самосопряжённых блоков.

\begin{center}{2. Доказательство}\end{center}
Доказательство состоит в построении уравнения для матрицы, подобной \( \mathscr{A}_{MN} \),
 но устроеной ``проще''. Решение возникающего нелинейного уравнения
 в банаховой алгебре \( \mathtt{Matr}_N\mathbb{C} \)
 доставляется методом простых итераций (см. \cite{baskakov-harmonic}).

Подобие матриц \( \mathcal{A}_1, \mathcal{A}_2 \)
 понимается в смысле существования обратимой матрицы \( \mathcal{U} \),
 такой что \( \mathcal{A}_1 \mathcal{U} = \mathcal{U} \mathcal{A}_2 \).
Спектры подобные матриц совпадают.

Провед\"ем предварительные преобразования.

\begin{lem}
    Матрица единиц 
    \( \mathcal{J}_N =
    \begin{pmatrix}
        1 & \cdots & 1 \\
        \vdots & \ddots & \vdots \\ 
    1 & \cdots & 1 \end{pmatrix} \),
    подобна матрице
    \[
        \mathcal{A} = \begin{pmatrix}
            N & 0 & \cdots & 0 \\
            0 & 0 & \cdots & 0 \\
            \vdots & \vdots & \ddots & \vdots \\
            0 & 0 & \cdots & 0 \end{pmatrix}. \]
    Точнее, существует ортогональная матрица \( \mathcal{U} \),
    такая что
    \( \mathcal{A} = \mathcal{U}\mathcal{J}_N \mathcal{U}^{-1} \).
\end{lem}
\begin{proof}
    Собственному значению \( N \) матрицы \( \mathcal{J}_N \) 
    соответствует собственный вектор \( f_1 = {\left(1,\ldots,1\right)} \),
        а собственному значению \( 0 \) соответствует \( N-1 \) независимый собственный вектор
        \( f_2 = {\left(1,-1,0,\ldots,0\right)}, ...,
           f_N = {\left(0,\ldots,0,1,-1\right)} \).
    Применив ортогонализацию Грамма-Шмидта, получим ортонормальную систему \( h_1, \ldots, h_N \).
    В качестве матрицы \( \mathcal{U} \) выберем матрицу,
    имеющую столбцами векторы \( h_1, \ldots, h_N \):
    \[ \mathcal{U} =
    \begin{pmatrix}
        \frac{1}{\sqrt N} &  \frac{1}{\sqrt2} &  \frac{1}{\sqrt{6}} &   \alpha_3 & \cdots & \alpha_{N-1} \\
        \frac{1}{\sqrt N} & -\frac{1}{\sqrt2} &  \frac{1}{\sqrt{6}} &   \alpha_3 & \cdots & \alpha_{N-1} \\
        \frac{1}{\sqrt N} & 0                 & -\frac{2}{\sqrt{6}} &   \alpha_3 & \cdots & \alpha_{N-1} \\
        \frac{1}{\sqrt N} & 0                 &  0                  & -3\alpha_3 & \cdots & \alpha_{N-1} \\
        \frac{1}{\sqrt N} & 0                 &  0                  & 0          & \cdots & \alpha_{N-1} \\
        \vdots    & \vdots            &  \vdots             & \vdots     & \ddots & \vdots  \\
        \frac{1}{\sqrt N} & 0                 &  0                  & 0          & \cdots & \alpha_{N-1} \\
        \frac{1}{\sqrt N} & 0                 &  0                  & 0          & \cdots & -(N-1)\alpha_{N-1}
    \end{pmatrix},\]
    \[
        \alpha_1 = \frac{1}{\sqrt{2}}, \]
    \[
        \alpha_k = \frac{1}{k\sqrt{k(k-1)^2 \alpha_{k-1}^4 + 1}}, \quad k=\overline{2,N-1}.\]
\end{proof}

Таким образом, исходная матрица \( \mathcal{A}_{MN} \) подобна матрице
\( \mathcal{A} - \mathcal{B} \), где \( \mathcal{B} = \mathcal{U} \mathscr{B}_{MN} \mathcal{U}^{-1} \).

% \clearpage
Матрицы из \( \mathbb{C}^{N{\times}N} \) будем записывать в~блочном виде
\( X \sim
    \begin{pmatrix}
    x_{11} & X_{12} \\
    X_{21} & X_{22}
    \end{pmatrix}, \)
    где \( x_{11} \)~--- число,
    \( X_{12} \)~--- строка, \( X_{21} \)~--- столбец,
    \( X_{22} \)~--- квадратный блок размерности \( N-1 \).
Такие блочные матрицы сами образуют алгебру, изоморфную исходной
и~их можно естественным образом умножать
на элементы пространства \( \mathbb{C}\times\mathbb{C}^{N-1} \),
изоморфного~\( \mathbb{C}^N \):
\[
    \begin{pmatrix}
    x_{11} & X_{12} \\
    X_{21} & X_{22}
    \end{pmatrix}
    \begin{pmatrix} x_1 \\ x_2 \end{pmatrix}
  = \begin{pmatrix}
      x_{11} x_1 + X_{12} x_2 \\
      X_{21} x_1 + X_{22} x_2
      \end{pmatrix},\quad x \sim \begin{pmatrix} x_1 \\ x_2 \end{pmatrix}\in \mathbb{C}\times\mathbb{C}^{N-1}.
    \]
В~дальнейших выкладках изоморфные объекты понимаются взаимозаменяемыми.

Следуя общей схеме метода подобных операторов~\cite{baskakov-harmonic,baskakov1983},
будем искать более ``простую'' матрицу, подобную \( \mathcal{A} - \mathcal{B} \),
в~виде \( \mathcal{A} - \mathfrak{J} X \)
с~матрицей преобразования подобия \( E + \Gamma X \),
где \( E\in{\mathbb{C}^{N{\times}N}} \)~--- единичная матрица,
\( \mathfrak{J},\Gamma{:}\ \mathbb{C}^{N{\times}N}{\to}\mathbb{C}^{N{\times}N} \)~--- линейные операторы,
действующие на алгебре \( \mathbb{C}^{N{\times}N} \), подбираемые
в~ходе решения,
      прич\"ем \( \mathfrak{J} \) --- проектор (\(\mathfrak{J}^2=\mathfrak{J}\)),
      ``упрощающий'' возмущение \( \mathfrak{J}X \),
      а \( \Gamma \)
      при всех \( X\in {\mathbb{C}^{N{\times}N}} \)
      удовлетворяет уравнению
          \( \mathcal{A}\Gamma X - (\Gamma X) \mathcal{A} = X - \mathfrak{J}X. \)

\begin{lem}
    В качестве \( \mathfrak{J} \)
        естественно выбрать оператор блочной диагонализации:
    \[
        \mathfrak{J} X = \begin{pmatrix} x_{11} & 0 \\ 0 & X_{22} \end{pmatrix}. \]
    При этом:
    \[
        \Gamma X = \frac{1}{N} \begin{pmatrix} 0 & X_{12} \\ -X_{21} & 0 \end{pmatrix}, \]
        для \( X\sim \begin{pmatrix}x_{11} & X_{12} \\ X_{21} & X_{22}\end{pmatrix} \in \mathbb{C}^{N{\times}N} \)
    и имеет место равенство
    \[
        \gamma = \frac1N
                \sup_{{\left\|X\right\|}_{F}=1}{\left\|\begin{pmatrix}0 & X_{12} \\ -X_{21} & 0\end{pmatrix}\right\|}_{F}
                = \frac1N. \]

\end{lem}
\begin{crl}
    Спектр блочно-диагональной матрицы
    \( \mathcal{A} - \mathfrak{J}X = \begin{pmatrix} N - x_{11} & 0 \\ 0 & -X_{22} \end{pmatrix} \)
    есть объединение спектров е\"е диагональных блоков:
    \[
        \sigma(\mathcal{A} - \mathfrak{J} X) = \{ N - x_{11} \} \cup \sigma(-X_{22}). \]
\end{crl}
\begin{proof}
Пусть \( \Gamma \) действует по формуле
\( \Gamma X = \begin{pmatrix} \Gamma_{11}(X) & \Gamma_{12}(X) \\
                              \Gamma_{21}(X) & \Gamma_{22}(X)
                              \end{pmatrix} \), тогда
\[
    \mathcal{A} \Gamma X - (\Gamma X)\mathcal{A} = 
    \begin{pmatrix} 0 & N\Gamma_{12}(X) \\
        - N\Gamma_{21}(X) & 0
        \end{pmatrix}, \]
и~уравнение для \( \Gamma X \) сводится~к
\[
    X - \mathfrak{J} X =
    N \begin{pmatrix} 0 & \Gamma_{12}(X) \\
        - \Gamma_{21}(X) & 0
        \end{pmatrix}. \]

Значит, \( \mathfrak{J} \) может обнулить вс\"е,
    кроме двух диагональных блоков размеров \( 1\times 1 \)
    и \( (N-1)\times(N-1) \),
поэтому для \( X =
    \begin{pmatrix}
    x_{11} & X_{12} \\
    X_{21} & X_{22}
    \end{pmatrix} \in \mathbb{C}^{N{\times}N} \):
\[
    \mathfrak{J} X = \begin{pmatrix} x_{11} & 0 \\ 0 & X_{22} \end{pmatrix}, \]
\[
    \Gamma X = \frac{1}{N}\begin{pmatrix} 0 & X_{12} \\ -X_{21} & 0 \end{pmatrix}. \]
\end{proof}

Теперь выпишем уравнение подобия матриц \( \mathcal{A} - \mathcal{B} \)
и \( \mathcal{A} - \mathfrak{J} X \):
\begin{equation}\label{kozlukovsv:eq:similarity}
    A(E+\Gamma X) = (E+\Gamma X)(\mathcal{A} - \mathfrak{J} X), \quad X\in\mathbb{C}^{N{\times}N}.
\end{equation}
\begin{lem}
    Уравнение~\eqref{kozlukovsv:eq:similarity} эквивалентно уравнению
    \begin{equation}\label{kozlukovsv:eq:fixptn}
        X = \mathcal{B} \Gamma X + \mathcal{B} - (\Gamma X)(\mathfrak{J}(\mathcal{B} (E + \Gamma X))), \quad X\in\mathbb{C}^{N{\times}N}.
    \end{equation}
\end{lem}
\begin{proof}
Раскрывая скобки, уравнение~\eqref{kozlukovsv:eq:similarity} можно преобразовать к виду
\begin{equation}\label{kozlukovsv:eq:fixptn-ini}
    X = \mathcal{B} \Gamma X + \mathcal{B} - (\Gamma X) \mathfrak{J} X.
\end{equation}
Пусть для \( X \) выполнено~\eqref{kozlukovsv:eq:fixptn-ini}.
Тогда
    \begin{equation}\label{kozlukovsv:eq:jx}
        \mathfrak{J} X = \mathfrak{J}(\mathcal{B} (E + \Gamma X)).
    \end{equation}
Подставляя это выражение обратно в~\eqref{kozlukovsv:eq:fixptn-ini}
    получим~\eqref{kozlukovsv:eq:fixptn}.
Обратно, применяя \( \mathfrak{J} \) к обеим частям уравнения~\eqref{kozlukovsv:eq:fixptn},
    получаем~\eqref{kozlukovsv:eq:jx} и~\eqref{kozlukovsv:eq:fixptn-ini}.
\end{proof}

Выражение в правой части уравнения~\eqref{kozlukovsv:eq:fixptn} обозначим как
\[
    \Phi(X) = \mathcal{B} \Gamma X + \mathcal{B} - (\Gamma X)(\mathfrak{J}(\mathcal{B} (E + \Gamma X))).\]
Теперь покажем, что, при определ\"енных условиях,
возникшее нелинейное отображение
\( \Phi{:}\ \mathbb{C}^{N{\times}N}{\to}\mathbb{C}^{N{\times}N} \) имеет инвариантным множеством
некоторый шар \( \Omega \subset \mathbb{C}^{N{\times}N} \) с~центром в~нуле
(т.е.~\( \Phi(\Omega)\subset\Omega \)),
на котором оно является сжимающим.

Пусть в~\( \mathbb{C}^{N{\times}N} \)
выбрана какая-нибудь субмультипликативная норма \( \|\cdot\| \)
(т.е.~норма, удовлетворяющая неравенству
 \( \| \mathcal{A}_1\mathcal{A}_2 \| \leq \|\mathcal{A}_1\|\|\mathcal{A}_2\| \)
 при всех \( \mathcal{A}_1, \mathcal{A}_2 \in \mathbb{C}^{N{\times}N} \)).
В пространстве \( L(\mathbb{C}^{N{\times}N}) \)
  линейных преобразований матриц размера \( N{\times}N \)
  будем рассматривать операторную норму
  \[
      \|\Psi\|_{\mathrm{op}} = \sup_{\|X\|=1,\ X\in\mathbb{C}^{N{\times}N}} \|\Psi X\|,\ \Psi\in L(\mathbb{C}^{N{\times}N})
      \]
Нам нужно найти такой радиус \( r \geq 0 \),
что при \( \|X\|,\|Y\| \leq r \) выполнялись бы неравенства \( \|\Phi(X)\| \leq r \)
и~\( \|\Phi(X) - \Phi(Y)\| < q\|X-Y\| \), \( q\in(0,1) \).
Обозначим
\( \beta = \|\mathcal{B}\| \), \( \gamma = \|\Gamma\|_{\mathrm{op}} = \sup_{\|X\|=1} \|\Gamma X\| \).

\begin{lem}
    Пусть \( \gamma\beta < \frac14\),
    тогда шар
    \[
        \Omega = \left\{ X\in \mathbb{C}^{N{\times}N}; \|X\| \leq r_0 \right\}, \]
    \[  0 < r_0 = \frac{1 - 2\gamma\beta - \sqrt{1-4\gamma\beta}}{2\gamma^2\beta} < 4\beta, \]
    удовлетворяет условию \( \Phi(\Omega)\subset\Omega \).
\end{lem}
\begin{proof}
Очевидно неравенство
    \[ \| \Phi(X) \| \leq
     \beta \gamma^2 \|X\|^2 + 2\beta\gamma\|X\| + \beta. \]
Значит, если \( r \) удовлетворяет неравенству
    \begin{equation}\label{kozlukovsv:ineq:invariance-radius}
        \beta \gamma^2 r^2 + (2\beta\gamma - 1)r + \beta \leq 0,
    \end{equation}
    то \( \|\Phi(X)\| \leq r \) при всех \( \|X\| \leq r \).
Если \( \gamma\beta \leq \frac14 \),
    то дискриминант \( \Delta = 1-4\gamma\beta \)
    соответствующего уравнения положителен и~его корни вещественны.
Из знаков коэффициентов возникшего многочлена видно, что оба корня положительны.
Следовательно, наименьший положительный \( r \),
    удовлетворяющий неравенству~\eqref{kozlukovsv:ineq:invariance-radius}
    есть наименьший корень
    соответствующего уравнения:
    \[ r_0 = \frac{1 - 2\gamma\beta - \sqrt{1-4\gamma\beta}}{2\gamma^2\beta}. \]
Учитывая \( \gamma\beta<\frac14 \), имеем \( r_0 < 4\beta \).
\end{proof}

Аналогичным образом устанавливается
\begin{lem}
    Пусть \(\gamma\beta<\frac14\),
    тогда \( \Phi \)~--- сжимающее отображение:
    \[ \| \Phi(X) - \Phi(Y) \| \leq q \|X - Y\|, \quad X,Y\in\Omega \]
    \[ q = (1+2\gamma r_0) \gamma\beta \leq (1+8\gamma\beta)\gamma\beta \leq \frac34. \]
\end{lem}
\begin{proof}
    \begin{align*} \| \Phi(X) - \Phi(Y) \| = \| \mathcal{B}\Gamma (X-Y) + (\Gamma X)(\mathcal{B}\Gamma X + \mathcal{B})
     - (\Gamma Y)(\mathcal{B} \Gamma Y + \mathcal{B}) \| \leq \\
        \leq
     \beta\gamma\|X-Y\| +
     \beta \gamma^2 \|X-Y\| \|X+Y\| \leq \\
        \leq
     \beta\gamma\|X-Y\| +
     2 r_0 \beta \gamma^2 \|X-Y\|.
    \end{align*}
Здесь использовано равенство
\[ (\Gamma X) \mathfrak{J}(\mathcal{B}\Gamma X) - (\Gamma Y) \mathfrak{J}(\mathcal{B}\Gamma Y) =
    \frac12\left[
        \Gamma(X-Y) \mathfrak{J}(\mathcal{B}\Gamma(X+Y))
    +   \Gamma(X+Y) \mathfrak{J}(\mathcal{B}\Gamma(X-Y))
    \right]. \]
\end{proof}

Отсюда и~из теоремы Банаха о~неподвижной точке следует:
\begin{lem}
В~шаре \[ \Omega = \left\{ X\in\mathbb{C}^{N{\times}N}; \quad \|X\| \leq r_0 \right\} \]
    существует и~при том единственное решение \( X^o \) уравнения~\eqref{kozlukovsv:eq:fixptn},
    являющееся пределом последовательности \( \{ \Phi^k(0); k\in\mathbb{N} \} \),
    где \( \Phi^k = \Phi\circ\Phi^{k-1} \)~--- композиция.
\end{lem}

\begin{crl}
    Матрица \( \mathcal{A} - \mathcal{B} \) подобна блочно-диагональной матрице \( \mathcal{A} - \mathfrak{J} X^o \):
    \[ \mathcal{A} - \mathcal{B} \sim
    \begin{pmatrix}
    N - x_{11}^o & 0 \\
    0 & -X_{22}^o
    \end{pmatrix}, \]
        при этом выполняются условия:
    \[ \sigma\left(\mathcal{A} - \mathcal{B}\right) = \left\{N-x_{11}^o\right\}\cup \sigma\left(-X_{22}^o\right), \]
        \[ x_{11}^o\in\mathbb{R}, \lvert x_{11}^o \rvert < r_0 \leq 4\beta, \]
    \[ \sigma\left(-X_{22}^o\right) \subset \{ \mu\in\mathbb{C}; \lvert x \rvert < r_0 \leq 4\beta \}. \]

    Собственное значение \( N - x_{11}^o \)
    совпадает со спектральным радиусом матрицы \( A \)
    и ему отвечает собственный вектор
    \[ 
        w = \mathcal{U}(E + \Gamma X^o)\begin{pmatrix}1\\0\end{pmatrix},
        \]
    \[
        \|w - \mathcal{U}\begin{pmatrix}1\\0\end{pmatrix}\|_2 \leq \frac{4\beta}{N}.
        \]
\end{crl}
\begin{proof}
    Матрица \( \mathcal{A} - \mathcal{B} \) подобна блочно-диагональной \( \mathcal{A} - \mathfrak{J} X^o \),
    поэтому их спектры совпадают.
    Спектр матрицы \( \mathcal{A} - \mathfrak{J} X^o \) есть объединение спектров е\"е диагональных блоков.
    В~виду субмультипликативности нормы имеют место неравенства
    \[ \mathtt{spr}(X^o) = \max_{\lambda\in\sigma(X^o)}\lvert\lambda\rvert \leq \|X^o\| \leq r_0. \]
    Кроме того, собственное значение \( x_{11}^o \) является вещественным, как предел сходящейся вещественной последовательности.

    Пусть теперь \( v \) --- собственный вектор матрицы \( \mathcal{A}-\mathfrak{J}X^o \):
    \[
        (\mathcal{A} - \mathfrak{J}X^o) v = \lambda v.
        \]
    Тогда
    \[
        (E+\Gamma X^o)^{-1}\mathcal{U}^{-1} A \mathcal{U} (E+\Gamma X^o) v = (\mathcal{A} - \mathfrak{J}X^o)v = \lambda v,
        \]
    \[
        A \underbrace{\mathcal{U} (E+\Gamma X^o) v}_{w} = \lambda \underbrace{\mathcal{U} (E+\Gamma X^o) v}_{w},
        \]
    при этом
    \[
        \|w - \mathcal{U}v\|_2 = \|\mathcal{U}\Gamma X^o v\|_2 \leq
        \|\Gamma\|_{\mathrm{op}} \|X^o\|_{\mathrm{op}} \|\mathcal{U} v\|_2 \leq
        \frac{4\beta}{N} \|v\|.
        \]
    Здесь использована ортогональность матрицы \( \mathcal{U} \).
    Наибольшему собственному значению \( N - x_{11}^o \)
        матрицы \( \mathcal{A} - \mathfrak{J}X \) соответствует
        собственный вектор \( \begin{pmatrix}1\\ 0\end{pmatrix} \).
    Подставляя \( v = \begin{pmatrix}1\\ 0\end{pmatrix} \)
            получим желаемый результат.
\end{proof}

Доказательство основной теоремы (стр. \pageref{thm:almostallones-spectra})
состоит в выборе подходящей субмультипликативной нормы.
Матрица \( B = U^{-1} \perturbmatrix{M}{N} U \)
получена из матрицы возмущения с \( M \) единицами
ортогональным преобразованием подобия \eqref{eq:diagtransform},
диагонализирующим матрицу единиц.
\( \Gamma \) --- оператор, действующий в \( \matr{N}{} \)
по формуле
\( { \Gamma X = \frac1N \begin{pmatrix}0 & X_{12} \\ -X_{21} & 0\end{pmatrix} } \).

Рассмотрим в пространстве \( \matr{N}{} \)
норму Фробениуса \( \normex{F}{\cdot} \),
определённую формулой
\( \normex{F}{X} = \sqrt{\sum_{ij} \lvert x_{ij}\rvert^2}. \)
Эта норма субмультипликативна.

Тогда:
\[ \gamma = \frac1N
            \sup_{\normex{F}{X}=1}\normex{F}{\begin{pmatrix}0 & X_{12} \\ -X_{21} & 0\end{pmatrix}}
          = \frac1N,
    \]
и, так как умножение на унитарную матрицу \( U \)
    (и \( U^{-1} \)) есть изометрия в \( \matr{N}{} \),
    а \( \perturbmatrix{M}{N} \) состоит из \( M \) единиц:
\[
    \beta = \normex{F}{B} =
    \normex{F}{\perturbmatrix{M}{N}} = \sqrt{M},
    \]

Значит, если
\( \sqrt{M} < \frac{N}{4} \), т.е.
\( M < \frac{N^2}{16} \),
то выполняются условия леммы,
причём \( r_0 < 4\sqrt{M} \).

% A nice bibtex' include was here,
% which used to generate a proper bibliography
% enforcing a single and consistent style;
% In these times of darkness we're not allowed
% to use macros or appropriate package
% or any other kind of generic solution
% --- you hardcode, so that you can't maintain
\begin{thebibliography}{9}
\bibitem{baskakov-harmonic}  Баскаков~А.~Г. Гармонический анализ линейных операторов
    / Баскаков~А.~Г.
    --- Воронеж : Издательство Воронежского Государственного Университета,
        1987.
    ---  с.~93--121.
\bibitem{epidemic-eigenvalues}  Yang Wang and D. Chakrabarti and Chenxi Wang and C. Faloutsos.
    Epidemic spreading in real networks: an eigenvalue viewpoint
        --- 22nd International Symposium on Reliable Distributed Systems, Oct 2003. Proceedings., --- pages~25-34.
\end{thebibliography}


\end{document}
