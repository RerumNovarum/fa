\documentclass[11pt,a4paper,twoside]{article}

\usepackage{vvmph2}
\usepackage{rumathbr}
\usepackage{mathrsfs}

%\usepackage{marginnote}
%\usepackage{hyperref}
%\usepackage{tabularx}

\begin{document}

\udk{517.984.3 : 519.177}
\bbk{22.161}
\art{СПЕКТР МАТРИЦЫ СМЕЖНОСТЕЙ ПОЧТИ ПОЛНОГО ОРГРАФА}{ON SPECTRUM OF AN ADJACENCIES MATRIX OF ALMOST-COMPLETE GRAPH}

\fio{Козлуков}{Сергей}{Викторович}
\pos{Студент}{Student}
\emp{Воронежский Государственный Университет}{Voronezh State University}
\email{rerumnovarum@openmailbox.org}
\addr{Университетская ул., 1, 394000, г. Воронеж, Российская Федерация}{Universitetskaya St, 1, 394000, Voronezh, Russian Federation}

\maketitle

\begin{abstract}
    С помощью метода подобных операторов изучаются спектральные свойства
    матриц смежностей близких к полным ориентированных графов с петлями.
    Даны оценки собственных значений таких матриц.
\end{abstract}

\keywords{Метод Подобных Операторов,
          собственные значения,
          спектр графа}{
          The Similar Operators Method,
          eigenvalues,
          graphs spectra}

\section{Введение и основной результат}
\section{Введение}

Пусть \( \almostallones{M}{N} \) --- матрица порядка \( N \),
состоящая из \( N^2 - M \) единиц и \( M \) нулей
и пусть число нулей невелико.
Матрица такого вида (рассматриваемая, как матрица смежностей) соответствует
полному орграфу, с небольшим числом ``удалённых'' рёбер.
В статье получена оценка спектра
матриц такого вида.
Здесь под спектром \( \spec{A} \) матрицы \( A \)
понимается множество таких \( \lambda\in\CC \), для которых
матрица \( A - \lambda E \), необратима (\( E \) --- единичная матрица).

Матрица \( \almostallones{M}{N} \) представима в виде
\[
    \almostallones{M}{N} = \allones{N} - \perturbmatrix{M}{N},
    \]
где \(
\allones{N} =
\begin{pmatrix}1 & \cdots & 1 \\
\vdots & \ddots & \vdots \\
1 & \cdots & 1
\end{pmatrix} \) --- матрица единиц,
а \( \perturbmatrix{M}{N} \) имеет единицы в точности на тех \( M \)
местах, на которых в \( \almostallones{M}{N} \) стоят нули.

Спектр матрицы \( \allones{N} \) легко считается:
\( \spec{\allones{N}} = \left\{0, N \right\} \),
а при \( M \ll N^2 \) можно ожидать ``близость'' спектров
\( \almostallones{M}{N} \) и \( \allones{N} \).

Методом подобных операторов \cite{baskakov-harmonic}
достигается основной результат:
\begin{thm}\label{thm:almostallones-spectra}
    Пусть
    \(M~<~\displaystyle{\frac{1}{16}N^2}. \)

Тогда матрица \( \mathcal{A}_{MN} \) имеет в отрезке
    радиуса \( 4\sqrt{M} \) с центром в \( N \)
    в точности одно собственное значение \( \lambda_1 \):
    \[ \lvert \lambda_1 - N \rvert < 4\sqrt{M}, \]
    а остальные собственные числа
    \( \lambda\in\sigma\left({\mathcal{A}_{MN}}\right)\setminus\{\lambda_1\} \)
    удовлетворяют неравенству
    \[ \lvert \lambda \rvert < 4\sqrt{M}. \]
\end{thm}

Полные (почти-полные) графы --- довольно редкое явление,
но могут описывать, например, топологию полносвязной (близкую к ней) сети,
встречающуюся в задачах, требующих повышенной робастности.

В приложениях вместо спектра матрицы смежности чаще рассматривается
спектрм матрицы-лапласиана \( L = D - A \),
где \( D \) --- (диагональная) матрица степеней вершин,
а \( A \) --- матрица смежности.

Пример приложения спектра матрицы смежности можно увидеть
в следующей модели распространения компьютерного вируса \cite{epidemic-eigenvalues}:
пусть \( G = (V, E) \) --- связная сеть (связный орграф),
\( V = \{1, \ldots, N\} \) --- множество узлов (вершин),
\( E \) --- множество соединений (рёбер).
Время дискретное, в каждый интервал времени каждый заражённый узел
пытается заразить смежные ему узлы с одинаковой для всех узлов вероятностью успеха \( \beta \).
В то же время с вероятностью \( \delta \) каждая заражённая вершина может "исцелиться".
% Эволюция заражённой популяции \( \eta \) описывается уравнением
% \( \frac{\mathrm{d}\eta}{\mathrm{d}t}(t) = \beta (\mathbb{E} k) \eta_t (1-\eta_t) - \delta \eta_t \),
% где \( \mathbb{E} k \) --- средняя степень вершины графа.
Известно, существует пороговое значение \( \tau \) отношения \( {^\beta}/_{\delta} \),
выше которого вспышка вируса не гаснет и может превратиться в эпидемию
(формально: эпидемия гаснет \( \implies ^{\beta}/_{\delta} < \tau \)).
В \cite{epidemic-eigenvalues} показано, что \( \tau = 1/\spr{A} \),
где \( \spr{A} \) -- спектральный радиус мартрицы \( A \) смежностей графа \( G \).

\section{Доказательство}
\subsection*{Предварительные преобразования}
Доказательство состоит в~построении уравнения для матрицы, подобной \( A \),
 но устроеной ``проще''. Решение возникающего нелинейного уравнения
 в~банаховой алгебре \( \mathbb{C}^{N{\times}N} \)
 матриц размера \( N{\times}N \)
 доставляется методом простых итераций (см., например,~\cite{baskakov-harmonic}).

Подобие матриц \( \mathcal{A}_1, \mathcal{A}_2 \)
 понимается в~смысле существования обратимой матрицы \( \mathcal{U} \),
 такой что \( \mathcal{A}_1 \mathcal{U} = \mathcal{U} \mathcal{A}_2 \).
Подобные матрицы изоспектральны (их спектры совпадают).

Провед\"ем предварительные преобразования.

\begin{lem}
    Матрица единиц 
    \( \mathcal{J}_N =
    \begin{pmatrix}
        1 & \cdots & 1 \\
        \vdots & \ddots & \vdots \\ 
    1 & \cdots & 1 \end{pmatrix} \),
    подобна матрице
    \[
        \mathcal{A} = \begin{pmatrix}
            N & 0 & \cdots & 0 \\
            0 & 0 & \cdots & 0 \\
            \vdots & \vdots & \ddots & \vdots \\
            0 & 0 & \cdots & 0 \end{pmatrix}. \]
    Точнее, существует ортогональная матрица \( \mathcal{U} \),
    такая что
    \( \mathcal{J}_N = \mathcal{U}\mathcal{A} \mathcal{U}^{-1} \).
\end{lem}
\begin{proof}
    Собственному значению \( 0 \) соответствует \( N-1 \) независимый собственный вектор
        \( f_1 = {\left(1,-1,0,\ldots,0\right)}, \ldots,
           f_{N-1} = {\left(0,\ldots,0,1,-1\right)} \),
    а~собственному значению \( N \) матрицы \( \mathcal{J}_N \) 
    соответствует собственный вектор \( f_N = {\left(1,\ldots,1\right)} \).
    Применив ортогонализацию Грамма-Шмидта, получим ортонормальную систему \( h_1, \ldots, h_N \):
    \[
        h_k = \frac{1}{\sqrt{k(k+1)}}
        \left(\smash{\underbrace{1,~\ldots,~1,}_{k \text{ раз}}}~-k,~0,~\ldots,~0\right)
        \in \mathbb{R}^N, \quad k={1, \ldots, N-1} \]
    \[
        h_N = {\left(1,~\ldots,~1\right)} \in \mathbb{R}^N, \]
    В~качестве матрицы \( \mathcal{U} \) выберем матрицу,
    имеющую столбцами векторы \( h_N, h_1, \ldots, h_{N-1} \):
    \[ \mathcal{U} =
    \begin{pmatrix}
        \frac{1}{\sqrt N} &  \frac{1}{\sqrt2} &  \frac{1}{\sqrt{6}} & \cdots & \frac{1}{\sqrt{(N-2)(N-1)}} & \frac{1}{\sqrt{N(N-1)}} \\
        \frac{1}{\sqrt N} & -\frac{1}{\sqrt2} &  \frac{1}{\sqrt{6}} & \cdots & \frac{1}{\sqrt{(N-2)(N-1)}} & \frac{1}{\sqrt{N(N-1)}} \\
        \frac{1}{\sqrt N} & 0                 & -\frac{2}{\sqrt{6}} & \cdots & \frac{1}{\sqrt{(N-2)(N-1)}} & \frac{1}{\sqrt{N(N-1)}} \\
        \frac{1}{\sqrt N} & 0                 &  0                  & \cdots & \frac{1}{\sqrt{(N-2)(N-1)}} & \frac{1}{\sqrt{N(N-1)}} \\
        \vdots            & \vdots            &  \vdots             & \ddots & \vdots                      & \vdots   \\
        \frac{1}{\sqrt N} & 0                 &  0                  & \cdots & \frac{1}{\sqrt{(N-2)(N-1)}} & \frac{1}{\sqrt{N(N-1)}} \\
        \frac{1}{\sqrt N} & 0                 &  0                  & \cdots & \frac{2-N}{\sqrt{(N-2)(N-1)}} & \frac{1}{\sqrt{N(N-1)}} \\
        \frac{1}{\sqrt N} & 0                 &  0                  & \cdots & 0                  & \frac{1-N}{\sqrt{N(N-1)}}
    \end{pmatrix}.\]
\end{proof}

Таким образом, исходная матрица смежностей \( A \) подобна матрице
\( \mathcal{A}{-}\mathcal{B} \), где \( \mathcal{B}{=}\mathcal{U}^{-1} B \mathcal{U} \).
Далее ортогональность матрицы \( U \) будет играть важную роль.

% \clearpage
\subsection*{Расщепление матрицы и результат}
Следуя общей схеме \cite{baskakov-harmonic}, мы будем искать матрицу, подобную \( A - B \)
в виде
\( A - J X, \)
с матрицей подобия вида
\( E + \Gamma X ,\)
где
\( J, \Gamma \in \mathrm{End}\left(\mathfrak U\right) \)
--- непрерывные линейные операторы (трансформаторы),
действующие на некотором подпространстве \( \mathfrak U \)
алгебры \( \mathrm{Matr}_{N} \),
называемом пространством допустимых возмущений,
и подбираемые в ходе решения так, чтобы матрица \( A - JX \)
имела как можно более простую структуру.

Иначе говоря, будем искать решение \( X \in\mathrm{Matr}_{N} \)
уравнения подобия
\begin{equation}\label{eq:similarity-orig}
    (A - B)(E+\Gamma X) = (E+\Gamma X) (A - JX).
\end{equation}

В подпространстве \( \mathfrak U \subset \mathrm{Matr}_{N} \)
мы будем запускать итерационный процесс,
и, как будет видно ниже, необходимо, кроме прочего,
чтобы \( \mathfrak U \) содержало \( B \).
В виду способа построения матрицы \( B \), сказать что-либо о её структуре затруднительно,
поэтому мы положим \( \mathfrak U = \mathrm{Matr}_{N}\).

Пространство \( \mathscr{X} \) представим в виде
декартова произведения
\( \mathscr{X}_1~\times~\mathscr{X}_2 \)
одномерного пространства \( \mathscr{X}_1=\mathbb{R} \)
и \((n-1)\)-мерного пространства \( \mathscr{X}_2 \),
элементы \( \mathscr{X} \) в виде
\( x=\begin{pmatrix}x_1\\x_2\end{pmatrix}, \)
\( x_i\in\mathscr{X}_i \).
Матрицы из \( \mathrm{Matr}_{N} \) будем записывать
в блочном виде:
\(
    X = \begin{pmatrix}
    X_{11} & X_{12} \\
    X_{21} & X_{22}
    \end{pmatrix},
    \)
где \( { X_{11}\in\mathbb{R} } \) -- число,
    \( X_{12} \) -- строка длины \( N-1 \),
    \( X_{21} \) -- столбец,
    \( X_{22} \) -- матрица порядка \( N-1 \).
При этом произведению матриц соответствует произведение блочных матриц.

Вернёмся к задаче.
Оператор \( \Gamma \) определим как решение уравнения\footnotemark
\[
    A\Gamma X - (\Gamma X) A = X - JX,\quad X\in\mathrm{Matr}_{N}.
    \]%
\footnotetext{в левой части стоит \( \mathrm{ad}_{A}\left({\Gamma X}\right) \)
--- оператор коммутирования с \(A\), применённый к \(\Gamma X\): \( \mathrm{ad}_{Y}\left({Z}\right) = YZ - ZY \)}%
Пусть \( \Gamma \) действует по формуле
\( \Gamma X =
\begin{pmatrix}
    \Gamma_{11}(X) & \Gamma_{12}(X) \\
    \Gamma_{21}(X) & \Gamma_{22}(X)
\end{pmatrix}, \)
тогда
\[
    A \Gamma X - (\Gamma X) A =
    N
    \begin{pmatrix}
        0               & \Gamma_{12}(X) \\
        -\Gamma_{21}(X) & 0
    \end{pmatrix},
\]
и уравнение приводится к виду
\[
    X - JX =
    N
    \begin{pmatrix}
        0               & \Gamma_{12}(X) \\
        -\Gamma_{21}(X) & 0
    \end{pmatrix}.
\]
Можем положить
\[
    JX = \begin{pmatrix}
        X_{11} & 0 \\
        0      & X_{22}
    \end{pmatrix},\quad
    \Gamma X =
    \frac1N
    \begin{pmatrix}
        0       & X_{12} \\
        -X_{21} & 0
    \end{pmatrix}.
    \]
При этом матрица \( A - JX \) оказывается блочно-диагональной
и её спектр есть объединение спектров её диагональных блоков.
Заметим, естественнее было бы начать с выбора оператора \( J \) доставляющего структуру,
как можно более близкую к \( A \), а именно
оператор \( X\mapsto X_{11} \).
К сожалению, такой выбор \( J \) не позволяет решить
уже уравнение для \(\Gamma X\) во всём \(\mathfrak U = \mathrm{Matr}_{N}\),
и мы приходим к блочно-диагональному виду.

Уравнение \eqref{eq:similarity-orig} перепишем в виде
\[
    A - B + A\Gamma X - B\Gamma X = A - JX + (\Gamma X) A - (\Gamma X) JX,
    \]
\[
    A\Gamma X - (\Gamma X) A + JX = B \Gamma X + B - (\Gamma X) JX,
    \]
\[
    X = B \Gamma X + B - (\Gamma X) JX.
    \]
Применяя слева оператор \( J \) и подставляя результат обратно, получим уравнение
\begin{equation}\label{eq:similarity2}
    X = B \Gamma X + B - (\Gamma X) (J(B\Gamma X + B))
\end{equation}

Теперь мы докажем существование единственного решения уравнения \eqref{eq:similarity2}
и выведем оценки спектра,
показав, что в правой части этого уравнения стоит образ \( X \)
при сжимающем (в некотором шаре с центром в нуле) отображении \( \Phi \):
\[ \Phi(X) = B \Gamma X + B - (\Gamma X) (J(B\Gamma X + B)) \]

Покажем сначала существование такого шара \( \{ X\in\mathrm{Matr}_{N}; {\left\|X\right\|} \leq \varepsilon \} \),
который содержит свой образ при отображении \( \Phi \)
(другими словами, матрицы из этого шара
 не покидают его под действием \( \Phi \))

Будем считать, что в алгебре матриц \( \mathrm{Matr}_{N} \)
введена некоторая субмультипликативная норма
(i.e. норма, удовлетворяющая неравенству
 \( {\left\|XY\right\|}\leq {\left\|X\right\|}{\left\|Y\right\|} \) для всех \( X, Y \in\mathrm{Matr}_N\)),
а в пространстве \( \mathrm{End}\left({\mathfrak U}\right) \) трансформаторов
используется обычная операторная норма
(i.e. \( \left\|X\right\|_\mathrm{op} =
          \inf
          \left\{ C>0;
          {\left\|X x\right\|} \leq C {\left\|x\right\|} \text{ для всех } x\in\mathscr{X}
          \right\}
          \)).
Тогда имеют место оценки
\begin{align*}
    {\left\|\Phi(X)\right\|} &= {\left\|B \Gamma X + B - (\Gamma X) (J(B\Gamma X + B))\right\|} \leq \\
%    &\leq {\left\|B\right\|}{\left\|\Gamma\right\|}{\left\|X\right\|} + {\left\|B\right\|} + {\left\|\Gamma\right\|}{\left\|X\right\|} ({\left\|B\right\|}{\left\|\Gamma\right\|}{\left\|X\right\|}+{\left\|B\right\|}) \leq \\
    &= \gamma^2\beta{\left\|X\right\|}^2 + 2\gamma\beta{\left\|X\right\|} + \beta, \\
    \beta&={\left\|B\right\|}, \gamma={\left\|\Gamma\right\|}.
\end{align*}
\begin{lemma}
    Если
    \( \gamma\beta \leq \frac14, \)
    то \( \Phi \) отображает в себя шар
    \begin{equation}\label{def:omega}
        \Omega = \{ X\in\mathrm{Matr}_{N} ; {\left\|X\right\|}\leq r_0 \},
    \end{equation}
    \[0 \leq r_0 = \frac{1 - 2\gamma\beta - \sqrt{1-4\gamma\beta}}{2\gamma^2\beta} < 4\beta, \]
\end{lemma}
\begin{proof}
    Обозначим \( r={\left\|X\right\|} \). Тогда
    \( {\left\|\Phi(X)\right\|} \leq {\left\|X\right\|} \), если
    \begin{equation}\label{eq:omega-invariance}
        \gamma^2\beta r^2 + (2\gamma\beta - 1) r + \beta \leq 0.
    \end{equation}
    Детерминант возникшего многочлена:
    \( \Delta = 1 - 4\gamma\beta\).\-
    Стало быть, если \( {\gamma\beta \leq \frac14} \),
    то определённое выше \( r_0 \) есть наименьшее \( r \),
    удовлетворяющее неравенству \eqref{eq:omega-invariance}.
    Непосредственно проверяется, что \( 0 \leq  r_0 \leq 4\beta \).
\end{proof}

Определим теперь, при каких условиях \( \Phi \) будет сжимающим в найдённом шаре.
\begin{lemma}
    Пусть \( \gamma\beta \leq \frac14, \)
    тогда \( \Phi \) есть сжимающее отображение \(\Omega\to\Omega\) шара \(\Omega\) в себя:
    \[ {\left\|\Phi(X)-\Phi(Y)\right\|} \leq q {\left\|X-Y\right\|}, \quad q<1.\]
\end{lemma}
\begin{proof}
    Непосредтвенно проверяется, что
    \begin{align*}
        \Phi(X) - \Phi(Y) &= B\Gamma (X-Y) + \\
        &- \frac12 \Gamma (X + Y)J(B\Gamma(X - Y)) + \\
        &- \frac12 \Gamma (X - Y)J(B\Gamma(X + Y)),
        \marginnote{\footnotemark}
    \end{align*}
    \[
        {\left\|\Phi(X) - \Phi(Y)\right\|} \leq (1 + \gamma{\left\|X+Y\right\|})\beta\gamma {\left\|X-Y\right\|} \leq q{\left\|X-Y\right\|}.
        \]
    Здесь
    \(
        q = (1+2\gamma r_0)\gamma\beta
        \leq (1+8\gamma\beta)\gamma\beta \leq \frac{3}{4}.
    \)
    Это и значит, что \( \Phi \) --- сжимающее отображение в найдённом шаре.
\end{proof}

\begin{lemma}
    Пусть \( {\left\|\cdot\right\|} \) --- какая-нибудь субмультипликативная норма в \( \mathrm{Matr}_{N} \),
    \( \beta={\left\|B\right\|}, \gamma=\sup_{{\left\|X\right\|}=1}{\left\|\Gamma X\right\|} \)
    и пусть
    \( \gamma\beta\leq\frac14. \)

    Тогда в \( \Omega \) существует единственное решение
    \( X^o = \begin{pmatrix}
        x_{11}^o & X_{12}^o \\
        X_{21}^o & X_{22}^o
    \end{pmatrix}\) уравнения подобия
    \eqref{eq:similarity2}.
    Спектр матрицы \( A - B \) представим в виде
    \( \left\{ N - x_{11}^o \right\} \cup \mathrm{spec}\left({-X_{22}^o}\right) \).
    При этом имеют место оценки:
    \( \lvert x_{11} \rvert, \mathrm{spr}\left({-X_{22}^o}\right) \leq \mathrm{spr}\left({X^o}\right)
    \leq {\left\|X^o\right\|} \leq r_0 \leq 4\beta,
        \)
    где \( \mathrm{spr}\left({X}\right)
            = \max_{\lambda\in\mathrm{spec}\left(X\right)}
            \lvert \lambda \rvert \)
    означает спектральный радиус матрицы \( X \in\mathrm{Matr}_{N}. \)
\end{lemma}
\begin{proof}
    Сужение \( \Phi \) на шар \( \Omega \) есть сжимающее отображение \( \Omega\to\Omega \).
    Замкнутое подмножество \( \Omega \)
    банахова пространства \( \mathrm{Matr}_{N} \)
    образует полное метрическое пространство.
    По теореме Банаха о неподвижной точке, в этом шаре существует единственное решение \( X^o \)
    уравнения \( X = \Phi (X) \),
    доставляемое методом простых итераций (c нулём в качестве начального приближения),
    как предел сходящейся последовательности
    \[
        \{X_k = \Phi^k(0); k\in\mathbb{N}\},
        \]
    где \( \Phi^k = \Phi \circ \Phi^{k-1}, \)
    \( \Phi^0 = I \) (\( I \) --- тождественный оператор).
    Норма этого решения не превосходит радиус \( r_0 \) шара \( \Omega \),
    а собственные значения по модулю не превосходят нормы (см. \cite{baskakov-harmonic}).
    Кроме того, имеет место оценка скорости сходимости
    \[
        {\left\|X_k - X^o\right\|} \leq \frac{q^k}{1-q} {\left\|\Phi(0) - 0\right\|} = \frac{q^k}{1-q}{\left\|B\right\|},
        \]
    \[
        {\left\|X_k - X^o\right\|} \leq \frac{3^k}{4^{k-1}}\beta.\marginnote{\footnotemark}
        \]

    Значит, матрица \( A-B \) подобна блочно-диагональной матрице \( A - JX^o \):
    \[
        A-B \sim
    \begin{pmatrix}
        N - x_{11}^o & 0 \\
        0            & -X_{22}^o
    \end{pmatrix},
    \]
    поэтому спектр матрицы \( A - B \) есть объединение спектров
    диагональных блоков матрицы \( A - JX^o \):
    \[ \mathrm{spec}\left({A-B}\right) = \left\{N - x_{11}^o\right\} \cup \mathrm{spec}\left({-X_{22}^o}\right). \]
\end{proof}

Доказательство основной теоремы (стр. \pageref{thm:almostallones-spectra})
состоит в выборе подходящей субмультипликативной нормы.
Матрица \( B = U^{-1} \mathscr{B}_{M,N} U \)
получена из матрицы возмущения с \( M \) единицами
ортогональным преобразованием подобия \eqref{eq:diagtransform},
диагонализирующим матрицу единиц.
\( \Gamma \) --- оператор, действующий в \( \mathrm{Matr}_{N} \)
по формуле
\( { \Gamma X = \frac1N \begin{pmatrix}0 & X_{12} \\ -X_{21} & 0\end{pmatrix} } \).

Рассмотрим в пространстве \( \mathrm{Matr}_{N} \)
норму Фробениуса \( {\left\|\cdot\right\|}_{F} \),
определённую формулой
\( {\left\|X\right\|}_{F} = \sqrt{\sum_{ij} \lvert x_{ij}\rvert^2}. \)
Эта норма субмультипликативна.

Тогда:
\[ \gamma = \frac1N
            \sup_{{\left\|X\right\|}_{F}=1}{\left\|\begin{pmatrix}0 & X_{12} \\ -X_{21} & 0\end{pmatrix}\right\|}_{F}
          = \frac1N,
    \]
и, так как умножение на унитарную матрицу \( U \)
    (и \( U^{-1} \)) есть изометрия в \( \mathrm{Matr}_{N} \),
    а \( \mathscr{B}_{M,N} \) состоит из \( M \) единиц:
\[
    \beta = {\left\|B\right\|}_{F} =
    {\left\|\mathscr{B}_{M,N}\right\|}_{F} = \sqrt{M},
    \]

Значит, если
\( \sqrt{M} < \frac{N}{4} \), т.е.
\( M < \frac{N^2}{16} \),
то выполняются условия леммы,
причём \( r_0 < 4\sqrt{M} \).

\begin{thebibliography}{9}
\bibitem{baskakov-harmonic}
    \baut{Баскаков}{А.~Г.}
    \btit{Гармонический анализ линейных операторов}[Harmonic analysis of linear operators]
    \bpub{Издательство
        Воронежского Государственного Университета}[
            Voronezh State University Publishing House]
    \bcity{Воронеж}
    \byr{1987.}
    \bpp{93--121}
    \mkbookr
\bibitem{baskakov-split} Баскаков~А.~Г.
    \baut{Баскаков}{А.~Г.}
        \btit{Расщепление возмущённого дифференциального оператора
              с неограниченными операторными коэффициентами}[Analysis
              of linear differential equations by methods
              of the spectral theory of difference operators and linear relations]
        \bj{Фундаментальная и прикладная математика}[Russian Mathematical Surveys]
        \byr{2002}
        \bvol{8}
        \bnum{1}
        \bpp{1--16}
        \mkpaperr
\bibitem{epidemic}
    \baut{Wang}{Yang}
        \baut{Chakrabarti}{D.}
        \baut{Wang}{Chenxi}
        \baut{Faloutsos}{C.}
        \btit{Epidemic spreading in real networks: an eigenvalue viewpoint}
        \bj{22nd International Symposium on Reliable Distributed Systems, Oct 2003. Proceedings}
        \byr{2003}
        \bpp{25--34}
        \mkpapere
\bibitem{cvet}
    \baut{Cvetkovic}{D.~M.}
    \baut{Doob}{M.}
    \baut{Sachs}{H.}
    \btit{Spectra of Graphs: Theory and Applications (3rd revision)}
    \bpub{Wiley}
    \bcity{New York}
    \byr{1998}
    \mkbooke

\end{thebibliography}



\begin{summary}
    Suppose
\begin{equation}\label{eq:kozlukovsv:amn}
    \mathscr{A}_{MN} =
    \begin{pmatrix}
        1 & \cdots & 1 \\
        \vdots & \ddots & \vdots \\
        1 & \cdots & 1
    \end{pmatrix} - \mathscr{B}_{MN}
\end{equation}
    is a~\( N\times N \) matrix composed of
    \( N^2 - M \) unities and \( M \) zeroes.
If considered as an adjacency matrix \( \mathscr{A}_{MN} \)
    defines a~complete digraph (with loops) on \( N \) vertices
    with some \( M \) out of \( N^2 \) total edges removed.
Some important properties of a~graph are determined by its spectrum.
For example Wang et al.\ \cite{epidemic} proposed a~discrete-time model
    of viral propagation in a~network.
In this work it is shown that the ratio
    of the cure and the infection probabilities
    being belove or above a certain treshold value
    determines whether the contamination will take an epidemic form
    or fade out.
Wang et al.\ have shown that treshold value
    to be the spectral radius of the adjacency matrix of the network,
    i.e.\ the largest absolute value of its eigenvalues.
A more comprehensive description of spectral graph theory
    and its application is given by Cv\`etkovic et al.\ \cite{cvet}.

This article analyzes spectral properties of matrices such as \eqref{eq:kozlukovsv:amn}.
First of all the matrix \( \mathscr{A}_{MN} \) can be represented in the form
    \( \mathscr{A}_{MN} = \mathcal{J}_N - \mathscr{B}_{MN} \),
    where \( \mathcal{J}_N \) is a~\( N\times N \) matrix
    whose all entries are ones
    and \( \mathcal{B}_{MN} \) has ones exactly at these \( M \)
    places where \( \mathscr{A}_{MN} \) has zeros.
The spectrum of \( \mathcal{J}_N \) can be easily computed:
    \( \mathcal{J}_N^2 = N \mathcal{J} \),
    so \( \lambda(\lambda - N) \) is the minimal
    annihilating polynomial of \( \mathcal{J}_N \)
    and thus the spectrum of \( \mathcal{J}_{N} \) is
    \( \sigma(\mathcal{J}_N) = \{ 0,N \} \).

Be \( M \) small enough
    the eigenvalues of \( \mathscr{A}_{MN} \) would be close to those of \( \mathcal{J}_N \).
Applying the method of similar operators (see \cite{baskakov-harmonic,baskakov-split})
    we prove the following theorem:
\begin{thm}\label{kozlukovsv:thm:almost-all-ones}
    Let \( M < \frac{N^2}{16} \),
    then the spectrum of \( \mathscr{A}_{MN} \)
    can be represented as a~disjoint union
    \( \sigma\left(\mathscr{A}_{MN}\right) = \sigma_1 \cup \sigma_2 \)
    of a~singletone \( \sigma_1=\{\lambda_1\} \)
    and the set \( \sigma_2 \) satisfying the following restrictions:
    \[
        \sigma_1 \subset \left\{
            \mu\in\mathbb{R};\ \lvert N - \mu \rvert < 4\sqrt{M}
            \right\},
        \]
    \[ \sigma_2 \subset \left\{ \mu\in\mathbb{C};\ \lvert \mu \rvert < 4\sqrt{M} \right\}. \]
\end{thm}


\end{summary}
\end{document}
