\documentclass[a4paper]{article}

% \usepackage{libertine}
% \usepackage{polyglossia}
% \setdefaultlanguage{russian}
% \setotherlanguage{english}

\usepackage[utf8]{inputenc}
\usepackage[russian]{babel}
\usepackage{graphicx}
\usepackage{amsmath}
\usepackage{amsfonts}
\usepackage{amssymb}
\pagestyle{empty} \textwidth=108mm \textheight=165mm
\usepackage{calc,ifthen}


\newcommand{\tezislarge}{\normalsize}
\providecommand{\No}{\textnumero}
\newcommand{\me}[1]{#1\index{#1}}

\newcounter{first}
\newcommand{\tezis}[7][1]{\protect\setcounter{first}{\thepage}%
\begin{center}\tezislarge\textbf{#2}%
\ifthenelse{\not\equal{#3}{}}{{}}{}%
%\setcounter{theorem}{0}\setcounter{teo}{0}\setcounter{lemma}{0}\setcounter{figure}{0}%
\ifthenelse{\not\equal{#3}{}}{\footnote{#3}}{}%
\\*\textbf{#4}%
%\index{#5}%
\footnotemark\ #6\\*%\addcontentsline{loa}{author}{#3}\\*%
\footnotetext{\copyright\ #5, 2017}
\ifthenelse{\not\equal{#7}{}}{\textit{#7}}{}%
\end{center}\setcounter{equation}{0}%\addcontentsline{toc}{section}{#3, #1}%
}

\newcommand{\liter}{\smallskip\centerline{\textbf{Литература}}\nopagebreak}
 % not a package? why?

\begin{document}
\tezis{СПЕКТР МАТРИЦЫ СМЕЖНОСТЕЙ ПОЧТИ ПОЛНОГО ОРГРАФА}{}
{С. В. Козлуков}
{\me{Козлуков С.В.}}{ (Воронеж)}{rerumnovarum@openmailbox.org}

Пусть \( \mathcal{A}_{MN} \) --- матрица порядка \( N \),
состоящая из \( N^2 - M \) единиц и \( M \) нулей
и пусть число нулей невелико.
Матрица такого вида (рассматриваемая, как матрица смежностей) соответствует
полному орграфу, с небольшим числом ``удалённых'' рёбер.
В статье получена оценка спектра
матриц такого вида.

Матрица \( \mathcal{A}_{MN} \) представима в виде
\[
    \mathcal{A}_{MN} = \mathcal{J}_{N} - \mathcal{B}_{MN},
    \]
где \(
\mathcal{J}_{N} =
\begin{pmatrix}1 & \cdots & 1 \\
\vdots & \ddots & \vdots \\
1 & \cdots & 1
\end{pmatrix} \) --- матрица единиц,
а \( \mathcal{B}_{MN} \) имеет единицы в точности на тех \( M \)
местах, на которых в \( \mathcal{A}_{MN} \) стоят нули.

Спектр \( \sigma(\mathcal{J}_{N}) \) матрицы \( \mathcal{J}_{N} \) легко считается:
\( \sigma\left({\mathcal{J}_{N}}\right) = \left\{0, N \right\} \),
а при \( M~\ll~N^2 \) можно ожидать ``близость'' спектров
\( \mathcal{A}_{MN} \) и \( \mathcal{J}_{N} \).

Методом подобных операторов [1]
достигается основной результат:

% you better just provide an environment for theorems...
\textbf{Теорема.}
\begin{center}\relax
    Пусть
    \(M~<~\displaystyle{\frac{N^2}{16}}. \)

Тогда спектр матрицы \( \mathcal{A}_{MN} \) есть объединение
    одноточечного вещественного множества \( \sigma_1 \)
    и множества \( \sigma_2 \),
    лежащих в шарах
    радиуса \( 4\sqrt{M} \)
    с центрами в \( N \) и \( 0 \), соответственно:
    \[ \sigma\left(\mathcal{A}_{MN}\right) = \sigma_1 \cup \sigma_2 \]
    \[ \sigma_1 = \left\{\lambda_1 \right\}
        \subset \left\{\lambda\in\mathbb{R}; \lvert \lambda - N \rvert < 4\sqrt{M} \right\}, \]
    \[ \sigma_2
        \subset \left\{ \lambda\in\mathbb{C}; \lvert \lambda \rvert < 4\sqrt{M} \right\}. \]
\end{center}

\liter

1. Баскаков~А.~Г. Гармонический анализ линейных операторов. 1987
\end{document}
