\documentclass[a4paper]{article}

% \usepackage{libertine}
% \usepackage{polyglossia}
% \setdefaultlanguage{russian}
% \setotherlanguage{english}

\usepackage[utf8]{inputenc}
\usepackage[russian]{babel}
\usepackage{graphicx}
\usepackage{amsmath}
\usepackage{amsfonts}
\usepackage{amssymb}
\pagestyle{empty} \textwidth=108mm \textheight=165mm
\usepackage{calc,ifthen}

\input{style.tex} % not a package? why?

\begin{document}
\tezis{СПЕКТР МАТРИЦЫ СМЕЖНОСТЕЙ ПОЧТИ ПОЛНОГО ОРГРАФА}{}
{С. В. Козлуков}
{\me{Козлуков С.В.}}{ (Воронеж)}{rerumnovarum@openmailbox.org}

Пусть \( \mathcal{A}_{MN} \) --- матрица порядка \( N \),
состоящая из \( N^2 - M \) единиц и \( M \) нулей
и пусть число нулей невелико.
Матрица такого вида (рассматриваемая, как матрица смежностей) соответствует
полному орграфу, с небольшим числом ``удалённых'' рёбер.
В статье получена оценка спектра
матриц такого вида.
Здесь под спектром \( \sigma\left({A}\right) \) матрицы \( A \)
понимается множество таких \( \lambda\in\mathbb{C} \), для которых
матрица \( A - \lambda E \), необратима (\( E \) --- единичная матрица).

Матрица \( \mathcal{A}_{MN} \) представима в виде
\[
    \mathcal{A}_{MN} = \mathcal{J}_{N} - \mathcal{B}_{MN},
    \]
где \(
\mathcal{J}_{N} =
\begin{pmatrix}1 & \cdots & 1 \\
\vdots & \ddots & \vdots \\
1 & \cdots & 1
\end{pmatrix} \) --- матрица единиц,
а \( \mathcal{B}_{MN} \) имеет единицы в точности на тех \( M \)
местах, на которых в \( \mathcal{A}_{MN} \) стоят нули.

Спектр матрицы \( \mathcal{J}_{N} \) легко считается:
\( \sigma\left({\mathcal{J}_{N}}\right) = \left\{0, N \right\} \),
а при \( M~\ll~N^2 \) можно ожидать ``близость'' спектров
\( \mathcal{A}_{MN} \) и \( \mathcal{J}_{N} \).

Методом подобных операторов [1]
достигается основной результат:

% you better just provide an environment for theorems...
\textbf{Теорема.}
\begin{center}\relax
    Пусть
    \(M~<~\displaystyle{\left(\frac{N}{4}\right)^p}, \)
    для некоторого \( p\in\mathbb{N} \).

Тогда матрица \( \mathcal{A}_{MN} \) имеет в интервале
    радиуса \( 4\sqrt[p]{M} \) с центром в \( N \)
    в точности одно собственное значение \( \lambda_1 \):
    \[ \lvert \lambda_1 - N \rvert < 4\sqrt[p]{M}, \]
    а остальные собственные числа
    \( \lambda\in\sigma\left({\mathcal{A}_{MN}}\right)\setminus\{\lambda_1\} \)
    удовлетворяют неравенству
    \[ \lvert \lambda \rvert < 4\sqrt[p]{M}. \]
\end{center}

\liter

1. Баскаков~А.~Г. Гармонический анализ линейных операторов. 1987
\end{document}
