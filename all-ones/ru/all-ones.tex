\section{Матрица единиц}

Матрицей единиц, размерности \( N \) будем называть матрицу
\( \allones{N} \in \matr{N}{N} \):
\[
    \allones{N} =
    \begin{pmatrix}
    1 & 1  & \cdots & 1 \\
    1 & 1  & \cdots & 1 \\
    \vdots & \vdots & \ddots & \vdots \\
    1 & 1  & \cdots & 1
    \end{pmatrix}
\]

Спектром матрицы \( A \) называется множество таких \( \lambda\in\CC \),
что матрица \( E + \lambda A \) вырождена

\begin{propose}
\[
    \allones{N}^2 = N \allones{N}
    \]
\[
    \spec{\allones{N}} = \left\{ 0, N \right\}
    \]
\end{propose}

\begin{dfn}
    Две матрицы \( A_1, A_2 \)
    называются подобными, если существует обратимая матрица \( U \),
    такая что
    \[
        A_1 U = U A_2
        \]
    Или, эквивалентно:
    \[
        U^{-1} A_1 U = A_2
        \]
\end{dfn}

\begin{propose}
    Матрица \( \allones{N} \) подобна матрице
    \[\begin{pmatrix}
        N &   &        & \\
        & 0 &        & \\
        &   & \ddots & \\
        &   &        & 0
    \end{pmatrix}\]

    Диагонализирующая матрица преобразования имеет вид:
    \begin{equation}\label{eq:diagtransform}
    T =
    \begin{pmatrix}
        1/\sqrt N &  1/\sqrt2 & 1/\sqrt6  & 1/2\sqrt3  & \cdots & a_{n-1} \\
        \vdots    & -1/\sqrt2 & 1/\sqrt6  & 1/2\sqrt3  & \cdots & a_{n-1} \\
        \vdots    & 0         & -2/\sqrt6 & 1/2\sqrt3  & \cdots & a_{n-1} \\
        \vdots    & \vdots    & 0         & -3/2\sqrt3 & \cdots & a_{n-1} \\
        \vdots    & \vdots    & \vdots    & 0          & \cdots & a_{n-1} \\
        \vdots    & \vdots    & \vdots    & \vdots     & \ddots & \vdots  \\
        \vdots    & \vdots    & \vdots    & \vdots     & \cdots & a_{n-1} \\
        1/\sqrt N & 0         & 0         & 0          & \cdots & -(n-1)a_{n-1}
        \end{pmatrix}
        \end{equation}

    Её столбцы \( f_0, \ldots, f_n \) -- ортонормированный собственный базис матрицы \( \allones{N} \):

    \[f_0 = \begin{pmatrix} \frac{1}{\sqrt N} & \cdots & \frac{1}{\sqrt N} \end{pmatrix}^\transposed\]
    \providecommand{\fknorm}{\sqrt{k a_{k-1}^2 b_{k-1}^2 + 1}}
    \[
        f_k =
        \begin{pmatrix}a_k \\ \vdots \\ a_k \\ -ka_k \\ 0 \\ \vdots \\ 0 \end{pmatrix} =
        \frac{1}{k\fknorm}
        \begin{pmatrix}1 \\ \vdots \\ 1 \\ -k \\ 0 \\ \vdots \\ 0 \end{pmatrix}
        \]
    \[
         a_k = \frac{1}{k\fknorm}
         \]
         Где \( k=\overline{2,N-1} \)

    \[
             f_1
             = \begin{pmatrix}a_1 \\ -a_1 \\ 0 \\ \vdots \\ 0 \end{pmatrix}
             = \begin{pmatrix}\frac1{\sqrt2} \\ - \frac{1}{\sqrt2} \\ 0 \\ \vdots \\ 0 \end{pmatrix}
             \]

    Эта матрица ортогональна и поэтому
    \( T^{-1} = T^\conjtransposed = T^\transposed \),
    \( \det T = 1 \).
\end{propose}
