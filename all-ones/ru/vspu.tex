\documentclass[11pt]{article}

\usepackage{fontspec}
\setmainfont{CMU Serif}
\usepackage{polyglossia}
\setdefaultlanguage{russian}
\setotherlanguage{english}


%\usepackage[a5paper]{geometry}
\usepackage[a5paper,left=20mm, right=20mm, top=20mm, bottom=20mm, includefoot]{geometry}
\usepackage[intlimits]{amsmath}
\interdisplaylinepenalty=2500
\usepackage{amsfonts}
\usepackage{amssymb}
\usepackage{latexsym}

\usepackage{mathrsfs}
\usepackage{mathtools}
\usepackage{amsthm}
\newtheorem{thm}{Теорема.}

\begin{document}
\addcontentsline{toc}{section}{{\it{Козлуков С.В.}} Spectra of an almost-all-ones matrix}
\setcounter{thm}{0}
\small УДК 517.984.3 : 519.177

\begin{center}
\textbf{СПЕКТР МАТРИЦЫ СМЕЖНОСТЕЙ ПОЧТИ ПОЛНОГО ОРГРАФА}\\
\small{Козлуков С.В.} \\
\small{Воронежский Государственный Университет}
\small{<rerumnovarum@openmailbox.org>} \\
\end{center}

\section{Введение}

Пусть \( \almostallones{M}{N} \) --- матрица порядка \( N \),
состоящая из \( N^2 - M \) единиц и \( M \) нулей
и пусть число нулей невелико.
Матрица такого вида (рассматриваемая, как матрица смежностей) соответствует
полному орграфу, с небольшим числом ``удалённых'' рёбер.
В статье получена оценка спектра
матриц такого вида.
Здесь под спектром \( \spec{A} \) матрицы \( A \)
понимается множество таких \( \lambda\in\CC \), для которых
матрица \( A - \lambda E \), необратима (\( E \) --- единичная матрица).

Матрица \( \almostallones{M}{N} \) представима в виде
\[
    \almostallones{M}{N} = \allones{N} - \perturbmatrix{M}{N},
    \]
где \(
\allones{N} =
\begin{pmatrix}1 & \cdots & 1 \\
\vdots & \ddots & \vdots \\
1 & \cdots & 1
\end{pmatrix} \) --- матрица единиц,
а \( \perturbmatrix{M}{N} \) имеет единицы в точности на тех \( M \)
местах, на которых в \( \almostallones{M}{N} \) стоят нули.

Спектр матрицы \( \allones{N} \) легко считается:
\( \spec{\allones{N}} = \left\{0, N \right\} \),
а при \( M \ll N^2 \) можно ожидать ``близость'' спектров
\( \almostallones{M}{N} \) и \( \allones{N} \).

Методом подобных операторов \cite{baskakov-harmonic}
достигается основной результат:
\begin{thm}\label{thm:almostallones-spectra}
    Пусть
    \(M~<~\displaystyle{\frac{1}{16}N^2}. \)

Тогда матрица \( \mathcal{A}_{MN} \) имеет в отрезке
    радиуса \( 4\sqrt{M} \) с центром в \( N \)
    в точности одно собственное значение \( \lambda_1 \):
    \[ \lvert \lambda_1 - N \rvert < 4\sqrt{M}, \]
    а остальные собственные числа
    \( \lambda\in\sigma\left({\mathcal{A}_{MN}}\right)\setminus\{\lambda_1\} \)
    удовлетворяют неравенству
    \[ \lvert \lambda \rvert < 4\sqrt{M}. \]
\end{thm}

Полные (почти-полные) графы --- довольно редкое явление,
но могут описывать, например, топологию полносвязной (близкую к ней) сети,
встречающуюся в задачах, требующих повышенной робастности.

В приложениях вместо спектра матрицы смежности чаще рассматривается
спектрм матрицы-лапласиана \( L = D - A \),
где \( D \) --- (диагональная) матрица степеней вершин,
а \( A \) --- матрица смежности.

Пример приложения спектра матрицы смежности можно увидеть
в следующей модели распространения компьютерного вируса \cite{epidemic-eigenvalues}:
пусть \( G = (V, E) \) --- связная сеть (связный орграф),
\( V = \{1, \ldots, N\} \) --- множество узлов (вершин),
\( E \) --- множество соединений (рёбер).
Время дискретное, в каждый интервал времени каждый заражённый узел
пытается заразить смежные ему узлы с одинаковой для всех узлов вероятностью успеха \( \beta \).
В то же время с вероятностью \( \delta \) каждая заражённая вершина может "исцелиться".
% Эволюция заражённой популяции \( \eta \) описывается уравнением
% \( \frac{\mathrm{d}\eta}{\mathrm{d}t}(t) = \beta (\mathbb{E} k) \eta_t (1-\eta_t) - \delta \eta_t \),
% где \( \mathbb{E} k \) --- средняя степень вершины графа.
Известно, существует пороговое значение \( \tau \) отношения \( {^\beta}/_{\delta} \),
выше которого вспышка вируса не гаснет и может превратиться в эпидемию
(формально: эпидемия гаснет \( \implies ^{\beta}/_{\delta} < \tau \)).
В \cite{epidemic-eigenvalues} показано, что \( \tau = 1/\spr{A} \),
где \( \spr{A} \) -- спектральный радиус мартрицы \( A \) смежностей графа \( G \).

\begin{thebibliography}{9}
\bibitem{baskakov-harmonic}
    \baut{Баскаков}{А.~Г.}
    \btit{Гармонический анализ линейных операторов}[Harmonic analysis of linear operators]
    \bpub{Издательство
        Воронежского Государственного Университета}[
            Voronezh State University Publishing House]
    \bcity{Воронеж}
    \byr{1987.}
    \bpp{93--121}
    \mkbookr
\bibitem{baskakov-split} Баскаков~А.~Г.
    \baut{Баскаков}{А.~Г.}
        \btit{Расщепление возмущённого дифференциального оператора
              с неограниченными операторными коэффициентами}[Analysis
              of linear differential equations by methods
              of the spectral theory of difference operators and linear relations]
        \bj{Фундаментальная и прикладная математика}[Russian Mathematical Surveys]
        \byr{2002}
        \bvol{8}
        \bnum{1}
        \bpp{1--16}
        \mkpaperr
\bibitem{epidemic}
    \baut{Wang}{Yang}
        \baut{Chakrabarti}{D.}
        \baut{Wang}{Chenxi}
        \baut{Faloutsos}{C.}
        \btit{Epidemic spreading in real networks: an eigenvalue viewpoint}
        \bj{22nd International Symposium on Reliable Distributed Systems, Oct 2003. Proceedings}
        \byr{2003}
        \bpp{25--34}
        \mkpapere
\bibitem{cvet}
    \baut{Cvetkovic}{D.~M.}
    \baut{Doob}{M.}
    \baut{Sachs}{H.}
    \btit{Spectra of Graphs: Theory and Applications (3rd revision)}
    \bpub{Wiley}
    \bcity{New York}
    \byr{1998}
    \mkbooke

\end{thebibliography}


\end{document}
