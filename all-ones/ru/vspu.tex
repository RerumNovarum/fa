\documentclass[12pt]{article}

\usepackage{fontspec}
\setmainfont{CMU Serif}
\usepackage{polyglossia}
\setdefaultlanguage{russian}
\setotherlanguage{english}

\usepackage{amsmath}
\usepackage{amsfonts}
\usepackage{mathrsfs}
\usepackage{mathtools}

\usepackage{amsthm}
\theoremstyle{definition}
\newtheorem{dfn}{Определение}
\theoremstyle{plain}
\newtheorem{thm}{Теорема}
\newtheorem{lemma}{Лемма}
\theoremstyle{remark}
\newtheorem{prop}{Утверждение}
\newtheorem{corollary}{Следствие}
\newtheorem{conjecture}{Предположение}

% Mimic vestnik' style
\usepackage{indentfirst}
\oddsidemargin=-3mm
\renewcommand{\baselinestretch}{1.2}
%\evensidemargin=-3mm %
\textwidth 163mm %
\textheight220mm %
\topmargin-8mm %
\sloppy
\let\thefootnote\relax

\usepackage{marginnote}
\usepackage{hyperref}
\usepackage{tabularx}

\begin{document}
\setcounter{thm}{0}
\setcounter{lemma}{0}
\quad УДК 517.984.3 : 519.177

\begin{center}
% The whole point of LaTeX (as opposed to e.g. ConTeXt)
% is to not care about layout and typesetting in the document itself.
% Instead, macro's and environments are used,
% which hide all the details and let you enforce a consistent style
% throughout the whole document by simply modifying
% these environments' definitions.
%
% So the publisher may just ask authors
% ``please, put theorems in `thm` environments (which we'll provide)''.
% This way author does not suffer from useless verbosity
% and an editor can easily adapt an article for publication.
\vskip0.5cm
\textbf{СПЕКТР МАТРИЦЫ СМЕЖНОСТЕЙ ПОЧТИ ПОЛНОГО ОРГРАФА}\\
\vskip0.5cm
\textbf{
    \textbf{Козлуков С.В.\footnotemark} \\
    \textit{<rerumnovarum@openmailbox.org>} \\
    \textit{Воронежский Государственный Университет}
}
\footnotetext{
    \copyright Козлуков С.В.
}
\end{center}

\addcontentsline{tos}{subsection}{{\it{Козлуков С.в.}} Спектр матрицы смежностей почти полного орграфа}
\addcontentsline{toc}{subsection}{{\it{Sergey Kozlukov}} Spectra of an almost-all-ones matrix}

\begin{quote}
    \small{{\bf Аннотация.}
    В статье найдены оценки собственных значений
    матриц смежностей почти полных орграфов.
    }
    \textbf{Ключевые слова:}
    \small{метод подобных операторов, собственные значения, спектр графов}
\end{quote}

\begin{quote}
    \small{{\bf Abstract.}
    This article considers almost-all-ones matrices
.%    and oriented graphs with a large number of edges.
    Their eigenvalues estimates are derived.
    }
    \textbf{Keywords:}
    \small{similar operators method, eigenvalues, graph spectra}
\end{quote}

Пусть \( X \) --- \( M \)-мерное ( \( M < \infty \) ) линейное пространство,
\( A: X\to X \) --- обратимый линейный оператор простой структуры:
\( A \) имеет ровно \( M \) линейно-независимых собственных векторов \( e_1,~\ldots,~e_M \),
которым отвечают, вообще говоря не все различные, ненулевые собственные значения
\( \lambda_1,~\ldots,~\lambda_M \) (см. \cite{baskakov-algebra}).

Рассмотрим оператор
\( \mathbb{A}: X^N\to X^N \), действующий на пространстве \( X^N \) по формуле
\[ \mathbb{A}x =
    \begin{pmatrix}
        A & \cdots & A \\
        \vdots & \ddots & \vdots \\
        A & \cdots & A
    \end{pmatrix}
    \begin{pmatrix}
        x_1 \\
        \vdots \\
        x_N
    \end{pmatrix}
    = \begin{pmatrix}
        A \sum_{i=1}^N x_i \\
        \vdots \\
        A \sum_{i=1}^N x_i
    \end{pmatrix},
    \quad x=(x_1,\ldots,x_N) \in X^N. \]

В статье даны спектр и жорданов базис для операторов такого вида,
а также уточнения для случая блочных матриц, составленных из самосопряжённых блоков.


% A nice bibtex' include was here,
% which used to generate a proper bibliography
% enforcing a single and consistent style;
% In these times of darkness we're not allowed
% to use macros or appropriate package
% or any other kind of generic solution
% --- you hardcode, so that you can't maintain
\begin{thebibliography}{9}
\bibitem{baskakov-harmonic}  Баскаков~А.~Г. Гармонический анализ линейных операторов
    / Баскаков~А.~Г.
    --- Воронеж : Издательство Воронежского Государственного Университета,
        1987.
    ---  с.~93--121.
\bibitem{epidemic-eigenvalues}  Yang Wang and D. Chakrabarti and Chenxi Wang and C. Faloutsos.
    Epidemic spreading in real networks: an eigenvalue viewpoint
        --- 22nd International Symposium on Reliable Distributed Systems, Oct 2003. Proceedings., --- pages~25-34.
\end{thebibliography}
\end{document}
