\documentclass[11pt]{article}

\usepackage{fontspec}
\setmainfont{CMU Serif}
\usepackage{polyglossia}
\setdefaultlanguage{russian}
\setotherlanguage{english}


%\usepackage[a5paper]{geometry}
\usepackage[a5paper,left=20mm, right=20mm, top=20mm, bottom=20mm, includefoot]{geometry}
\usepackage[intlimits]{amsmath}
\interdisplaylinepenalty=2500
\usepackage{amsfonts}
\usepackage{amssymb}
\usepackage{latexsym}

\usepackage{mathrsfs}
\usepackage{mathtools}
\usepackage{amsthm}
\newtheorem{thm}{Теорема.}

\begin{document}
\addcontentsline{toc}{section}{{\it{Козлуков С.В.}} Spectra of an almost-all-ones matrix}
\setcounter{thm}{0}
\small УДК 517.984.3 : 519.177

\begin{center}
\textbf{СПЕКТР МАТРИЦЫ СМЕЖНОСТЕЙ ПОЧТИ ПОЛНОГО ОРГРАФА}\\
\small{Козлуков С.В.} \\
\small{Воронежский Государственный Университет} \\
\small{<rerumnovarum@openmailbox.org>} \\
\end{center}

Пусть \( X \) --- \( M \)-мерное ( \( M < \infty \) ) линейное пространство,
\( A: X\to X \) --- обратимый линейный оператор простой структуры:
\( A \) имеет ровно \( M \) линейно-независимых собственных векторов \( e_1,~\ldots,~e_M \),
которым отвечают, вообще говоря не все различные, ненулевые собственные значения
\( \lambda_1,~\ldots,~\lambda_M \) (см. \cite{baskakov-algebra}).

Рассмотрим оператор
\( \mathbb{A}: X^N\to X^N \), действующий на пространстве \( X^N \) по формуле
\[ \mathbb{A}x =
    \begin{pmatrix}
        A & \cdots & A \\
        \vdots & \ddots & \vdots \\
        A & \cdots & A
    \end{pmatrix}
    \begin{pmatrix}
        x_1 \\
        \vdots \\
        x_N
    \end{pmatrix}
    = \begin{pmatrix}
        A \sum_{i=1}^N x_i \\
        \vdots \\
        A \sum_{i=1}^N x_i
    \end{pmatrix},
    \quad x=(x_1,\ldots,x_N) \in X^N. \]

В статье даны спектр и жорданов базис для операторов такого вида,
а также уточнения для случая блочных матриц, составленных из самосопряжённых блоков.

% A nice bibtex' include was here,
% which used to generate a proper bibliography
% enforcing a single and consistent style;
% In these times of darkness we're not allowed
% to use macros or appropriate package
% or any other kind of generic solution
% --- you hardcode, so that you can't maintain
\begin{thebibliography}{9}
\bibitem{baskakov-harmonic}  Баскаков~А.~Г. Гармонический анализ линейных операторов
    / Баскаков~А.~Г.
    --- Воронеж : Издательство Воронежского Государственного Университета,
        1987.
    ---  с.~93--121.
\bibitem{epidemic-eigenvalues}  Yang Wang and D. Chakrabarti and Chenxi Wang and C. Faloutsos.
    Epidemic spreading in real networks: an eigenvalue viewpoint
        --- 22nd International Symposium on Reliable Distributed Systems, Oct 2003. Proceedings., --- pages~25-34.
\end{thebibliography}


\end{document}
