\documentclass[11pt]{article}

% \usepackage{fontspec}
% \setmainfont{CMU Serif}
% \usepackage{polyglossia}
% \setotherlanguage{english}

\usepackage[utf8]{inputenc}
\usepackage{lmodern}

%\usepackage[a5paper]{geometry}
\usepackage[a5paper,left=20mm, right=20mm, top=20mm, bottom=20mm, includefoot]{geometry}
\usepackage[intlimits]{amsmath}
\interdisplaylinepenalty=2500
\usepackage{amsfonts}
\usepackage{amssymb}
\usepackage{latexsym}

\usepackage{mathrsfs}
\usepackage{mathtools}

\begin{document}
\addcontentsline{toc}{section}{{\it{Козлуков С.В.}} Spectra of almost-all-ones matrices}
\small{UDC 517.984.3 : 519.177}

\begin{center}
\textbf{Spectra of almost-all-ones matrices}\\
\small{S.V. Kozlukov} \\
\small{Voronezh State University} \\
\small{rerumnovarum@openmailbox.org} \\
\end{center}

Let \( \mathscr{A}_{MN} \) be an \( N\times N \) matrix,
consisting of \( N^2 - M \) ones and \( M \) zeroes.
Considered as an adjacency matrix,
\( \mathscr{A}_{MN} \) corresponds to a complete directed graph
with \( N \) vertices,
from which \( M \) edges were ``removed''.
Estimates for eigenvalues of such a matrix are given in this article.

The matrix \( \mathscr{A}_{MN} \) can be represented in the form
\[
    \mathscr{A}_{MN} = \mathscr{J}_{N} - \mathscr{B}_{MN},
    \]
where \(
\mathscr{J}_{N} =
\begin{pmatrix}1 & \cdots & 1 \\
\vdots & \ddots & \vdots \\
1 & \cdots & 1
\end{pmatrix} \) is an all-ones matrix
and \( \mathscr{B}_{MN} \) has unities
exactly at the \( M \) places,
where \( \mathscr{A}_{MN} \) has zeroes and vice versa.

The spectrum of \( \mathscr{J}_{N}, N\in\mathbb{N} \) can be easily computed:
\( \sigma\left({\mathscr{J}_{N}}\right) = \left\{0, N \right\} \).
If the number \( M \) of edges removed
is small enough
we can expect the eigenvalues of
\( \mathscr{A}_{MN} \) to be ``close'' to those of \( \mathscr{J}_{N} \).

Using the Method of Similar Operators (see A.G. Baskakov [1,2])
the main result is achieved:

\textbf{Theorem.} {\it
Let
\(M~<~\displaystyle{N^2/16}. \)

Then the spectrum \( \sigma\left({\mathscr{A}_{MN}}\right) \)
of the matrix \( \mathscr{A}_{MN} \)
can be represented as a union \( \sigma\left({\mathscr{A}_{MN}}\right) = \sigma_1 \cup \sigma_2 \)
of a singleton \( \sigma_1 \subset \mathbb{R} \)
and a set \( \sigma_2 \subset \mathbb{C} \),
which satisfy
\begin{equation*}\begin{aligned}
    & \sigma_1 = \left\{ \lambda_1 \right\}
      \subset \left\{ \lambda\in\mathbb{R}; \lvert \lambda - N\rvert < 4\sqrt{M} \right\}, \\
    & \sigma_2 \subset \left\{\lambda\in\mathbb{C}; \lvert\lambda\rvert <4\sqrt{M} \right\}. \\
\end{aligned}\end{equation*}
}

Complete (almost-complete) graphs model (almost-)fully-connected networks,
which, although rarely, do
occur in some military and other robustness-sensitive applications.

The practical value of knowing the spectrum of adjacency matrix
can be seen in the following discrete-time model of a computer virus spread [3]:
let \( G = (V, E) \) be a connected network (connected digraph)
with the set of nodes (vertices) \( V = \{1, \ldots, N\} \)
and the set of links (edges) \( E \).
During each time-interval every ``infected'' node
attempts to infect its neighbours (adjacent nodes)
with a uniform (the same for the whole network) success probability \( \beta \) (called ``virus birth rate'').
At the same time every infected node might be ``cured'' with probability \( \delta \) (``curing rate'').
It is known that there exists a threshold value \( \tau \) of the ratio \( {^\beta}/_{\delta} \),
above which an infection becomes an epidemic,
and below which the virus dies out (infection probability decreases exponentially in time).
Yang Wang, D. Chakrabarti, Chenxi Wang, and C. Faloutsos have shown in [3]
that \( \tau = 1/\mathrm{spr}\left({A}\right) \),
where \( \mathrm{spr}\left({A}\right) = \max\left\{\lvert \lambda\rvert; \lambda\in\sigma\left(A\right) \right\} \)
denotes the spectral radius of the adjacency matrix \( A \) of the graph \( G \).

% A nice bibtex' include was here,
% which used to generate a proper bibliography
% enforcing a single and consistent style;
% In these times of darkness we're not allowed
% to use macros or appropriate package
% or any other kind of generic solution
% --- you hardcode, so that you can't maintain
\centerline{\textbf{Bibliography}}

1. Baskakov A. G. Harmonic analysis of linear operators //Voronezh Univ., Voronezh. – 1987. -- pp.~93--121.

2. Baskakov A. G. A theorem on splitting an operator, and some related questions in the analytic theory of perturbations
   //Mathematics of the USSR-Izvestiya. – 1987. – Vol. 28. – No 3. – pp.~421.

3. Wang Y. et al. Epidemic spreading in real networks: An eigenvalue viewpoint
   //Reliable Distributed Systems, 2003. Proceedings. 22nd International Symposium on. – IEEE, 2003. – pp.~25-34.



{\bf Sergey Kozlukov}, student of Voronezh State University, Voronezh
\end{document}
