\section{Обозначения}

\begin{tabularx}{\textwidth}{l X}
\( \NN \) & множество натуральных чисел \\
\( \RR \) & поле вещественных чисел \\
\( \CC \) & поле комплексных чисел \\
\( \KK \) & \( \in \{ \RR, \CC \} \) \\
\( \RR^n \) & вещественное линейное пространство размерности \( n \) \\
\( \matr{m\times n}{\KK} \) & линейное пространство матриц размерности \( m\times n \) \\
\( \matr{n}{\KK} \) & = \( \matr{n\times n}{\KK} \) ---
                        банахова (образующая полное, относительно нормы, метрическое пространство)
                        алгебра (линейное пространство с ассоциативной операцией умножения)
                        матриц размерности \( n \) над полем \( \KK \) \\
\( \Hom{\mathscr X}{\mathscr Y} \) & банахова алгебра непрерывных (\(\equiv\) имеющих конечную операторную норму)
                                        линейных операторов
                                        \(\mathscr X\to\mathscr Y\) \\
\( \End{\mathscr X} \) & \( = \Hom{\mathscr X}{\mathscr X} \) \\
\( I \) & тождественный оператор ( единица алгебры \( \End{\mathscr X} \) ) \\
\( E \) & единичная матрица (единица алгебры \(\matr{n}{}\)) \\
\( \spec{A} \) & спектр элемента \( A \) алгебры --- множество таких \( \lambda\in\CC \)
                 для которых элемент \( A - \lambda 1 \) необратим
                 (\( 1 \) -- единица рассматриваемой алгебры) \\
\( \spr{A} \) & спектральный радиус элемента \( A \) --- абсолютная величина
                наибольшего по модулю собственного значения:
                \( \spr{A} = \max_{\lambda\in\spec{A}} \lvert\lambda\rvert \)
\end{tabularx}
