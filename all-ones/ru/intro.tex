\section{Введение}

Статья является введением в язык, основные уравнения и идеологию
метода подобных операторов \cite{baskakov-harmonic}.
Общая схема применяется к простому случаю матрицы из нулей и единиц,
в которой число нулей невелико.
Такая матрица (рассматриваемая как матрица смежностей)
соответствует полному орграфу, с небольшим числом ``удалённых'' рёбер.
Эта матрица рассматривается, как возмущённый случай матрицы,
полностью заполнёной единицами (её спектр легко считается).
Выписываются основные уравнения общей схемы,
сводящиеся к поиску неподвижной точки сжимающего отображения.
Основная теорема, используемая в решении
--- теорема Банаха о неподвижной точке:
\begin{thm}[Банаха]
Пусть \( \mathscr{X} \) --- полное метрическое пространство с метрикой \( d \),
а \( f: \mathscr{X}\to \mathscr{X} \) --- сжимающее отображение,
то есть удовлетворяет условию
\[ d(f(x_1), f(x_2)) \leq q d(x_1,x_2), \]
при всех \( x_1,x_2 \in \mathscr{X} \),
где \( q<1 \) --- фиксированное число, называемое \emph{константой Липшица}.

Тогда существует и при том единственное
решение уравнения \( f(x)=x \).

Это решение можно найти методом простых итераций,
как предел сходящейся последовательности \( \{ x_k; x_k = f(x_{k-1}), k\in\NN \} \)
с произвольным начальным приближением \( x_0 \in \mathscr{X} \).
\end{thm}

Полные (почти-полные) графы --- довольно редкое явление,
но могут описывать, например, топологию полносвязной (близкую к ней) сети,
встречающуюся в задачах, требующих повышенной робастности.

В приложениях вместо спектра матрицы смежности чаще рассматривается
спектрм матрицы-лапласиана \( L = D - A \),
где \( D \) --- (диагональная) матрица степеней вершин,
а \( A \) --- матрица смежности.

Пример приложения спектра матрицы смежности можно увидеть
в следующей модели распространения компьютерного вируса \cite{epidemic-eigenvalues}:
пусть \( G = (V, E) \) --- связная сеть (связный орграф),
\( V = \{1, \ldots, N\} \) --- множество узлов (вершин),
\( E \) --- множество соединений (рёбер).
Время дискретное, в каждый интервал времени каждый заражённый узел
пытается заразить смежные ему узлы с одинаковой для всех узлов вероятностью успеха \( \beta \).
В то же время с вероятностью \( \delta \) каждая заражённая вершина может "исцелиться".
% Эволюция заражённой популяции \( \eta \) описывается уравнением
% \( \frac{\mathrm{d}\eta}{\mathrm{d}t}(t) = \beta (\mathbb{E} k) \eta_t (1-\eta_t) - \delta \eta_t \),
% где \( \mathbb{E} k \) --- средняя степень вершины графа.
Известно, существует пороговое значение \( \tau \) отношения \( {^\beta}/_{\delta} \),
выше которого вспышка вируса не гаснет и может превратиться в эпидемию
(формально: эпидемия гаснет \( \implies ^{\beta}/_{\delta} < \tau \)). 
В \cite{epidemic-eigenvalues} показано, что \( \tau = 1/\spr{A} \),
где \( \spr{A} \) -- спектральный радиус мартрицы \( A \) смежностей графа \( G \).
