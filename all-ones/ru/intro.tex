\section{Введение}

Пусть \( \almostallones{M}{N} \) --- матрица порядка \( N \),
состоящая из \( N^2 - M \) единиц и \( M \) нулей
и пусть число нулей невелико.
Матрица такого вида (рассматриваемая, как матрица смежностей) соответствует
полному орграфу, с небольшим числом ``удалённых'' рёбер.
В статье получена оценка спектра
матриц такого вида.
Здесь под спектром \( \spec{A} \) матрицы \( A \)
понимается множество таких \( \lambda\in\CC \), для которых
матрица \( A - \lambda E \), необратима (\( E \) --- единичная матрица).

Матрица \( \almostallones{M}{N} \) представима в виде
\[
    \almostallones{M}{N} = \allones{N} - \perturbmatrix{M}{N},
    \]
где \(
\allones{N} =
\begin{pmatrix}1 & \cdots & 1 \\
\vdots & \ddots & \vdots \\
1 & \cdots & 1
\end{pmatrix} \) --- матрица единиц,
а \( \perturbmatrix{M}{N} \) имеет единицы в точности на тех \( M \)
местах, на которых в \( \almostallones{M}{N} \) стоят нули.

Спектр матрицы \( \allones{N} \) легко считается:
\( \spec{\allones{N}} = \left\{0, N \right\} \),
а при \( M \ll N^2 \) можно ожидать ``близость'' спектров
\( \almostallones{M}{N} \) и \( \allones{N} \).

Методом подобных операторов \cite{baskakov-harmonic}
достигается основной результат:
\begin{thm}\label{thm:almostallones-spectra}
    Пусть
    \(M~<~\displaystyle{\left(N/4\right)^p}, \)
    для некоторого \( p\in\NN \).

    Тогда спектр \( \spec{\almostallones{M}{N}} \)
    матрицы \( \almostallones{M}{N} \)
    представим в виде объединения непересекающихся
    вещественного одноточечного множества \( \sigma_1 \)
    и множества \( \sigma_2 \),
    лежащих в шарах (интервалах) радиуса \( 4M^{1/p} \)
    с центрами в \( N \) и \( 0 \) соответственно
    \begin{equation}\begin{aligned}
        & \spec{ \almostallones{M}{N} } &&=
        \sigma_1 + \sigma_2 \\
        & \sigma_1 = \left\{\lambda_1\right\} \subset\RR
                   && \lvert\lambda_1 - N\rvert < 4M^{1/p} \\
        & \forall \lambda\in\sigma_2 \subset\CC
                   && \lvert\lambda\rvert < 4M^{1/p}
    \end{aligned}\end{equation}
\end{thm}

Полные (почти-полные) графы --- довольно редкое явление,
но могут описывать, например, топологию полносвязной (близкую к ней) сети,
встречающуюся в задачах, требующих повышенной робастности.

В приложениях вместо спектра матрицы смежности чаще рассматривается
спектрм матрицы-лапласиана \( L = D - A \),
где \( D \) --- (диагональная) матрица степеней вершин,
а \( A \) --- матрица смежности.

Пример приложения спектра матрицы смежности можно увидеть
в следующей модели распространения компьютерного вируса \cite{epidemic-eigenvalues}:
пусть \( G = (V, E) \) --- связная сеть (связный орграф),
\( V = \{1, \ldots, N\} \) --- множество узлов (вершин),
\( E \) --- множество соединений (рёбер).
Время дискретное, в каждый интервал времени каждый заражённый узел
пытается заразить смежные ему узлы с одинаковой для всех узлов вероятностью успеха \( \beta \).
В то же время с вероятностью \( \delta \) каждая заражённая вершина может "исцелиться".
% Эволюция заражённой популяции \( \eta \) описывается уравнением
% \( \frac{\mathrm{d}\eta}{\mathrm{d}t}(t) = \beta (\mathbb{E} k) \eta_t (1-\eta_t) - \delta \eta_t \),
% где \( \mathbb{E} k \) --- средняя степень вершины графа.
Известно, существует пороговое значение \( \tau \) отношения \( {^\beta}/_{\delta} \),
выше которого вспышка вируса не гаснет и может превратиться в эпидемию
(формально: эпидемия гаснет \( \implies ^{\beta}/_{\delta} < \tau \)).
В \cite{epidemic-eigenvalues} показано, что \( \tau = 1/\spr{A} \),
где \( \spr{A} \) -- спектральный радиус мартрицы \( A \) смежностей графа \( G \).
