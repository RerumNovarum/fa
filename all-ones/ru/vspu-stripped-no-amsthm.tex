\documentclass[11pt]{article}

% \usepackage{fontspec}
% \setmainfont{CMU Serif}
% \usepackage{polyglossia}
% \setdefaultlanguage{russian}
% \setotherlanguage{english}
\usepackage[utf8]{inputenc}
\usepackage[russian]{babel}

%\usepackage[a5paper]{geometry}
\usepackage[a5paper,left=20mm, right=20mm, top=20mm, bottom=20mm, includefoot]{geometry}
\usepackage[intlimits]{amsmath}
\interdisplaylinepenalty=2500
\usepackage{amsfonts}
\usepackage{amssymb}
\usepackage{latexsym}

\usepackage{mathrsfs}
\usepackage{mathtools}

\begin{document}
\addcontentsline{toc}{section}{{\it{Козлуков С.В.}} Spectra of an almost-all-ones matrix}
\small{УДК 517.984.3 : 519.177}

\begin{center}
\textbf{СПЕКТР МАТРИЦЫ СМЕЖНОСТЕЙ ПОЧТИ ПОЛНОГО ОРГРАФА}\\
\small{Козлуков С.В.} \\
\small{Воронежский Государственный Университет}
\small{<rerumnovarum@openmailbox.org>} \\
\end{center}

Пусть \( \mathscr{A}_{MN} \) --- матрица порядка \( N \),
состоящая из \( N^2 - M \) единиц и \( M \) нулей
и пусть число нулей в каком-то смысле мало.
Матрица такого вида (рассматриваемая, как матрица смежностей) соответствует
полному орграфу, с небольшим числом ``удалённых'' рёбер.
В статье получена оценка спектра
матриц такого вида.

Матрица \( \mathscr{A}_{MN} \) представима в виде
\[
    \mathscr{A}_{MN} = \mathscr{J}_{N} - \mathscr{B}_{MN},
    \]
где \(
\mathscr{J}_{N} =
\begin{pmatrix}1 & \cdots & 1 \\
\vdots & \ddots & \vdots \\
1 & \cdots & 1
\end{pmatrix} \) --- матрица единиц,
а \( \mathscr{B}_{MN} \) имеет единицы в точности на тех \( M \)
местах, на которых в \( \mathscr{A}_{MN} \) стоят нули.

Спектр матрицы \( \mathscr{J}_{N} \) легко считается:
\( \mathrm{spec}\left({\mathscr{J}_{N}}\right) = \left\{0, N \right\} \),
а при \( M \ll N^2 \) можно ожидать ``близость'' спектров
\( \mathscr{A}_{MN} \) и \( \mathscr{J}_{N} \).

Методом подобных операторов (см. \cite{baskakov-harmonic} и \cite{baskakov-split})
достигается основной результат:

\textbf{Теорема.}

{\it
Пусть
\(M~<~\displaystyle{N^2/16}. \)

Тогда спектр \( \mathrm{spec}\left({\mathscr{A}_{MN}}\right) \)
матрицы \( \mathscr{A}_{MN} \)
представим в виде объединения
одноточечного вещественного множества \( \sigma_1 \)
и множества \( \sigma_2 \),
лежащих в шарах радиуса \( 4\sqrt{M} \)
с центрами в \( N \) и \( 0 \) соответственно
\begin{equation*}\begin{aligned}
    & \mathrm{spec}\left({A-B}\right) = \sigma_1 \cup \sigma_2, \\
    & \sigma_1 = \left\{ \lambda_1 \right\}
      \subset \left\{ \lambda\in\mathbb{R}; \lvert \lambda - N\rvert < 4\beta \right\}, \\
    & \sigma_2 \subset \left\{\lambda\in\mathbb{C}; \lvert\lambda\rvert <4\beta \right\}. \\
\end{aligned}\end{equation*}
}

Полные (почти полные) графы --- довольно редкое явление,
но могут описывать, например, топологию полносвязной (близкую к ней) сети,
встречающуюся в задачах, требующих повышенной робастности.

В приложениях вместо спектра матрицы смежности чаще рассматривается
спектр матрицы-лапласиана \( L = D - A \),
где \( D \) --- (диагональная) матрица степеней вершин,
а \( A \) --- матрица смежности.

Пример приложения спектра матрицы смежности можно увидеть
в следующей модели распространения компьютерного вируса \cite{epidemic-eigenvalues}:
пусть \( G = (V, E) \) --- связная сеть (связный орграф),
\( V = \{1, \ldots, N\} \) --- множество узлов (вершин),
\( E \) --- множество соединений (рёбер).
Время дискретное, в каждый интервал времени каждый заражённый узел
пытается инфецировать смежные ему узлы с одинаковой для всей сети вероятностью успеха \( \beta \).
В то же время с вероятностью \( \delta \) каждая заражённая вершина может "исцелиться".
% Эволюция заражённой популяции \( \eta \) описывается уравнением
% \( \frac{\mathrm{d}\eta}{\mathrm{d}t}(t) = \beta (\mathbb{E} k) \eta_t (1-\eta_t) - \delta \eta_t \),
% где \( \mathbb{E} k \) --- средняя степень вершины графа.
Известно, существует пороговое значение \( \tau \) отношения \( {^\beta}/_{\delta} \),
выше которого вспышка вируса не гаснет и может превратиться в эпидемию
(формально: эпидемия гаснет \( \implies ^{\beta}/_{\delta} < \tau \)).
В \cite{epidemic-eigenvalues} показано, что \( \tau = 1/\mathrm{spr}\left({A}\right) \),
где \( \mathrm{spr}\left({A}\right) \) --- спектральный радиус матрицы \( A \) смежностей графа \( G \).
% A nice bibtex' include was here,
% which used to generate a proper bibliography
% enforcing a single and consistent style;
% In these times of darkness we're not allowed
% to use macros or appropriate package
% or any other kind of generic solution
% --- you hardcode, so that you can't maintain
\begin{thebibliography}{9}
\bibitem{baskakov-harmonic}  Баскаков~А.~Г. Гармонический анализ линейных операторов
    / Баскаков~А.~Г.
    --- Воронеж : Издательство Воронежского Государственного Университета,
        1987.
    ---  с.~93--121.
\bibitem{baskakov-split} Баскаков~А.~Г., Теорема о расщеплении оператора и некоторые смежные вопросы аналитической теории возмущений
    / Баскаков~А.~Г.
    --- Изв. АН СССР. Сер. матем., 50:3, 1986
    --- с. 435-437
\bibitem{epidemic-eigenvalues}  Yang Wang and D. Chakrabarti and Chenxi Wang and C. Faloutsos.
    Epidemic spreading in real networks: an eigenvalue viewpoint
        --- 22nd International Symposium on Reliable Distributed Systems, Oct 2003. Proceedings., --- pages~25-34.
\end{thebibliography}

\end{document}
