Матрицы из \( \mathtt{Matr}_N\mathbb{C} \) будем записывать в блочном виде
\( X \sim
    \begin{pmatrix}
    x_{11} & X_{12} \\
    X_{21} & X_{22}
    \end{pmatrix}, \)
    где \( x_{11} \) --- число,
    \( X_{12} \) --- строка, \( X_{21} \) --- столбец,
    \( X_{22} \) --- квадратный блок размерности \( N-1 \).
Такие блочные матрицы сами образуют алгебру, изоморфную исходной
и их можно естественным образом умножать
на элементы пространства \( \mathbb{C}\times\mathbb{C}^{N-1} \),
изоморфного \( \mathbb{C}^N \),
поэтому далее мы не будем их никак различать.

Следуя общей схеме метода подобных операторов \cite{baskakov-split},
будем искать более ``простую'' матрицу, подобную \( \mathcal{A} - \mathcal{B} \),
в виде \( A - \mathfrak{J} X \)
с матрицей преобразования подобия \( E + \Gamma X \),
где \( E \) --- единичная матрица,
\( \mathfrak{J},\Gamma \) --- линейные операторы,
действующие на алгебре \( \mathtt{Matr}_N\mathbb{C} \),
подбираемые в ходе решения так, чтобы упростить \( \mathcal{A} - \mathfrak{J}X \).
Матрица \( \mathcal{A} \) при этом интерпретируется как ``идеальная'',
а \( \mathcal{B} \) считается возмущением.
Короче, будем решать в банаховой алгебре матриц порядка \( N \) уравнение
\begin{equation}\label{simeqn}
    (\mathcal{A-B})(E+\Gamma X) = (E+\Gamma X)(\mathcal{A} - \mathfrak{J} X), \quad X\in\mathtt{Matr}_N\mathbb{C}.
    \end{equation}
Оператор \( \mathfrak{J} \) обычно выбирают проектором (\(\mathfrak{J}^2=\mathfrak{J}\)).
\( \Gamma \) определяют поточечно, как решение уравнения
\( A\Gamma X - (\Gamma X) A = A - \mathfrak{J} X, \quad X\in\mathtt{Matr}_N\mathbb{C} \).
Ясно, \( \mathcal{A}-\mathcal{B} \) тем ближе к идеальной \( \mathcal{A} \),
чем шире ядро оператора \( \mathfrak{J} \).
Уравнение для \( \Gamma \) в свою очередь не позволяет сузить ядро слишком сильно.

Пусть \( \Gamma \) действует по формуле
\( \Gamma X = \begin{pmatrix} \Gamma_{11}(X) & \Gamma_{12}(X) \\
                              \Gamma_{21}(X) & \Gamma_{22}(X)
                              \end{pmatrix} \), тогда
\[
    \mathcal{A} \Gamma X - (\Gamma X)\mathcal{A} = 
    \begin{pmatrix} 0 & N\Gamma_{12}(X) \\
        - N\Gamma_{21}(X) & 0
        \end{pmatrix}, \]
\[
    X - \mathfrak{J} X =
    N \begin{pmatrix} 0 & \Gamma_{12}(X) \\
        - \Gamma_{21}(X) & 0
        \end{pmatrix}.
    \]

Положим
\[
    \mathfrak{J} X = \begin{pmatrix} x_{11} & 0 \\ 0 & X_{22} \end{pmatrix}, \]
\[
    \mathfrak{\Gamma} X = \frac{1}{N}\begin{pmatrix} 0 & X_{12} \\ -X_{21} & 0 \end{pmatrix}. \]

Ясно, что при этом спектр блочно-диагональной матрицы \( \mathcal{A} - \mathfrak{J}X \)
есть объединение спектров её диагональных блоков.

Уравнение \eqref{simeqn} можно свести к уравнению
\[ X = \mathcal{B} \Gamma X + \mathcal{B} - (\Gamma X) \mathfrak{J} X. \]
Применим к обеим частям уравнения оператор \( \mathfrak{J} \) и получим выражение
 \( (\Gamma X) \mathfrak{J} X = (\Gamma X)(\mathfrak{J}(\mathcal{B} (\Gamma X + E))) \),
 которое подставим обратно в уравнение:

\begin{equation}\label{fixptneqn}
    X = \Phi(X) \equiv \mathcal{B} \Gamma X + \mathcal{B} - (\Gamma X)(\mathfrak{J}(\mathcal{B} (\Gamma X + E))), \quad X\in\mathtt{Matr}_N\mathbb{C}.
\end{equation}

Теперь покажем, что, при определённых условиях,
возникшее отображение \( \Phi \) инвариантно
относительного некоторого шара с центром в нуле,
%(т.е. \(\Phi\) отображает этот шар в самого себя),
на котором оно является сжимающим.

Пусть в \( \mathtt{Matr}_N\mathbb{C} \)
выбрана какая-нибудь субмультипликативная норма \( \|\cdot\| \)
(т.е. норма, при всех \( \mathcal{A}_1, \mathcal{A}_2 \in \mathtt{Matr}_N\mathbb{C} \)
 удовлетворяющая неравенству
 \( \| \mathcal{A}_1\mathcal{A}_2 \| \leq \|\mathcal{A}_1\|\|\mathcal{A}_2\| \)).
Нам нужно найти такой радиус \( r \geq 0 \),
что при \( \|X\|,\|Y\| \leq r \) выполнялись бы условия \( \|\Phi(X)\| \leq \|X\| \)
и \( \|\Phi(X) - \Phi(Y)\| < 1 \).
Обозначим
\( \beta = \|\mathcal{B}\| \), \( \gamma = \sup_{\|X\|=1} \|\Gamma X\| \).
Очевидно неравенство


\[ \| \Phi(X) \| \leq
     \beta \gamma^2 \|X\|^2 + 2\beta\gamma\|X\| + \beta. \]

\begin{lemma}
    Пусть \( \gamma\beta < \frac14\),
    тогда шар
    \[
        \Omega = \left\{ X\in \mathtt{Matr}_N\mathbb{C}; \|X\| \leq r_0 \right\}, \]
    \[  r_0 = \frac{1 - 2\gamma\beta - \sqrt{1-4\gamma\beta}}{2\gamma^2\beta} < 4\beta, \]
    удовлетворяет условию \( \Phi(\Omega)\subset\Omega \).
\end{lemma}

Аналогичным образом устанавливается
\begin{lemma}
    Пусть \(\gamma\beta<\frac14\),
    тогда \( \Phi \) --- сжимающее отображение:
    \[ \| \Phi(X) - \Phi(Y) \| \leq q \|X - Y\|, \]
    \[ q = (1+2\gamma r_0) \gamma\beta \leq (1+8\gamma\beta)\gamma\beta \leq \frac34. \]
\end{lemma}
Здесь использовано равенство
\[ (\Gamma X) \mathfrak{J}(\mathcal{B}\Gamma X) - (\Gamma Y) \mathfrak{J}(\mathcal{B}\Gamma Y) =
    \frac12\left[
        \Gamma(X-Y) \mathfrak{J}(\mathcal{B}\Gamma(X+Y))
    +   \Gamma(X+Y) \mathfrak{J}(\mathcal{B}\Gamma(X-Y))
    \right]. \]

Таким образом, по теореме Банаха о неподвижной точке,
в шаре \( \Omega = \left\{ X\in\mathtt{Matr}_N\mathbb{C}; \|X\| \leq r_0 \right\} \)
существует и при том единственное решение \( X^o \) уравнения \eqref{fixptneqn},
являющееся пределом последовательности \( \{ \Phi^k(\mathcal{0}); k\in\mathbb{N} \} \),
где \( \Phi^k = \Phi\circ\Phi^{k-1} \) --- композиция.

\begin{lemma}
Матрица \( \mathcal{A} - \mathcal{B} \) подобна блочно-диагональной матрице \( \mathcal{A} - \mathfrak{J} X^o \):
\[ \mathcal{A} - \mathcal{B} \sim
\begin{pmatrix}
N - x_{11}^o & 0 \\
0 & -X_{22}^o
\end{pmatrix}, \]
и её собственные значения удовлетворяют условиям:
\[ \sigma\left(\mathcal{A} - \mathcal{B}\right) = \left\{N-x_{11}^o\right\}\cup \sigma\left(-X_{22}^o\right), \]
\[ \lambda_1 = N-x_{11}^o \in \{ \mu\in\mathbb{R}; \lvert x \rvert < r_0 \leq 4\beta \}, \]
\[ \sigma\left(-X_{22}^o\right) \subset \{ \mu\in\mathbb{C}; \lvert x \rvert < r_0 \leq 4\beta \}. \]
\end{lemma}
\begin{proof}
    Матрица \( \mathcal{A} - \mathcal{B} \) подобна блочно-диагональной \( \mathcal{A} - \mathfrak{J} X^o \),
    поэтому их спектры совпадают.
    Спектр матрицы \( \mathcal{A} - \mathfrak{J} X^o \) есть объединение спектров её диагональных блоков.
    В виду субмультипликативности нормы имеют место неравенства
    \[ \mathtt{spr}(X^o) = \max_{\lambda\in\sigma(X^o)}\lvert\lambda\rvert \leq \|X^o\| \leq r_0. \]
    Кроме того, собственное значение \( x_{11}^o \) является вещественным, как предел сходящейся вещественной последовательности.
\end{proof}
