Следуя общей схеме, мы будем искать матрицу, подобную \( A - B \)
в виде
\[ A - J X, \]
с матрицей подобия вида
\[ E + \Gamma X ,\]
где
\( J, \Gamma \in \mathrm{End}\left(\mathfrak U\right),\quad \mathfrak{U}\subset\mathrm{Matr}_{N} \)
--- линейные операторы (трансформаторы)
\footnote{\( \mathrm{End}\left(\mathfrak U\right) \) --- пространство непрерывных линейных операторов,
действующих на подпространстве \( \mathfrak U \) пространства матриц порядка \( N \)},
действующие на некотором подпространстве \( \mathfrak U \) алгебры \( \mathrm{Matr}_{N} \),
называемом пространством допустимых возмущений,
и подбираемые в ходе решения так, чтобы матрица \( A - JX \)
имела как можно более простую структуру.

Иначе говоря, будем искать решение \( X \in\mathrm{Matr}_{N} \)
уравнения подобия
\begin{equation}\label{eq:similarity-orig}
    (A - B)(E+\Gamma X) = (E+\Gamma X) (A - JX).
\end{equation}

В подпространстве \( \mathfrak U \subset \mathrm{Matr}_{N} \)
мы будем запускать итерационный процесс,
и, как будет видно ниже, потребуется \( B \in \mathfrak U \)
и замыкания образов возникающих далее операторов также
должны содержаться в
пространстве допустимых возмущенний.
В виду способа построения матрицы \( B \), сказать что-либо о её структуре затруднительно,
поэтому мы вынуждены выбрать \( \mathfrak U = \mathrm{Matr}_{N}\).

Пространство \( \mathscr{X} \) представим в виде
декартова произведения
\( \mathscr{X}_1~\times~\mathscr{X}_2 \)
одномерного пространства \( \mathscr{X}_1=\mathbb{R} \)
и \((n-1)\)-мерного пространства \( \mathscr{X}_2 \),
элементы \( \mathscr{X} \) в виде
\( x=\begin{pmatrix}x_1\\x_2\end{pmatrix}, \)
\( x_i\in\mathscr{X}_i \).
Матрицы из \( \mathrm{Matr}_{N} \) будем записывать
в блочном виде:
\[
    X = \begin{pmatrix}
    X_{11} & X_{12} \\
    X_{21} & X_{22}
    \end{pmatrix},
    \]
где \( { X_{11}\in\mathbb{R} } \) -- число,
    \( X_{12} \) -- строка длины \( N-1 \),
    \( X_{21} \) -- столбец,
    \( X_{22} \) -- матрица порядка \( N-1 \).
При этом произведению матриц соответствует произведение блочных матриц.

Вернёмся к задаче.
Оператор \( \Gamma X \) определим как решение уравнения\footnotemark
\[
    A\Gamma X - (\Gamma X) A = X - JX,\quad X\in\mathrm{Matr}_{N}
    \]%
\footnotetext{в левой части стоит \( \mathrm{ad}_{A}\left({\Gamma X}\right) \)
--- оператор коммутирования с \(A\), применённый к \(\Gamma X\): \( \mathrm{ad}_{Y}\left({Z}\right) = YZ - ZY \)}%
Пусть
\[ \Gamma X =
\begin{pmatrix}
    \Gamma_{11}(X) & \Gamma_{12}(X) \\
    \Gamma_{21}(X) & \Gamma_{22}(X)
\end{pmatrix}, \]
тогда
\[
    A \Gamma X - (\Gamma X) A =
    N
    \begin{pmatrix}
        0               & \Gamma_{12}(X) \\
        -\Gamma_{21}(X) & 0
    \end{pmatrix},
\]
и уравнение приводится к виду
\[
    X - JX =
    N
    \begin{pmatrix}
        0               & \Gamma_{12}(X) \\
        -\Gamma_{21}(X) & 0
    \end{pmatrix}.
\]
Можем положить
\[
    JX = \begin{pmatrix}
        X_{11} & 0 \\
        0      & X_{22}
    \end{pmatrix},\quad
    \Gamma X =
    \frac1N
    \begin{pmatrix}
        0       & X_{12} \\
        -X_{21} & 0
    \end{pmatrix}.
    \]
При этом матрица \( A - JX \) оказывается блочно-диагональной
и её спектр есть объединение спектров её диагональных блоков.
Заметим, естественнее было бы начать с выбора оператора \( J \) доставляющего структуру,
как можно более близкую к \( A \), а именно
оператор \( X\mapsto X_{11} \).
К сожалению, такой выбор \( J \) не позволяет решить
уже уравнение для \(\Gamma X\) во всём \(\mathfrak U = \mathrm{Matr}_{N}\),
и мы приходим к блочно-диагональному виду.

Уравнение \eqref{eq:similarity-orig} перепишем в виде

\[
    A - B + A\Gamma X - B\Gamma X = A - JX + (\Gamma X) A - (\Gamma X) JX
    \]
\[
    A\Gamma X - (\Gamma X) A + JX = B \Gamma X + B - (\Gamma X) JX
    \]
\[
    X = B \Gamma X + B - (\Gamma X) JX
    \]
Применяя слева оператор \( J \) и подставляя результат обратно, получим уравнение
\begin{equation}\label{eq:similarity2}
    X = B \Gamma X + B - (\Gamma X) (J(B\Gamma X + B))
\end{equation}

Теперь мы докажем существование единственного решения уравнения \eqref{eq:similarity2}
и выведем оценки спектра,
показав, что в правой части этого уравнения стоит образ \( X \)
при сжимающем (в некотором шаре с центром в нуле) отображении \( \Phi \):
\[ \Phi(X) = B \Gamma X + B - (\Gamma X) (J(B\Gamma X + B)) \]

Покажем сначала существование такого шара \( \{ X\in\mathrm{Matr}_{N}; {\left\|X\right\|} \leq \varepsilon \} \),
который содержит свой образ при отображении \( \Phi \)
(другими словами, матрицы из этого шара
 не покидают его под действием \( \Phi \))

Будем считать, что в алгебре матриц \( \mathrm{Matr}_{N} \)
введена некоторая субмультипликативная норма%
\footnote{субмультипликативной называют норму, удовлетворяющую неравенству \( {\left\|XY\right\|}\leq {\left\|X\right\|}{\left\|Y\right\|} \) при всех \( X, Y \) },
а в пространстве \( \mathrm{End}\left({\mathfrak U}\right) \) трансформаторов
используется обычная операторная норма%
\footnote{операторная норма
(матрицы или оператора) \( \mathcal A \)
вводится как точная нижняя грань таких констант \( C > 0 \),
что для всех \( x\in\mathscr X\) имеет место
(\( {\left\|A x\right\|} \leq C {\left\|x\right\|} \))}.
Тогда имеют место оценки

\begin{align*}
    {\left\|\Phi(X)\right\|} &= {\left\|B \Gamma X + B - (\Gamma X) (J(B\Gamma X + B))\right\|} \leq \\
%    &\leq {\left\|B\right\|}{\left\|\Gamma\right\|}{\left\|X\right\|} + {\left\|B\right\|} + {\left\|\Gamma\right\|}{\left\|X\right\|} ({\left\|B\right\|}{\left\|\Gamma\right\|}{\left\|X\right\|}+{\left\|B\right\|}) \leq \\
    &= \gamma^2\beta{\left\|X\right\|}^2 + 2\gamma\beta{\left\|X\right\|} + \beta,
\end{align*}
где \[ \beta={\left\|B\right\|}, \gamma={\left\|\Gamma\right\|} \]
Пока что отложим оценку этих величин.

\begin{lemma}
    Если
    \( \gamma\beta \leq \frac14, \)
    то \( \Phi \) отображает в себя шар
    \begin{equation}\label{def:omega}
        \Omega = \{ X\in\mathrm{Matr}_{N} ; {\left\|X\right\|}\leq r_0 \},
    \end{equation}
    \[0 \leq r_0 = \frac{1 - 2\gamma\beta - \sqrt{1-4\gamma\beta}}{2\gamma^2\beta} < 4\beta, \]
\end{lemma}
\begin{proof}
    Обозначим \( r={\left\|X\right\|} \). Тогда
    \( {\left\|\Phi(X)\right\|} \leq {\left\|X\right\|} \), если
    \begin{equation}\label{eq:omega-invariance}
        \gamma^2\beta r^2 + (2\gamma\beta - 1) r + \beta \leq 0.
    \end{equation}
    Детерминант возникшего многочлена:
    \( \Delta = 1 - 4\gamma\beta\).\-
    Стало быть, если \( {\gamma\beta \leq \frac14} \),
    то определённое выше \( r_0 \) есть наименьшее \( r \),
    удовлетворяющее неравенству \eqref{eq:omega-invariance}.
    Непосредственно проверяется, что \( 0 \leq  r_0 \leq 4\beta \).
\end{proof}

Определим теперь, при каких условиях \( \Phi \) будет сжимающим в найдённом шаре.
\begin{lemma}
    Пусть \( \gamma\beta \leq \frac14, \)
    тогда \( \Phi \) есть сжимающее отображение \(\Omega\to\Omega\) шара \(\Omega\) в себя:
    \[ {\left\|\Phi(X)-\Phi(Y)\right\|} \leq q {\left\|X-Y\right\|}, \quad q<1.\]
\end{lemma}
\begin{proof}
    Имеют место:
    \begin{align*}
        \Phi(X) - \Phi(Y) &= B\Gamma (X-Y) + \\
        &- \frac12 \Gamma (X + Y)J(B\Gamma(X - Y)) + \\
        &- \frac12 \Gamma (X - Y)J(B\Gamma(X + Y)),
        \marginnote{\footnotemark}
    \end{align*}
    \[
        {\left\|\Phi(X) - \Phi(Y)\right\|} \leq (1 + \gamma{\left\|X+Y\right\|})\beta\gamma {\left\|X-Y\right\|} \leq q{\left\|X-Y\right\|}.
        \]
    Здесь
    \begin{equation*}\label{eq:lipconst}
        q = (1+2\gamma r_0)\gamma\beta
        \leq (1+8\gamma\beta)\gamma\beta \leq \frac{3}{4}.
        \marginnote{\footnotemark}
    \end{equation*}
    \footnotetext{Здесь используются неравенства \(r_0\leq 4\beta\) и \(\gamma\beta \leq \frac14\)}
    Это и значит, что \( \Phi \) --- сжимающее отображение в найдённом шаре.
\end{proof}

\begin{lemma}
    Пусть \( {\left\|\cdot\right\|} \) --- какая-нибудь субмультипликативная норма в \( \mathrm{Matr}_{N} \),
    \( \beta={\left\|B\right\|}, \gamma=\sup_{{\left\|X\right\|}=1}{\left\|\Gamma X\right\|} \)
    и пусть
    \( \gamma\beta\leq\frac14. \)

    Тогда в \( \Omega \) существует единственное решение
    \( X^o = \begin{pmatrix}
        x_{11}^o & X_{12}^o \\
        X_{21}^o & X_{22}^o
    \end{pmatrix}\) уравнения подобия
    \eqref{eq:similarity2}.
    Спектр матрицы \( A - B \) представим в виде
    \( \left\{ N - x_{11}^o \right\} \cup \mathrm{spec}\left({-X_{22}^o}\right) \).
    При этом имеют место оценки:
    \( \lvert x_{11} \rvert, \mathrm{spr}\left({-X_{22}^o}\right) \leq \mathrm{spr}\left({X^o}\right)
    \leq {\left\|X^o\right\|} \leq r_0 \leq 4\beta.
        \)
    \footnotetext{\( \mathrm{spr}\left({X}\right) \) означает спектральный радиус матрицы \( X \in\mathrm{Matr}_{N} \)
    --- абсолютную величину её максимального по модулю собственного значения}
\end{lemma}
\begin{proof}
    (Сужение) \( \Phi \) есть сжимающее отображение \( \Omega\to\Omega \)
    замкнутого подмножества \( \Omega \)
    банахова пространства \( \mathrm{Matr}_{N} \) в себя.
    По теореме Банаха о неподвижной точке, в этом шаре существует единственное решение \( X^o \)
    уравнения \( X = \Phi (X) \),
    доставляемое методом простых итераций (c нулём в качестве начального приближения),
    как предел сходящейся последовательности
    \[
        \{X_k = \Phi^k(0); k\in\mathbb{N}\},
        \]
    где \( \Phi^k = \Phi \circ \Phi^{k-1}, \)
    \( \Phi^0 = I \) (\( I \) --- тождественный оператор).
    Норма этого решения не превосходит радиус \( r_0 \) шара \( \Omega \),
    а собственные значения по модулю не превосходят нормы \cite{baskakov-harmonic}.
    Кроме того, имеет место оценка скорости сходимости
    \[
        {\left\|X_k - X^o\right\|} \leq \frac{q^k}{1-q} {\left\|\Phi(0) - 0\right\|} = \frac{q^k}{1-q}{\left\|B\right\|},
        \]
    \[
        {\left\|X_k - X^o\right\|} \leq \frac{3^k}{4^{k-1}}\beta.\marginnote{\footnotemark}
        \]

    Значит, матрица \( A-B \) подобна блочно-диагональной матрице \( A - JX^o \):
    \[
        A-B \sim
    \begin{pmatrix}
        N - x_{11}^o & 0 \\
        0            & -X_{22}^o
    \end{pmatrix},
    \]
    поэтому спектр матрицы \( A - B \) есть объединение спектров
    диагональных блоков матрицы \( A - JX^o \):
    \[ \mathrm{spec}\left({A-B}\right) = \left\{N - x_{11}^o\right\} \cup \mathrm{spec}\left({-X_{22}^o}\right). \]
\end{proof}
