Рассмотрим возмущённую матрицу единиц
\( \mathscr{A}_{MN} \) размерности \( N \),
состоящую из \( N^2 - M \) единиц и \( M \) нулей.

Её можно представить в виде
\[
    \mathscr{A}_{MN} = \mathscr{J}_{N} - \mathscr{B}_{MN}
    \]
Где \( \mathscr{J}_{N} \) --- матрица единиц,
а \( \mathscr{B}_{MN} \) -- матрица-возмущение,
имеющая единицы в точности в тех местах,
на которых в \( \mathscr{A}_{MN} \) стоят нули.

\begin{prop}
    Подобные матрицы имеют одинаковый спектр
\end{prop}

Применяя полученную в \eqref{eq:diagtransform} матрицу преобразования,
получаем, что \( \mathscr{A}_{MN} \) подобна (а значит, имеет такой же спектр)
матрице
\( A - B, \)
где
\[
    A = {U^{-1}}{\mathscr{J}_{N}}{U}
    = \begin{pmatrix}
      N &   &        & \\
        & 0 &        & \\
        &   & \ddots & \\
        &   &        & 0
        \end{pmatrix}
        \]
\[
    B = {U^{-1}}{\mathscr{B}_{MN}}{U}
    \]
Снова, пропущенные элементы матрицы -- нулевые.
