Доказательство состоит в~построении уравнения для матрицы, подобной \( \mathscr{A}_{MN} \),
 но устроеной ``проще''. Решение возникающего нелинейного уравнения
 в~банаховой алгебре \( \mathtt{Matr}_N\mathbb{C} \)
 доставляется методом простых итераций (см., например,~\cite{baskakov-harmonic}).

Подобие матриц \( \mathcal{A}_1, \mathcal{A}_2 \)
 понимается в~смысле существования обратимой матрицы \( \mathcal{U} \),
 такой что \( \mathcal{A}_1 \mathcal{U} = \mathcal{U} \mathcal{A}_2 \).
Подобные матрицы изоспектральны (их спектры совпадают).

Провед\"ем предварительные преобразования.

\begin{lem}
    Матрица единиц 
    \( \mathcal{J}_N =
    \begin{pmatrix}
        1 & \cdots & 1 \\
        \vdots & \ddots & \vdots \\ 
    1 & \cdots & 1 \end{pmatrix} \),
    подобна матрице
    \[
        \mathcal{A} = \begin{pmatrix}
            N & 0 & \cdots & 0 \\
            0 & 0 & \cdots & 0 \\
            \vdots & \vdots & \ddots & \vdots \\
            0 & 0 & \cdots & 0 \end{pmatrix}. \]
    Точнее, существует ортогональная матрица \( \mathcal{U} \),
    такая что
    \( \mathcal{J}_N = \mathcal{U}\mathcal{A} \mathcal{U}^{-1} \).
\end{lem}
\begin{proof}
    Собственному значению \( 0 \) соответствует \( N-1 \) независимый собственный вектор
        \( f_1 = {\left(1,-1,0,\ldots,0\right)}, \ldots,
           f_{N-1} = {\left(0,\ldots,0,1,-1\right)} \),
    а~собственному значению \( N \) матрицы \( \mathcal{J}_N \) 
    соответствует собственный вектор \( f_N = {\left(1,\ldots,1\right)} \).
    Применив ортогонализацию Грамма-Шмидта, получим ортонормальную систему \( h_1, \ldots, h_N \):
    \[
        h_k = \frac{1}{\sqrt{k(k+1)}}
            \left(\smash{\underbrace{1,~\ldots,~1,}_{k \text{ раз}}}~-k,~0,~\ldots,~0\right)
            \in \mathbb{R}^N, \quad k={1, \ldots, N-1} \]
    \[
        h_N = {\left(1,~\ldots,~1\right)} \in \mathbb{R}^N, \]
    В~качестве матрицы \( \mathcal{U} \) выберем матрицу,
    имеющую столбцами векторы \( h_N, h_1, \ldots, h_{N-1} \):
    \[ \mathcal{U} =
    \begin{pmatrix}
        \frac{1}{\sqrt N} &  \frac{1}{\sqrt2} &  \frac{1}{\sqrt{6}} & \cdots & \frac{1}{\sqrt{(N-2)(N-1)}} & \frac{1}{\sqrt{N(N-1)}} \\
        \frac{1}{\sqrt N} & -\frac{1}{\sqrt2} &  \frac{1}{\sqrt{6}} & \cdots & \frac{1}{\sqrt{(N-2)(N-1)}} & \frac{1}{\sqrt{N(N-1)}} \\
        \frac{1}{\sqrt N} & 0                 & -\frac{2}{\sqrt{6}} & \cdots & \frac{1}{\sqrt{(N-2)(N-1)}} & \frac{1}{\sqrt{N(N-1)}} \\
        \frac{1}{\sqrt N} & 0                 &  0                  & \cdots & \frac{1}{\sqrt{(N-2)(N-1)}} & \frac{1}{\sqrt{N(N-1)}} \\
        \vdots            & \vdots            &  \vdots             & \ddots & \vdots                      & \vdots   \\
        \frac{1}{\sqrt N} & 0                 &  0                  & \cdots & \frac{1}{\sqrt{(N-2)(N-1)}} & \frac{1}{\sqrt{N(N-1)}} \\
        \frac{1}{\sqrt N} & 0                 &  0                  & \cdots & \frac{2-N}{\sqrt{(N-2)(N-1)}} & \frac{1}{\sqrt{N(N-1)}} \\
        \frac{1}{\sqrt N} & 0                 &  0                  & \cdots & 0                  & \frac{1-N}{\sqrt{N(N-1)}}
    \end{pmatrix}.\]
\end{proof}

Таким образом, исходная матрица \( \mathcal{A}_{MN} \) подобна матрице
\( \mathcal{A} - \mathcal{B} \), где \( \mathcal{B} = \mathcal{U}^{-1} \mathscr{B}_{MN} \mathcal{U} \).
Далее ортогональность матрицы \( U \) будет играть важную роль.
