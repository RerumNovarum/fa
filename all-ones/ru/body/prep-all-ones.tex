Доказательство состоит в построении уравнения для матрицы, подобной \( \mathscr{A}_{MN} \),
но устроеной ``проще''. Решение возникающего уравнения в банаховой алгебре \( \mathtt{Matr}_N\mathbb{C} \)
доставляется методом простых итераций (см. \cite{baskakov-harmonic}).

Подобие матриц \( \mathcal{A}_1, \mathcal{A}_2 \)
понимается в смысле существования обратимой матрицы \( \mathcal{U} \),
такой что \( \mathcal{A}_1 \mathcal{U} = \mathcal{U} \mathcal{A}_2 \).
Спектры подобные матриц совпадают.

\begin{lemma}
    Матрица единиц \( \mathcal{J}_N \) подобна матрице
    \[
        \mathcal{A} = \begin{pmatrix}
            N & 0 & \cdots & 0 \\
            0 & 0 & \cdots & 0 \\
            \vdots & \vdots & \ddots & \vdots \\
            0 & 0 & \cdots & 0
        \end{pmatrix}. \]
    Точнее, существует ортогональная матрица \( \mathcal{U} \),
    такая что
    \( \mathcal{J}_N \mathcal{U} = \mathcal{U} \mathcal{A} \).
\end{lemma}
\begin{proof}
    Собственному значению \( N \) матрицы \( \mathcal{J}_N \) 
    соответствует собственный вектор \( f_1 = {\left(1,\ldots,1\right)} \),
        а собственному значению \( 0 \), \( N-1 \) независимый собственный вектор
        \( f_2 = {\left(1,-1,0,\ldots,0\right)}, ...,
           f_N = {\left(0,\ldots,0,1,-1\right)} \).
    Применив ортогонализацию Грамма-Шмидта, получим ортонормальную систему \( h_1, \ldots, h_N \).
    Эти координатные векторы и являются столбцами матрицы \( \mathcal{U} \).
\end{proof}

Таким образом, исходная матрица \( \mathcal{A}_{MN} \) подобна матрице
\( \mathcal{A} - \mathcal{B} \), где \( \mathcal{B} = \mathcal{U}^{-1} \mathscr{B}_{MN} \mathcal{U} \).
