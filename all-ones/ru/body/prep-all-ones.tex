Доказательство состоит в построении уравнения для матрицы, подобной \( \mathscr{A}_{MN} \),
 но устроеной ``проще''. Решение возникающего нелинейного уравнения
 в банаховой алгебре \( \mathtt{Matr}_N\mathbb{C} \)
 доставляется методом простых итераций (см. \cite{baskakov-harmonic}).

Подобие матриц \( \mathcal{A}_1, \mathcal{A}_2 \)
 понимается в смысле существования обратимой матрицы \( \mathcal{U} \),
 такой что \( \mathcal{A}_1 \mathcal{U} = \mathcal{U} \mathcal{A}_2 \).
Спектры подобные матриц совпадают.

Провед\"ем предварительные преобразования.

\begin{lem}
    Матрица единиц 
    \( \mathcal{J}_N =
    \begin{pmatrix}
        1 & \cdots & 1 \\
        \vdots & \ddots & \vdots \\ 
    1 & \cdots & 1 \end{pmatrix} \),
    подобна матрице
    \[
        \mathcal{A} = \begin{pmatrix}
            N & 0 & \cdots & 0 \\
            0 & 0 & \cdots & 0 \\
            \vdots & \vdots & \ddots & \vdots \\
            0 & 0 & \cdots & 0 \end{pmatrix}. \]
    Точнее, существует ортогональная матрица \( \mathcal{U} \),
    такая что
    \( \mathcal{A} = \mathcal{U}\mathcal{J}_N \mathcal{U}^{-1} \).
\end{lem}
\begin{proof}
    Собственному значению \( N \) матрицы \( \mathcal{J}_N \) 
    соответствует собственный вектор \( f_1 = {\left(1,\ldots,1\right)} \),
        а собственному значению \( 0 \) соответствует \( N-1 \) независимый собственный вектор
        \( f_2 = {\left(1,-1,0,\ldots,0\right)}, ...,
           f_N = {\left(0,\ldots,0,1,-1\right)} \).
    Применив ортогонализацию Грамма-Шмидта, получим ортонормальную систему \( h_1, \ldots, h_N \).
    В качестве матрицы \( \mathcal{U} \) выберем матрицу,
    имеющую столбцами векторы \( h_1, \ldots, h_N \):
    \[ \mathcal{U} =
    \begin{pmatrix}
        \frac{1}{\sqrt N} &  \frac{1}{\sqrt2} &  \frac{1}{\sqrt{6}} &   \alpha_3 & \cdots & \alpha_{N-1} \\
        \frac{1}{\sqrt N} & -\frac{1}{\sqrt2} &  \frac{1}{\sqrt{6}} &   \alpha_3 & \cdots & \alpha_{N-1} \\
        \frac{1}{\sqrt N} & 0                 & -\frac{2}{\sqrt{6}} &   \alpha_3 & \cdots & \alpha_{N-1} \\
        \frac{1}{\sqrt N} & 0                 &  0                  & -3\alpha_3 & \cdots & \alpha_{N-1} \\
        \frac{1}{\sqrt N} & 0                 &  0                  & 0          & \cdots & \alpha_{N-1} \\
        \vdots    & \vdots            &  \vdots             & \vdots     & \ddots & \vdots  \\
        \frac{1}{\sqrt N} & 0                 &  0                  & 0          & \cdots & \alpha_{N-1} \\
        \frac{1}{\sqrt N} & 0                 &  0                  & 0          & \cdots & -(N-1)\alpha_{N-1}
    \end{pmatrix},\]
    \[
        \alpha_1 = \frac{1}{\sqrt{2}}, \]
    \[
        \alpha_k = \frac{1}{k\sqrt{k(k-1)^2 \alpha_{k-1}^4 + 1}}, \quad k=\overline{2,N-1}.\]
\end{proof}

Таким образом, исходная матрица \( \mathcal{A}_{MN} \) подобна матрице
\( \mathcal{A} - \mathcal{B} \), где \( \mathcal{B} = \mathcal{U} \mathscr{B}_{MN} \mathcal{U}^{-1} \).
