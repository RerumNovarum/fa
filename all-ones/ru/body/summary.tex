Let \( \mathscr{A}_{MN} \) be a \( N\times N \) matrix composed of
\( N^2 - M \) unities and \( M \) zeroes.
Considered as an adjacencies matrix \( \mathscr{A}_{MN} \)
corresponds to a complete digraph with loops on \( N \) vertices
with some \( M \) out of \( N^2 \) edges removed.
Some important properties of a graph are determined by its spectrum.
For example Wang et al. \cite{epidemic} proposed a discrete-time model
of viral propagation in a network.
In that model the virus will die out or linger
depending on whether the ratio of curing and infection rates
is below or above the treshold value.
As Wang et al. have shown that treshold value
is the spectral radius of the network graph.
More comprehensive description of spectral graph theory
and its application is given by Cvetkovic et al. \cite{cvet}.

This article analyzes spectral properties of such matrices.
The matrix \( \mathscr{A}_{MN} \) can be represented in the form
\( \mathscr{A}_{MN} = \mathcal{J}_N - \mathscr{B}_{MN} \),
where \( \mathcal{J}_N \) is a \( N\times N \) matrix
whose all entries are ones
and \( \mathcal{B}_{MN} \) has unities exactly at these \( M \)
places where \( \mathscr{A}_{MN} \) has zeroes.
The spectrum of \( \mathcal{J}_N \) can be easily computed:
\( \mathcal{J}_N^2 = N \mathcal{J} \),
so \( \lambda(\lambda - N) \) is the minimal polynomial of \( \mathcal{J}_N \)
and hence the spectrum of \( \mathcal{J}_{N} \) is
\( \sigma(\mathcal{J}_N) = \{ 0,N \} \).

For small enough \( M \) the eigenvalues of \( \mathscr{A}_{MN} \)
will be ``close'' to those of \( \mathcal{J}_N \).
Using the Method of Similar Operators \cite{baskakov-harmonic,baskakov-split}
the following theorem is proved:
%\textbf{Theorem.}
\begin{center}
\it
    Let \( M < \frac{N^2}{16} \),
    then the spectrum of \( \mathscr{A}_{MN} \) can be represented as a disjoint union
    \( \sigma\left(\mathscr{A}_{MN}\right) = \sigma_1 \cup \sigma_2 \)
    of a singletone \( \sigma_1=\{\lambda_1\} \)
    and the set \( \sigma_2 \), satisfying the following conditions:
    \[ \sigma_1 \subset \left\{ \mu\in\mathbb{R}; \lvert \mu - N \rvert < 4\sqrt{M} \right\}, \]
    \[ \sigma_2 \subset \left\{ \mu\in\mathbb{C}; \lvert \mu \rvert < 4\sqrt{M} \right\}. \]
\end{center}
