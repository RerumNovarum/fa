В статье приведены рекурсивные формулы, позволяющие численно найти матрицу перехода \( U \).
Это даёт возможность для конкретной матрицы указать более точную оценку \( \beta \) и таким образом уточнить результаты.

Кроме того, можно заметить, что для решения уравнения \( X = \Phi(X) \)
нам достаточно решить уравнения \( \Psi_{12}(X_{12}) = X_{12} \) и \( \Psi_{21}(X_{21}) = X_{21} \),
где
\[
    \Phi X \sim
    \begin{pmatrix}
        \Phi_{11}(X) & \Phi_{12}(X) \\
        \Phi_{21}(X) & \Phi_{22}(X)
    \end{pmatrix}
\]
\begin{subequations}
    \begin{align}
        \label{eq:split11}
        & \Phi_{11}(X) = \Psi_{11}(X_{21}) = -\frac1N B_{12}X_{21} + B_{11} \\
        \label{eq:split12}
        & \Phi_{12}(X) = \Psi_{12}(X_{12}) = \frac{1}{N^2} X_{12}B_{21}X_{12} -
                                   \frac1N\left(X_{12}B_{22} +
                                   B_{11}X_{12}\right) + B_{12} \\
        \label{eq:split21}
        & \Phi_{21}(X) = \Psi_{21}(X_{21}) = -\frac{1}{N^2} X_{21}B_{12}X_{21} -
                                   \frac1N\left(X_{21}B_{11} +
                                   B_{22}X_{21}\right) +
                                   B_{21} \\
        \label{eq:split22}
        & \Phi_{22}(X) = \Psi_{22}(X_{12}) = \frac1N B_{21}X_{12} + B_{22}
    \end{align}
\end{subequations}

Найдя решения \( X_{12}^o, X_{21}^o \),
мы бы автоматически получили
\( \Phi(X^o) = X^o \) для
\( X^o \sim
\left(\begin{smallmatrix}
    X_{11}^o & X_{12}^o \\
    X_{21}^o & X_{22}^o
\end{smallmatrix}\right) \),
где \( X_{11}^o, X_{22}^o \) из \eqref{eq:split11} \eqref{eq:split22}.

Решения \eqref{eq:split12} \eqref{eq:split21} можно найти аналогично предыдущему параграфу
с такой же грубой оценкой нормы: \( \norm{X_{12}^o}, \norm{X_{21}^o} \leq 4\beta \).
При этом
\[ X_{11}^o = \frac1N B_{12} X_{21}^o + B_{11} \]
\[ \norm{X_{11}^o} \leq \frac4N \beta_{12}\beta + \beta_{11} \]
где \( \beta_{ij} = \norm{B_{ij}} \).

Таким образом для (возможного) уточнения оценки \( \lambda_1 \)
нужно отследить (при применении преобразования подобия \( T \)) (например, с помощью приведённых выше формул)
число \( B_{11} \), строку \( B_{12} \) и столбец \( B_{22} \).
