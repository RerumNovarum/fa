Рассмотрим матрицу \( \mathscr{A} \) размера \( N\times N \),
 составленную из \( M \) нулей и \( N^2 - M \) единиц.
Как матрица смежности, \( \mathscr{A} \) соответствует орграфу,
 полученному из полного графа с~петлями на \( N \) вершинах
 удалением некоторых \( M \) из \( N^2 \) р\"ебер.
Некоторые важные свойства графа связаны с~спектром его матрицы смежностей.
Так, например, в~\cite{wang2003epidemic,chakrabarti2008epidemic}
 распространение вируса в~сети смоделировано марковским процессом,
 причём спектральный радиус матрицы смежностей графа сети
 определяет пороговое значение
 \( \tau_0=\frac{1}{\operatorname{spr}(\mathscr{A})} \)
 отношения \( \tau = {^\nu/_\delta} \)
 интенсивности~\( \nu \) заражения узлов, смежных инфецированным
 и~интенсивности~\( \delta \) исцеления инфецированных узлов.
 Положение \(\tau\) относительно порога \({\tau_0}\) определяет
 (эндемический или эпидемический) характер заражения.
Доминирующий собственный вектор матрицы смежностей
имеет большое значение в задаче
ранжирования связанных ссылками документов
по их цитируемости~\cite{bonacich1972factoring,ilprints422}.
Подробно спектральная теория графов и~е\"е приложения
изложены в~монографии~\cite{cvetkovic1980spectra}.

Что можно сказать о~собственных значениях матриц рассматриваемого вида?

Матрицу \( \mathscr{A} \) можно представить в~виде
\[ \mathscr{A} = \mathcal{J}_N - \mathscr{B} =
    \begin{pmatrix}1 & \cdots & 1\\ \vdots & \ddots & \vdots \\ 1 & \cdots & 1\end{pmatrix} - \mathscr{B}, \]
 где \( \mathcal{J}_N \)~--- матрица, составленная из \( N\times N \) единиц,
 а~\( \mathscr{B} \) имеет единицы в~точности на тех \( M \) местах,
 где в~\( \mathscr{A} \) стоят нули.

Спектр \( \sigma\left( \mathcal{J}_N \right) \)
 матрицы \( \mathcal{J}_N \) легко считается:
 \( \mathcal{J}_N^2 = N \mathcal{J}_N, \) т.е.
 \( \lambda(\lambda - N) \)~--- аннулирующий и, что легко проверить,
 минимальный многочлен матрицы \( \mathcal{J}_N \), а~значит
 \( \sigma\left( \mathcal{J}_N \right) = \left\{ 0,N \right\}. \)

При достаточно малых \( M \),
 спектры матриц \( \mathcal{J}_N \) и~\( \mathscr{A} \) будут ``близки''.
Методом подобных операторов (см.~\cite{baskakov-harmonic,baskakov1983}),
 позволяющим для возмущений ``идеального'' объекта, спектральные свойства которого известны,
 найти элемент рассматриваемой алгебры, изоспектральный возмущ\"енному,
 но имеющий более удобную для вычислений структуру,
 в~статье доказывается
\begin{thm}\label{kozlukovsv:thm:almost-all-ones}
    Пусть \( M < \frac{N^2}{16} \),
    тогда спектр матрицы \( \mathscr{A} \) можно представить в~виде
    объединения \( \sigma\left(\mathscr{A}\right) = \sigma_1 \cup \sigma_2 \)
    непересекающихся
    одноэлементного множества \( \sigma_1=\{\lambda_1\} \)
    и~множества \( \sigma_2 \), удовлетворяющих условиям:
    \[ \sigma_1 \subset \left\{ \mu\in\mathbb{R}; \lvert \mu - N \rvert < 4\sqrt{M} \right\}, \]
    \[ \sigma_2 \subset \left\{ \mu\in\mathbb{C}; \lvert \mu \rvert < 4\sqrt{M} \right\}. \]
    Собственное значение \( \lambda_1 \) совпадает с спектральным радиусом
    \( \operatorname{spr}(\mathscr{A}) \),
    и ему соответствует собственный вектор \( v_1 \)
    удовлетворяющий условию
    \[ \left\|v_1 - \begin{pmatrix} N\\ \vdots \\ N\end{pmatrix}\right\|_2 < 4\frac{\sqrt{M}}{N}. \]
\end{thm}
