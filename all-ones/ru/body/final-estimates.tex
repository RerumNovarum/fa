Итак, если \( \gamma\beta \leq \frac14 \),
то существует единственное решение \( X^o \) уравнения
\( (A-B)(I+\Gamma X) = (I+\Gamma X)(A-JX) \),
или, другими словами, существует такая матрица
\( X^o =
\begin{pmatrix}
    x_{11} & X_{12} \\
    X_{21} & X_{22}
\end{pmatrix}
\),
удовлетворяющая \( \norm{X^o} \leq r_0 \leq 4\norm{B} \),
что матрица \( A-B \) подобна блочно-диагональной матрице \( A - JX^o \):
\[
    A-B \sim
    \begin{pmatrix}
        N - x_{11} & 0 \\
        0          & -X_{22}^o
    \end{pmatrix},
    \]
При этом спектр матрицы \( A - JX^o \) есть объединение спектров
её диагональных блоков:
\[ \spec{A-B} = \left\{N - x_{11}\right\} \cup \spec{-X_{22}}. \]

\begin{lemma}
    Пусть \( \norm{\cdot} \) --- норма в \( \matr{N}{} \),
    согласованная с какой-нибудь нормой в пространстве \( \mathscr{X} \) векторов,
    \( \beta = \norm{B}, \gamma=\sup_{\norm{X}=1} \norm{\Gamma X} \)
    и пусть верно строгое неравенство \( \gamma\beta < \frac14 \).

    Тогда спектр \( A - B \)
    есть объединение неперсекающихся одноточечного вещественного множества \( \sigma_1 \)
    и множества \( \sigma_2 \),
    лежащих в шарах радиуса \( 4\beta \)
    с центрами в \( N \) и \( 0 \) соответственно:

    \[ \spec{A-B} = \sigma_1 + \sigma_2 \]
    \[ \sigma_1 = \left\{\lambda_1\right\} \subset \left\{\lambda\in\RR; \lvert \lambda - N\rvert < 4\beta\right\} \]
    \[ \sigma_2 \subset \left\{\lambda\in\CC; \lvert\lambda\rvert <4\beta \right\} \]
\end{lemma}
\begin{proof}
    Согласованная норма субмультипликативна, поэтому применима предыдущая лемма.
    Согласованность также означает, что для любой матрицы \( Z \)
    и любого её нормированного собственного вектора \( h_j \),
    отвечающего собственному значению \( \mu_j \),
    имеет место \({ \normex{v}{A h_j} \leq \norm{A} }\),
    т.е. \({ \lvert \mu_j \rvert \leq \norm{A} }\).

    Значит \( \lvert\lambda_1\rvert = \lvert N - x_{11} \rvert \leq r_0 < 4\beta \)
    и \( \spr{-X_{22}} \leq r_0 < 4\beta  \)
\end{proof}


Доказательство теоремы \ref{thm:almostallones-spectra}
состоит в выборе подходящих норм.
\begin{proof}
    Напомню, \( B = U^{-1} \perturbmatrix{M}{N} U \)
    получена из матрицы возмущения с \( M \) единицами применением преобразования \eqref{eq:diagtransform},
    диагонализирующего матрицу единиц,
    а \( \Gamma \) --- оператор из \( \End{\matr{N}{}} \),
    определённый формулой
    \( \Gamma X = \frac1N \begin{pmatrix}0 & X_{12} \\ -X_{21} & 0\end{pmatrix} \).

        Рассмотрим в \( \mathscr{X} \) обычную \(p\)-норму:
    \( \normex{v}{x} = \left(\sum_j \lvert x_j \rvert^p \right)^{1/p} \),
    и аналогично положим в пространстве \( \matr{N}{} \)
    \( \normex{m}{Z} = \left( \sum_j {\normex{v}{Z e_j}}^p \right)^{1/p} \),
    где \( \{ e_j \} \) --- канонический базис в \( \mathscr{X} \).
    Не трудно убедиться, что \( \normex{m}{\cdot} \) согласована с \( \normex{v}{\cdot} \).

    Тогда оказывается:
    \[ \gamma = \frac1N, \]
    \[  \beta = M^{1/p} ,\]
    где \( M \) --- число единиц в матрице \( \perturbmatrix{M}{N} \) возмущения
    (число нулей в исходной матрице \( \almostallones{M}{N} \)).

    Подробнее:
    \[ \gamma = \frac1N \sup_{\normex{m}{X}=1}\normex{m}{\begin{pmatrix}0 & X_{12} \\ -X_{21} & 0\end{pmatrix}} = \frac1N \]

    Остаётся заметить, что умножение на унитарную матрицу \( U \)
    (и \( U^{-1} \)) есть изометрия
    в \( \matr{N}{} \)
    и поэтому
    \[ \beta = \normex{m}{B} = \normex{m}{\perturbmatrix{M}{N}} = \left(\sum_j{\normex{v}{\perturbmatrix{M}{N}e_j}}^p\right)^{1/p}.\]
    Ясно, что \( {\normex{v}{\perturbmatrix{M}{N}e_j}}^p \) --- это суть число единиц в \( j \)-ом столбце
    матрицы \( \perturbmatrix{M}{N} \).
    Тогда \( \beta \) это суть корень степени \( p \) из числа \( M \) единиц в \( \perturbmatrix{M}{N} \).

    Значит, если
    \( M^{1/p} < \frac{N}{4} \), т.е. \( M < \left(N/4\right)^p \),
    то выполняются условия леммы,
    причём \( r_0 < 4M^{1/p} \).
\end{proof}
