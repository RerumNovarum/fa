Вернёмся, наконец, к непосредственному доказательству основной теоремы:
\begin{proof}{\bf Теоремы 1}
    Для доказательства осталось лишь выбрать подходящую субмультипликативную норму.
    Заметим, что матрица \( \mathcal{U} \),
    приводящая \( \mathcal{J}_N \) к диагональному виду
    является ортогональной,
    поэтому умножение на \( \mathcal{U} \) или \(\mathcal{U}^{-1}\)
    является изометрией.
    Короче, \( \|\mathcal{B}\|=\|\mathscr{B}_{MN}\| \).
    Рассмотрим в пространстве \( \mathtt{Matr}_{N}\mathbb{C} \)
    норму Фробениуса \( {\left\|\cdot\right\|}_{F} \),
    определённую формулой
    \( {\left\|X\right\|}_{F} = \sqrt{\sum_{ij} \lvert x_{ij}\rvert^2}. \)
    Она субмультипликативна.
    При этом
    \( \mathscr{B}_{MN} \) состоит из \( M \) единиц, поэтому
    \[
        \beta = {\left\|B\right\|}_{F} =
        {\left\|\mathscr{B}_{MN}\right\|}_{F} = \sqrt{M}.
        \]
    Заметим также очевидное равенство
    \[
        \gamma = \frac1N
                \sup_{{\left\|X\right\|}_{F}=1}{\left\|\begin{pmatrix}0 & X_{12} \\ -X_{21} & 0\end{pmatrix}\right\|}_{F}
                = \frac1N. \]
    
    Если
     \( \sqrt{M} < \frac{N}{4} \),
     то выполняются условия леммы,
     причём \( r_0 < 4\sqrt{M} \).
    Это значит, что
     \( \sigma(\mathscr{A}_{MN}) = \sigma_1 \cup \sigma_2 \),
     где \( \sigma_1 = \{ \lambda_1 \}\subset\mathbb{R}, \lvert \lambda_1 - N \rvert < 4\sqrt{M} \),
     \( \sigma_2 \subset \{ \mu\in\mathbb{C}; \lvert\mu\rvert < 4\sqrt{M} \} \),
     \( \sigma_1 \cap \sigma_2 = \emptyset \).
    Но в этом и состоит утверждение теоремы.
    \end{proof}
