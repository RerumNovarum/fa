Доказательство основной теоремы (стр. \pageref{thm:almostallones-spectra})
состоит в выборе подходящей субмультипликативной нормы.
Матрица \( B = U^{-1} \perturbmatrix{M}{N} U \)
получена из матрицы возмущения с \( M \) единицами
ортогональным преобразованием подобия \eqref{eq:diagtransform},
диагонализирующим матрицу единиц.
\( \Gamma \) --- оператор, действующий в \( \matr{N}{} \)
по формуле
\( { \Gamma X = \frac1N \begin{pmatrix}0 & X_{12} \\ -X_{21} & 0\end{pmatrix} } \).

Рассмотрим в пространстве \( \matr{N}{} \)
норму Фробениуса \( \normex{F}{\cdot} \),
определённую формулой
\( \normex{F}{X} = \sqrt{\sum_{ij} \lvert x_{ij}\rvert^2}. \)
Эта норма субмультипликативна.

Тогда:
\[ \gamma = \frac1N
            \sup_{\normex{F}{X}=1}\normex{F}{\begin{pmatrix}0 & X_{12} \\ -X_{21} & 0\end{pmatrix}}
          = \frac1N,
    \]
и, так как умножение на унитарную матрицу \( U \)
    (и \( U^{-1} \)) есть изометрия в \( \matr{N}{} \),
    а \( \perturbmatrix{M}{N} \) состоит из \( M \) единиц:
\[
    \beta = \normex{F}{B} =
    \normex{F}{\perturbmatrix{M}{N}} = \sqrt{M},
    \]

Значит, если
\( \sqrt{M} < \frac{N}{4} \), т.е.
\( M < \frac{N^2}{16} \),
то выполняются условия леммы,
причём \( r_0 < 4\sqrt{M} \).
