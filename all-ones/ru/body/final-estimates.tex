Осталось выбрать подходящую субмультипликативную норму.
Заметим, что матрица \( \mathcal{U} \),
приводящая \( \mathcal{J}_N \) к диагональному виду
является ортогональной,
поэтому умножение на \( \mathcal{U} \) или \(\mathcal{U}^{-1}\)
является изометрией.
Короче, \( \|\mathcal{B}\|=\|\mathscr{B}_{MN}\| \).
Рассмотрим в пространстве \( \mathtt{Matr}_{N}\mathbb{C} \))
норму Фробениуса \( {\left\|\cdot\right\|}_{F} \),
определённую формулой
\( {\left\|X\right\|}_{F} = \sqrt{\sum_{ij} \lvert x_{ij}\rvert^2}. \)
Она субмультипликативна.
При этом
\( \mathscr{B}_{MN} \) состоит из \( M \) единиц, поэтому
\[
    \beta = {\left\|B\right\|}_{F} =
    {\left\|\mathscr{B}_{MN}\right\|}_{F} = \sqrt{M}.
    \]
Также имеет место
\[ \gamma = \frac1N
            \sup_{{\left\|X\right\|}_{F}=1}{\left\|\begin{pmatrix}0 & X_{12} \\ -X_{21} & 0\end{pmatrix}\right\|}_{F}
          = \frac1N.
    \]
Значит, если
\( \sqrt{M} < \frac{N}{4} \),
то выполняются условия леммы,
причём \( r_0 < 4\sqrt{M} \).
