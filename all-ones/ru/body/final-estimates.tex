Доказательство основной теоремы (стр. \pageref{thm:almostallones-spectra})
состоит в выборе подходящих норм.
Напомню, \( B = U^{-1} \perturbmatrix{M}{N} U \)
получена из матрицы возмущения с \( M \) единицами применением
ортогонального преобразования \eqref{eq:diagtransform},
диагонализирующего матрицу единиц,
а \( \Gamma \) --- оператор, действующий в \( \matr{N}{} \)
по формуле
\( { \Gamma X = \frac1N \begin{pmatrix}0 & X_{12} \\ -X_{21} & 0\end{pmatrix} } \).

Рассмотрим в пространстве \( \matr{N}{} \)
норму Гильберта-Шмидта \( \normex{HS}{\cdot} \),
определённую формулой
\[ \normex{HS}{X} = \sqrt{\sum_{ij} \lvert x_{ij}\rvert^2}. \]

Тогда оказывается:
\[ \gamma = \frac1N
            \sup_{\normex{HS}{X}=1}\normex{HS}{\begin{pmatrix}0 & X_{12} \\ -X_{21} & 0\end{pmatrix}}
          = \frac1N, \text{ и} \]
\[
    \beta = \normex{HS}{B} =
    \normex{HS}{\perturbmatrix{M}{N}} = \sqrt{M},
    \]
так как умножение на унитарную матрицу \( U \)
(и \( U^{-1} \)) есть изометрия в \( \matr{N}{} \),
а \( \perturbmatrix{M}{N} \) состоит из \( M \) единиц.

Значит, если
\( \sqrt{M} < \frac{N}{4} \), т.е.
\( M < \frac{N^2}{16} \),
то выполняются условия леммы,
причём \( r_0 < 4\sqrt{M} \).
