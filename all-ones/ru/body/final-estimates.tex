Доказательство основной теоремы (стр. \pageref{thm:almostallones-spectra})
состоит в выборе подходящей субмультипликативной нормы.
Матрица \( B = U^{-1} \mathscr{B}_{M,N} U \)
получена из матрицы возмущения с \( M \) единицами
ортогональным преобразованием подобия \eqref{eq:diagtransform},
диагонализирующим матрицу единиц.
\( \Gamma \) --- оператор, действующий в \( \mathrm{Matr}_{N} \)
по формуле
\( { \Gamma X = \frac1N \begin{pmatrix}0 & X_{12} \\ -X_{21} & 0\end{pmatrix} } \).

Рассмотрим в пространстве \( \mathrm{Matr}_{N} \)
норму Фробениуса \( {\left\|\cdot\right\|}_{F} \),
определённую формулой
\( {\left\|X\right\|}_{F} = \sqrt{\sum_{ij} \lvert x_{ij}\rvert^2}. \)
Эта норма субмультипликативна.

Тогда:
\[ \gamma = \frac1N
            \sup_{{\left\|X\right\|}_{F}=1}{\left\|\begin{pmatrix}0 & X_{12} \\ -X_{21} & 0\end{pmatrix}\right\|}_{F}
          = \frac1N,
    \]
и, так как умножение на унитарную матрицу \( U \)
    (и \( U^{-1} \)) есть изометрия в \( \mathrm{Matr}_{N} \),
    а \( \mathscr{B}_{M,N} \) состоит из \( M \) единиц:
\[
    \beta = {\left\|B\right\|}_{F} =
    {\left\|\mathscr{B}_{M,N}\right\|}_{F} = \sqrt{M},
    \]

Значит, если
\( \sqrt{M} < \frac{N}{4} \), т.е.
\( M < \frac{N^2}{16} \),
то выполняются условия леммы,
причём \( r_0 < 4\sqrt{M} \).
