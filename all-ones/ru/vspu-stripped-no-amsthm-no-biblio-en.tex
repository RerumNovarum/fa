\documentclass[11pt]{article}

% \usepackage{fontspec}
% \setmainfont{CMU Serif}
% \usepackage{polyglossia}
% \setotherlanguage{english}

\usepackage[utf8]{inputenc}
\usepackage{lmodern}

%\usepackage[a5paper]{geometry}
\usepackage[a5paper,left=20mm, right=20mm, top=20mm, bottom=20mm, includefoot]{geometry}
\usepackage[intlimits]{amsmath}
\interdisplaylinepenalty=2500
\usepackage{amsfonts}
\usepackage{amssymb}
\usepackage{latexsym}

\usepackage{mathrsfs}
\usepackage{mathtools}

\begin{document}
\addcontentsline{toc}{section}{{\it{Козлуков С.В.}} Spectra of an almost-all-ones matrix}
\small{UDC 517.984.3 : 519.177}

\begin{center}
\textbf{Spectra of almost-all-ones matrix}\\
\small{S.V. Kozlukov} \\
\small{Voronezh State University} \\
\small{rerumnovarum@openmailbox.org} \\
\end{center}

Let \( \mathscr{A}_{MN} \) be a matrix of degree \( N \),
consisting of \( N^2 - M \) unities and \( M \) zeroes
and let the number of zeroes be small in some sense.
Considered as an adjacency matrix,
\( \mathscr{A}_{MN} \) corresponds to a complete oriented graph
with \( N \) vertices
with some edges ``removed''.
Estimates for eigenvalues of such a matrix are derived in this article.

Matrix \( \mathscr{A}_{MN} \) can be represented in the form
\[
    \mathscr{A}_{MN} = \mathscr{J}_{N} - \mathscr{B}_{MN},
    \]
where \(
\mathscr{J}_{N} =
\begin{pmatrix}1 & \cdots & 1 \\
\vdots & \ddots & \vdots \\
1 & \cdots & 1
\end{pmatrix} \) --- is all-ones matrix
and matrix \( \mathscr{B}_{MN} \) has unities
exactly at these \( M \) places,
where \( \mathscr{A}_{MN} \) has zeroes.

Spectra \( \mathscr{J}_{N} \) can be easily computed:
\( \mathrm{spec}\left({\mathscr{J}_{N}}\right) = \left\{0, N \right\} \).
If the number \( M \) of edges removed
is significantly smaller than total amount \( N^2 \) of edges in a complete orgraph
we can expect the eigenvalues of
\( \mathscr{A}_{MN} \) to be ``close'' to those of \( \mathscr{J}_{N} \).

Using a method of similar operators (see Baskakov AG [1,2])
the main result is achieved:

\textbf{Theorem.}

{\it
Let
\(M~<~\displaystyle{N^2/16}. \)

Then spectra \( \mathrm{spec}\left({\mathscr{A}_{MN}}\right) \)
of a matrix \( \mathscr{A}_{MN} \)
can be represented as a union of
a single-point set \( \sigma_1 \subset \mathbb{R} \)
and a set \( \sigma_2 \subset \mathbb{C} \),
contained in balls of radia \( 4\sqrt{M} \)
centered at \( N \) and \( 0 \) respectively
\begin{equation*}\begin{aligned}
    & \mathrm{spec}\left({A-B}\right) = \sigma_1 \cup \sigma_2, \\
    & \sigma_1 = \left\{ \lambda_1 \right\}
      \subset \left\{ \lambda\in\mathbb{R}; \lvert \lambda - N\rvert < 4\beta \right\}, \\
    & \sigma_2 \subset \left\{\lambda\in\mathbb{C}; \lvert\lambda\rvert <4\beta \right\}. \\
\end{aligned}\end{equation*}
}

Complete (almost-complete) graphs model (almost-)fully-connected networks,
which are though rare and of rather academic interest,
yet might occur in some military applications.

In applications it is also more common practice to consider
the spectra of Laplacian matrix \( L = D - A \),
where \( D \) is a (diagonal) degrees matrix
and \( A \) is adjacency matrix.

The practical value of spectra of adjacencies matrix
can be seen in the following model of computer virus spread [3]:
let \( G = (V, E) \) be a connected network (connected orgraph)
with set of nodes (vertices)\( V = \{1, \ldots, N\} \)
and set of links (edges)\( E \).
Time is considered discrete.
During each time-interval every ``infected'' node
attempts infect its neighbours (adjacent nodes)
with a uniform for the whole network success probability \( \beta \) (called ``virus birth rate'').
At the same time every infected node might be ``cured'' with probability \( \delta \) (``curing rate'').
It is known that there exist a treshold value \( \tau \) of ratio \( {^\beta}/_{\delta} \),
above which an infection becomes epidemic,
and below which epidemic dies out (infection probability decreases exponentially in time).
Yang Wang and D. Chakrabarti and Chenxi Wang and C. Faloutsos have shown in [3]
that \( \tau = 1/\mathrm{spr}\left({A}\right) \),
where \( \mathrm{spr}\left({A}\right) = \max\left\{\lvert \lambda\rvert; \lambda\in\mathrm{spec}\left(A\right) \right\} \)
denotes a spectral radius of the adjacency matrix \( A \) of the graph \( G \).

% A nice bibtex' include was here,
% which used to generate a proper bibliography
% enforcing a single and consistent style;
% In these times of darkness we're not allowed
% to use macros or appropriate package
% or any other kind of generic solution
% --- you hardcode, so that you can't maintain
\centerline{\textbf{Bibliography}}

1. Baskakov A. G. Harmonic analysis of linear operators //Voronezh Univ., Voronezh. – 1987. -- pp.~93--121.

2. Baskakov A. G. A theorem on splitting an operator, and some related questions in the analytic theory of perturbations
   //Mathematics of the USSR-Izvestiya. – 1987. – Vol. 28. – No 3. – pp.~421.

3. Wang Y. et al. Epidemic spreading in real networks: An eigenvalue viewpoint
   //Reliable Distributed Systems, 2003. Proceedings. 22nd International Symposium on. – IEEE, 2003. – pp.~25-34.
\end{document}
