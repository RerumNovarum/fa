\documentclass[12pt,a4paper,twoside]{article}

\usepackage{amsmath}
\usepackage{mathtools}
\usepackage{mathrsfs}
\usepackage{unicode-math}
\usepackage{marginnote}
\usepackage{hyperref}
\usepackage{tabularx}

\usepackage{fontspec}
% \setmainfont{CMU Serif}

\usepackage{amsthm}
\newtheorem{thm}{Theorem}
\newtheorem{lem}[thm]{Lemma}
\newtheorem{crl}[thm]{Corollary}
\theoremstyle{definition}
\newtheorem{dfn}{Definition}

\begin{document}

\noindent {UDK 517.984.3 : 519.177}

\begin{center}
    \textbf{ON SPECTRAL PROPERTIES OF AN ADJACENCY MATRIX OF AN ALMOST-COMPLETE GRAPH}
    
    \textbf{Sergey Kozlukov}\\[2mm]
    \emph{Voronezh State University}
\end{center}

% \maketitle

\begin{abstract}
    Let \( \mathscr{A}_{MN} \) be a \( N\times N \) matrix composed of
\( N^2 - M \) unities and \( M \) zeroes.
Considered as an adjacencies matrix \( \mathscr{A}_{MN} \)
corresponds to a complete digraph with loops on \( N \) vertices
with some \( M \) out of \( N^2 \) edges removed.
Some important properties of a graph are determined by its spectrum.
For example Wang et al. \cite{epidemic} proposed a discrete-time model
of viral propagation in a network.
In that model the virus will die out or linger
depending on whether the ratio of curing and infection rates
is below or above the treshold value.
As Wang et al. have shown that treshold value
is the spectral radius of the network graph.
More comprehensive description of spectral graph theory
and its application is given by Cvetkovic et al. \cite{cvet}.

This article analyzes spectral properties of such matrices.
The matrix \( \mathscr{A}_{MN} \) can be represented in the form
\( \mathscr{A}_{MN} = \mathcal{J}_N - \mathscr{B}_{MN} \),
where \( \mathcal{J}_N \) is a \( N\times N \) matrix
whose all entries are ones
and \( \mathcal{B}_{MN} \) has unities exactly at these \( M \)
places where \( \mathscr{A}_{MN} \) has zeroes.
The spectrum of \( \mathcal{J}_N \) can be easily computed:
\( \mathcal{J}_N^2 = N \mathcal{J} \),
so \( \lambda(\lambda - N) \) is the minimal polynomial of \( \mathcal{J}_N \)
and hence the spectrum of \( \mathcal{J}_{N} \) is
\( \sigma(\mathcal{J}_N) = \{ 0,N \} \).

For small enough \( M \) the eigenvalues of \( \mathscr{A}_{MN} \)
will be ``close'' to those of \( \mathcal{J}_N \).
Using the Method of Similar Operators \cite{baskakov-harmonic,baskakov-split}
the following theorem is proved:
%\textbf{Theorem.}
\begin{center}
\it
    Let \( M < \frac{N^2}{16} \),
    then the spectrum of \( \mathscr{A}_{MN} \) can be represented as a disjoint union
    \( \sigma\left(\mathscr{A}_{MN}\right) = \sigma_1 \cup \sigma_2 \)
    of a singletone \( \sigma_1=\{\lambda_1\} \)
    and the set \( \sigma_2 \), satisfying the following conditions:
    \[ \sigma_1 \subset \left\{ \mu\in\mathbb{R}; \lvert \mu - N \rvert < 4\sqrt{M} \right\}, \]
    \[ \sigma_2 \subset \left\{ \mu\in\mathbb{C}; \lvert \mu \rvert < 4\sqrt{M} \right\}. \]
\end{center}

\end{abstract}

%\keywords{Метод Подобных Операторов,
%          собственные значения,
%          спектр графа}{
%          The Similar Operators Method,
%          eigenvalues,
%          graphs spectra}

\section*{Preliminary transformations}
Доказательство состоит в построении уравнения для матрицы, подобной \( \mathscr{A}_{MN} \),
 но устроеной ``проще''. Решение возникающего нелинейного уравнения
 в банаховой алгебре \( \mathtt{Matr}_N\mathbb{C} \)
 доставляется методом простых итераций (см. \cite{baskakov-harmonic}).

Подобие матриц \( \mathcal{A}_1, \mathcal{A}_2 \)
 понимается в смысле существования обратимой матрицы \( \mathcal{U} \),
 такой что \( \mathcal{A}_1 \mathcal{U} = \mathcal{U} \mathcal{A}_2 \).
Спектры подобные матриц совпадают.

Провед\"ем предварительные преобразования.

\begin{lem}
    Матрица единиц 
    \( \mathcal{J}_N =
    \begin{pmatrix}
        1 & \cdots & 1 \\
        \vdots & \ddots & \vdots \\ 
    1 & \cdots & 1 \end{pmatrix} \),
    подобна матрице
    \[
        \mathcal{A} = \begin{pmatrix}
            N & 0 & \cdots & 0 \\
            0 & 0 & \cdots & 0 \\
            \vdots & \vdots & \ddots & \vdots \\
            0 & 0 & \cdots & 0 \end{pmatrix}. \]
    Точнее, существует ортогональная матрица \( \mathcal{U} \),
    такая что
    \( \mathcal{A} = \mathcal{U}\mathcal{J}_N \mathcal{U}^{-1} \).
\end{lem}
\begin{proof}
    Собственному значению \( N \) матрицы \( \mathcal{J}_N \) 
    соответствует собственный вектор \( f_1 = {\left(1,\ldots,1\right)} \),
        а собственному значению \( 0 \) соответствует \( N-1 \) независимый собственный вектор
        \( f_2 = {\left(1,-1,0,\ldots,0\right)}, ...,
           f_N = {\left(0,\ldots,0,1,-1\right)} \).
    Применив ортогонализацию Грамма-Шмидта, получим ортонормальную систему \( h_1, \ldots, h_N \).
    В качестве матрицы \( \mathcal{U} \) выберем матрицу,
    имеющую столбцами векторы \( h_1, \ldots, h_N \):
    \[ \mathcal{U} =
    \begin{pmatrix}
        \frac{1}{\sqrt N} &  \frac{1}{\sqrt2} &  \frac{1}{\sqrt{6}} &   \alpha_3 & \cdots & \alpha_{N-1} \\
        \frac{1}{\sqrt N} & -\frac{1}{\sqrt2} &  \frac{1}{\sqrt{6}} &   \alpha_3 & \cdots & \alpha_{N-1} \\
        \frac{1}{\sqrt N} & 0                 & -\frac{2}{\sqrt{6}} &   \alpha_3 & \cdots & \alpha_{N-1} \\
        \frac{1}{\sqrt N} & 0                 &  0                  & -3\alpha_3 & \cdots & \alpha_{N-1} \\
        \frac{1}{\sqrt N} & 0                 &  0                  & 0          & \cdots & \alpha_{N-1} \\
        \vdots    & \vdots            &  \vdots             & \vdots     & \ddots & \vdots  \\
        \frac{1}{\sqrt N} & 0                 &  0                  & 0          & \cdots & \alpha_{N-1} \\
        \frac{1}{\sqrt N} & 0                 &  0                  & 0          & \cdots & -(N-1)\alpha_{N-1}
    \end{pmatrix},\]
    \[
        \alpha_1 = \frac{1}{\sqrt{2}}, \]
    \[
        \alpha_k = \frac{1}{k\sqrt{k(k-1)^2 \alpha_{k-1}^4 + 1}}, \quad k=\overline{2,N-1}.\]
\end{proof}

Таким образом, исходная матрица \( \mathcal{A}_{MN} \) подобна матрице
\( \mathcal{A} - \mathcal{B} \), где \( \mathcal{B} = \mathcal{U} \mathscr{B}_{MN} \mathcal{U}^{-1} \).

% \clearpage
\subsection*{Splitting the matrix and estimating eigenvalues}
Матрицы из \( \mathbb{C}^{N{\times}N} \) будем записывать в~блочном виде
\( X \sim
    \begin{pmatrix}
    x_{11} & X_{12} \\
    X_{21} & X_{22}
    \end{pmatrix}, \)
    где \( x_{11} \)~--- число,
    \( X_{12} \)~--- строка, \( X_{21} \)~--- столбец,
    \( X_{22} \)~--- квадратный блок размерности \( N-1 \).
Такие блочные матрицы сами образуют алгебру, изоморфную исходной
и~их можно естественным образом умножать
на элементы пространства \( \mathbb{C}\times\mathbb{C}^{N-1} \),
изоморфного~\( \mathbb{C}^N \):
\[
    \begin{pmatrix}
    x_{11} & X_{12} \\
    X_{21} & X_{22}
    \end{pmatrix}
    \begin{pmatrix} x_1 \\ x_2 \end{pmatrix}
  = \begin{pmatrix}
      x_{11} x_1 + X_{12} x_2 \\
      X_{21} x_1 + X_{22} x_2
      \end{pmatrix},\quad x \sim \begin{pmatrix} x_1 \\ x_2 \end{pmatrix}\in \mathbb{C}\times\mathbb{C}^{N-1}.
    \]
В~дальнейших выкладках изоморфные объекты понимаются взаимозаменяемыми.

Следуя общей схеме метода подобных операторов~\cite{baskakov-harmonic,baskakov1983},
будем искать более ``простую'' матрицу, подобную \( \mathcal{A} - \mathcal{B} \),
в~виде \( \mathcal{A} - \mathfrak{J} X \)
с~матрицей преобразования подобия \( E + \Gamma X \),
где \( E\in{\mathbb{C}^{N{\times}N}} \)~--- единичная матрица,
\( \mathfrak{J},\Gamma{:}\ \mathbb{C}^{N{\times}N}{\to}\mathbb{C}^{N{\times}N} \)~--- линейные операторы,
действующие на алгебре \( \mathbb{C}^{N{\times}N} \), подбираемые
в~ходе решения,
      прич\"ем \( \mathfrak{J} \) --- проектор (\(\mathfrak{J}^2=\mathfrak{J}\)),
      ``упрощающий'' возмущение \( \mathfrak{J}X \),
      а \( \Gamma \)
      при всех \( X\in {\mathbb{C}^{N{\times}N}} \)
      удовлетворяет уравнению
          \( \mathcal{A}\Gamma X - (\Gamma X) \mathcal{A} = X - \mathfrak{J}X. \)

\begin{lem}
    В качестве \( \mathfrak{J} \)
        естественно выбрать оператор блочной диагонализации:
    \[
        \mathfrak{J} X = \begin{pmatrix} x_{11} & 0 \\ 0 & X_{22} \end{pmatrix}. \]
    При этом:
    \[
        \Gamma X = \frac{1}{N} \begin{pmatrix} 0 & X_{12} \\ -X_{21} & 0 \end{pmatrix}, \]
        для \( X\sim \begin{pmatrix}x_{11} & X_{12} \\ X_{21} & X_{22}\end{pmatrix} \in \mathbb{C}^{N{\times}N} \)
    и имеет место равенство
    \[
        \gamma = \frac1N
                \sup_{{\left\|X\right\|}_{F}=1}{\left\|\begin{pmatrix}0 & X_{12} \\ -X_{21} & 0\end{pmatrix}\right\|}_{F}
                = \frac1N. \]

\end{lem}
\begin{crl}
    Спектр блочно-диагональной матрицы
    \( \mathcal{A} - \mathfrak{J}X = \begin{pmatrix} N - x_{11} & 0 \\ 0 & -X_{22} \end{pmatrix} \)
    есть объединение спектров е\"е диагональных блоков:
    \[
        \sigma(\mathcal{A} - \mathfrak{J} X) = \{ N - x_{11} \} \cup \sigma(-X_{22}). \]
\end{crl}
\begin{proof}
Пусть \( \Gamma \) действует по формуле
\( \Gamma X = \begin{pmatrix} \Gamma_{11}(X) & \Gamma_{12}(X) \\
                              \Gamma_{21}(X) & \Gamma_{22}(X)
                              \end{pmatrix} \), тогда
\[
    \mathcal{A} \Gamma X - (\Gamma X)\mathcal{A} = 
    \begin{pmatrix} 0 & N\Gamma_{12}(X) \\
        - N\Gamma_{21}(X) & 0
        \end{pmatrix}, \]
и~уравнение для \( \Gamma X \) сводится~к
\[
    X - \mathfrak{J} X =
    N \begin{pmatrix} 0 & \Gamma_{12}(X) \\
        - \Gamma_{21}(X) & 0
        \end{pmatrix}. \]

Значит, \( \mathfrak{J} \) может обнулить вс\"е,
    кроме двух диагональных блоков размеров \( 1\times 1 \)
    и \( (N-1)\times(N-1) \),
поэтому для \( X =
    \begin{pmatrix}
    x_{11} & X_{12} \\
    X_{21} & X_{22}
    \end{pmatrix} \in \mathbb{C}^{N{\times}N} \):
\[
    \mathfrak{J} X = \begin{pmatrix} x_{11} & 0 \\ 0 & X_{22} \end{pmatrix}, \]
\[
    \Gamma X = \frac{1}{N}\begin{pmatrix} 0 & X_{12} \\ -X_{21} & 0 \end{pmatrix}. \]
\end{proof}

Теперь выпишем уравнение подобия матриц \( \mathcal{A} - \mathcal{B} \)
и \( \mathcal{A} - \mathfrak{J} X \):
\begin{equation}\label{kozlukovsv:eq:similarity}
    A(E+\Gamma X) = (E+\Gamma X)(\mathcal{A} - \mathfrak{J} X), \quad X\in\mathbb{C}^{N{\times}N}.
\end{equation}
\begin{lem}
    Уравнение~\eqref{kozlukovsv:eq:similarity} эквивалентно уравнению
    \begin{equation}\label{kozlukovsv:eq:fixptn}
        X = \mathcal{B} \Gamma X + \mathcal{B} - (\Gamma X)(\mathfrak{J}(\mathcal{B} (E + \Gamma X))), \quad X\in\mathbb{C}^{N{\times}N}.
    \end{equation}
\end{lem}
\begin{proof}
Раскрывая скобки, уравнение~\eqref{kozlukovsv:eq:similarity} можно преобразовать к виду
\begin{equation}\label{kozlukovsv:eq:fixptn-ini}
    X = \mathcal{B} \Gamma X + \mathcal{B} - (\Gamma X) \mathfrak{J} X.
\end{equation}
Пусть для \( X \) выполнено~\eqref{kozlukovsv:eq:fixptn-ini}.
Тогда
    \begin{equation}\label{kozlukovsv:eq:jx}
        \mathfrak{J} X = \mathfrak{J}(\mathcal{B} (E + \Gamma X)).
    \end{equation}
Подставляя это выражение обратно в~\eqref{kozlukovsv:eq:fixptn-ini}
    получим~\eqref{kozlukovsv:eq:fixptn}.
Обратно, применяя \( \mathfrak{J} \) к обеим частям уравнения~\eqref{kozlukovsv:eq:fixptn},
    получаем~\eqref{kozlukovsv:eq:jx} и~\eqref{kozlukovsv:eq:fixptn-ini}.
\end{proof}

Выражение в правой части уравнения~\eqref{kozlukovsv:eq:fixptn} обозначим как
\[
    \Phi(X) = \mathcal{B} \Gamma X + \mathcal{B} - (\Gamma X)(\mathfrak{J}(\mathcal{B} (E + \Gamma X))).\]
Теперь покажем, что, при определ\"енных условиях,
возникшее нелинейное отображение
\( \Phi{:}\ \mathbb{C}^{N{\times}N}{\to}\mathbb{C}^{N{\times}N} \) имеет инвариантным множеством
некоторый шар \( \Omega \subset \mathbb{C}^{N{\times}N} \) с~центром в~нуле
(т.е.~\( \Phi(\Omega)\subset\Omega \)),
на котором оно является сжимающим.

Пусть в~\( \mathbb{C}^{N{\times}N} \)
выбрана какая-нибудь субмультипликативная норма \( \|\cdot\| \)
(т.е.~норма, удовлетворяющая неравенству
 \( \| \mathcal{A}_1\mathcal{A}_2 \| \leq \|\mathcal{A}_1\|\|\mathcal{A}_2\| \)
 при всех \( \mathcal{A}_1, \mathcal{A}_2 \in \mathbb{C}^{N{\times}N} \)).
В пространстве \( L(\mathbb{C}^{N{\times}N}) \)
  линейных преобразований матриц размера \( N{\times}N \)
  будем рассматривать операторную норму
  \[
      \|\Psi\|_{\mathrm{op}} = \sup_{\|X\|=1,\ X\in\mathbb{C}^{N{\times}N}} \|\Psi X\|,\ \Psi\in L(\mathbb{C}^{N{\times}N})
      \]
Нам нужно найти такой радиус \( r \geq 0 \),
что при \( \|X\|,\|Y\| \leq r \) выполнялись бы неравенства \( \|\Phi(X)\| \leq r \)
и~\( \|\Phi(X) - \Phi(Y)\| < q\|X-Y\| \), \( q\in(0,1) \).
Обозначим
\( \beta = \|\mathcal{B}\| \), \( \gamma = \|\Gamma\|_{\mathrm{op}} = \sup_{\|X\|=1} \|\Gamma X\| \).

\begin{lem}
    Пусть \( \gamma\beta < \frac14\),
    тогда шар
    \[
        \Omega = \left\{ X\in \mathbb{C}^{N{\times}N}; \|X\| \leq r_0 \right\}, \]
    \[  0 < r_0 = \frac{1 - 2\gamma\beta - \sqrt{1-4\gamma\beta}}{2\gamma^2\beta} < 4\beta, \]
    удовлетворяет условию \( \Phi(\Omega)\subset\Omega \).
\end{lem}
\begin{proof}
Очевидно неравенство
    \[ \| \Phi(X) \| \leq
     \beta \gamma^2 \|X\|^2 + 2\beta\gamma\|X\| + \beta. \]
Значит, если \( r \) удовлетворяет неравенству
    \begin{equation}\label{kozlukovsv:ineq:invariance-radius}
        \beta \gamma^2 r^2 + (2\beta\gamma - 1)r + \beta \leq 0,
    \end{equation}
    то \( \|\Phi(X)\| \leq r \) при всех \( \|X\| \leq r \).
Если \( \gamma\beta \leq \frac14 \),
    то дискриминант \( \Delta = 1-4\gamma\beta \)
    соответствующего уравнения положителен и~его корни вещественны.
Из знаков коэффициентов возникшего многочлена видно, что оба корня положительны.
Следовательно, наименьший положительный \( r \),
    удовлетворяющий неравенству~\eqref{kozlukovsv:ineq:invariance-radius}
    есть наименьший корень
    соответствующего уравнения:
    \[ r_0 = \frac{1 - 2\gamma\beta - \sqrt{1-4\gamma\beta}}{2\gamma^2\beta}. \]
Учитывая \( \gamma\beta<\frac14 \), имеем \( r_0 < 4\beta \).
\end{proof}

Аналогичным образом устанавливается
\begin{lem}
    Пусть \(\gamma\beta<\frac14\),
    тогда \( \Phi \)~--- сжимающее отображение:
    \[ \| \Phi(X) - \Phi(Y) \| \leq q \|X - Y\|, \quad X,Y\in\Omega \]
    \[ q = (1+2\gamma r_0) \gamma\beta \leq (1+8\gamma\beta)\gamma\beta \leq \frac34. \]
\end{lem}
\begin{proof}
    \begin{align*} \| \Phi(X) - \Phi(Y) \| = \| \mathcal{B}\Gamma (X-Y) + (\Gamma X)(\mathcal{B}\Gamma X + \mathcal{B})
     - (\Gamma Y)(\mathcal{B} \Gamma Y + \mathcal{B}) \| \leq \\
        \leq
     \beta\gamma\|X-Y\| +
     \beta \gamma^2 \|X-Y\| \|X+Y\| \leq \\
        \leq
     \beta\gamma\|X-Y\| +
     2 r_0 \beta \gamma^2 \|X-Y\|.
    \end{align*}
Здесь использовано равенство
\[ (\Gamma X) \mathfrak{J}(\mathcal{B}\Gamma X) - (\Gamma Y) \mathfrak{J}(\mathcal{B}\Gamma Y) =
    \frac12\left[
        \Gamma(X-Y) \mathfrak{J}(\mathcal{B}\Gamma(X+Y))
    +   \Gamma(X+Y) \mathfrak{J}(\mathcal{B}\Gamma(X-Y))
    \right]. \]
\end{proof}

Отсюда и~из теоремы Банаха о~неподвижной точке следует:
\begin{lem}
В~шаре \[ \Omega = \left\{ X\in\mathbb{C}^{N{\times}N}; \quad \|X\| \leq r_0 \right\} \]
    существует и~при том единственное решение \( X^o \) уравнения~\eqref{kozlukovsv:eq:fixptn},
    являющееся пределом последовательности \( \{ \Phi^k(0); k\in\mathbb{N} \} \),
    где \( \Phi^k = \Phi\circ\Phi^{k-1} \)~--- композиция.
\end{lem}

\begin{crl}
    Матрица \( \mathcal{A} - \mathcal{B} \) подобна блочно-диагональной матрице \( \mathcal{A} - \mathfrak{J} X^o \):
    \[ \mathcal{A} - \mathcal{B} \sim
    \begin{pmatrix}
    N - x_{11}^o & 0 \\
    0 & -X_{22}^o
    \end{pmatrix}, \]
        при этом выполняются условия:
    \[ \sigma\left(\mathcal{A} - \mathcal{B}\right) = \left\{N-x_{11}^o\right\}\cup \sigma\left(-X_{22}^o\right), \]
        \[ x_{11}^o\in\mathbb{R}, \lvert x_{11}^o \rvert < r_0 \leq 4\beta, \]
    \[ \sigma\left(-X_{22}^o\right) \subset \{ \mu\in\mathbb{C}; \lvert x \rvert < r_0 \leq 4\beta \}. \]

    Собственное значение \( N - x_{11}^o \)
    совпадает со спектральным радиусом матрицы \( A \)
    и ему отвечает собственный вектор
    \[ 
        w = \mathcal{U}(E + \Gamma X^o)\begin{pmatrix}1\\0\end{pmatrix},
        \]
    \[
        \|w - \mathcal{U}\begin{pmatrix}1\\0\end{pmatrix}\|_2 \leq \frac{4\beta}{N}.
        \]
\end{crl}
\begin{proof}
    Матрица \( \mathcal{A} - \mathcal{B} \) подобна блочно-диагональной \( \mathcal{A} - \mathfrak{J} X^o \),
    поэтому их спектры совпадают.
    Спектр матрицы \( \mathcal{A} - \mathfrak{J} X^o \) есть объединение спектров е\"е диагональных блоков.
    В~виду субмультипликативности нормы имеют место неравенства
    \[ \mathtt{spr}(X^o) = \max_{\lambda\in\sigma(X^o)}\lvert\lambda\rvert \leq \|X^o\| \leq r_0. \]
    Кроме того, собственное значение \( x_{11}^o \) является вещественным, как предел сходящейся вещественной последовательности.

    Пусть теперь \( v \) --- собственный вектор матрицы \( \mathcal{A}-\mathfrak{J}X^o \):
    \[
        (\mathcal{A} - \mathfrak{J}X^o) v = \lambda v.
        \]
    Тогда
    \[
        (E+\Gamma X^o)^{-1}\mathcal{U}^{-1} A \mathcal{U} (E+\Gamma X^o) v = (\mathcal{A} - \mathfrak{J}X^o)v = \lambda v,
        \]
    \[
        A \underbrace{\mathcal{U} (E+\Gamma X^o) v}_{w} = \lambda \underbrace{\mathcal{U} (E+\Gamma X^o) v}_{w},
        \]
    при этом
    \[
        \|w - \mathcal{U}v\|_2 = \|\mathcal{U}\Gamma X^o v\|_2 \leq
        \|\Gamma\|_{\mathrm{op}} \|X^o\|_{\mathrm{op}} \|\mathcal{U} v\|_2 \leq
        \frac{4\beta}{N} \|v\|.
        \]
    Здесь использована ортогональность матрицы \( \mathcal{U} \).
    Наибольшему собственному значению \( N - x_{11}^o \)
        матрицы \( \mathcal{A} - \mathfrak{J}X \) соответствует
        собственный вектор \( \begin{pmatrix}1\\ 0\end{pmatrix} \).
    Подставляя \( v = \begin{pmatrix}1\\ 0\end{pmatrix} \)
            получим желаемый результат.
\end{proof}

Доказательство основной теоремы (стр. \pageref{thm:almostallones-spectra})
состоит в выборе подходящей субмультипликативной нормы.
Матрица \( B = U^{-1} \perturbmatrix{M}{N} U \)
получена из матрицы возмущения с \( M \) единицами
ортогональным преобразованием подобия \eqref{eq:diagtransform},
диагонализирующим матрицу единиц.
\( \Gamma \) --- оператор, действующий в \( \matr{N}{} \)
по формуле
\( { \Gamma X = \frac1N \begin{pmatrix}0 & X_{12} \\ -X_{21} & 0\end{pmatrix} } \).

Рассмотрим в пространстве \( \matr{N}{} \)
норму Фробениуса \( \normex{F}{\cdot} \),
определённую формулой
\( \normex{F}{X} = \sqrt{\sum_{ij} \lvert x_{ij}\rvert^2}. \)
Эта норма субмультипликативна.

Тогда:
\[ \gamma = \frac1N
            \sup_{\normex{F}{X}=1}\normex{F}{\begin{pmatrix}0 & X_{12} \\ -X_{21} & 0\end{pmatrix}}
          = \frac1N,
    \]
и, так как умножение на унитарную матрицу \( U \)
    (и \( U^{-1} \)) есть изометрия в \( \matr{N}{} \),
    а \( \perturbmatrix{M}{N} \) состоит из \( M \) единиц:
\[
    \beta = \normex{F}{B} =
    \normex{F}{\perturbmatrix{M}{N}} = \sqrt{M},
    \]

Значит, если
\( \sqrt{M} < \frac{N}{4} \), т.е.
\( M < \frac{N^2}{16} \),
то выполняются условия леммы,
причём \( r_0 < 4\sqrt{M} \).

\newpage

\smallskip\centerline{\bf Bibliography} 
% A nice bibtex' include was here,
% which used to generate a proper bibliography
% enforcing a single and consistent style;
% In these times of darkness we're not allowed
% to use macros or appropriate package
% or any other kind of generic solution
% --- you hardcode, so that you can't maintain
\begin{thebibliography}{9}
\bibitem{baskakov-harmonic}  Баскаков~А.~Г. Гармонический анализ линейных операторов
    / Баскаков~А.~Г.
    --- Воронеж : Издательство Воронежского Государственного Университета,
        1987.
    ---  с.~93--121.
\bibitem{epidemic-eigenvalues}  Yang Wang and D. Chakrabarti and Chenxi Wang and C. Faloutsos.
    Epidemic spreading in real networks: an eigenvalue viewpoint
        --- 22nd International Symposium on Reliable Distributed Systems, Oct 2003. Proceedings., --- pages~25-34.
\end{thebibliography}



\end{document}
