We will write \( \mathbb{C}^{N\times N} \) matrices down in the block form:
\( X{\sim}
    \begin{pmatrix}
    x_{11} & X_{12} \\
    X_{21} & X_{22}
    \end{pmatrix}, \)
    where \( x_{11} \) is a~scalar,
    \( X_{12} \)~--- a~row, \( X_{21} \)~--- a~column vector,
    \( X_{22} \)~--- a \( {(N{-}1){\times}(N{-}1)} \)~matrix.
Such block matrices form an algebra that is isomorphic to \( \mathbb{C}^{N\times N} \)
and they can be multiplied by elements of \( \mathbb{C}\times\mathbb{C}^{N-1}\sim\mathbb{C}^N \)
in a natural way:
\[
    X x \sim
    \begin{pmatrix}
    x_{11} & X_{12} \\
    X_{21} & X_{22}
    \end{pmatrix}
    \begin{pmatrix} x_1 \\ x_2 \end{pmatrix}
  = \begin{pmatrix}
      x_{11} x_1 + X_{12} x_2 \\
      X_{21} x_1 + X_{22} x_2
      \end{pmatrix},
    \]
\[
      x \sim
      \begin{pmatrix} x_1 \\ x_2 \end{pmatrix}
          \in \mathbb{C}\times\mathbb{C}^{N-1}.
    \]
Throughout the rest of the document isomorphic objects
    are used interchangeably.

Following the similar operators method~\cite{baskakov-split},
    we'll seek for a matrix \( X{\in}\mathbb{C}^{N{\times}N} \) such that
\begin{equation}\label{kozlukovsv:eq:similarity}
    (\mathcal{A-B})(E+\Gamma X) =
    (E+\Gamma X)(\mathcal{A} - \mathfrak{J} X).
\end{equation}
That is \( \mathcal{A}{-}\mathfrak{J} X \) is a ``simpler'' matrix
    that is similar to \( \mathcal{A}{-}\mathcal{B} \)
    and \( E{+}\Gamma X \) is the similarity matrix.
Here \( E\in{\mathbb{C}^{N{\times}N}} \) is an identity matrix,
    \( \mathfrak{J},\Gamma{:}\ \mathbb{C}^{N{\times}N}{\to}\mathbb{C}^{N{\times}N} \)
    are yet-to-be-picked linear operators called \emph{transformations}.
\( \mathfrak{J} \) is some projection (\(\mathfrak{J}^2=\mathfrak{J}\))
    such that \( \mathcal{A} - \mathfrak{J}X \) is a block-diagonal matrix.
The \( \Gamma \) is defined on every
    \( X\in {\mathbb{C}^{N{\times}N}} \)
    by the equation
\begin{equation}\label{eq:kozlukovsv:gamma}
    \mathcal{A}\Gamma X - (\Gamma X) \mathcal{A} = X - \mathfrak{J}X.
\end{equation}

\begin{lem}
    Operators \( \mathfrak{J} \) and \( \Gamma \)
    shall be defined as
    \[
        \mathfrak{J} X = \begin{pmatrix} x_{11} & 0 \\ 0 & X_{22} \end{pmatrix}, \]
    \[
        \Gamma X = \frac{1}{N} \begin{pmatrix} 0 & X_{12} \\ -X_{21} & 0 \end{pmatrix}, \]
     for \( X\sim \begin{pmatrix}x_{11} & X_{12} \\ X_{21} & X_{22}\end{pmatrix} \in \mathbb{C}^{N{\times}N} \).

\end{lem}
\begin{crl}
    The spectrum of a block-diagonal matrix
    \( \mathcal{A} - \mathfrak{J}X = \begin{pmatrix} N{-}x_{11} & 0 \\ 0 & X_{22} \end{pmatrix} \)
    is the union of the spectra of its diagonal blocks:
    \[
        \sigma(\mathcal{A} - \mathfrak{J} X) = \{ N - x_{11} \} \cup \sigma(X_{22}). \]
\end{crl}
\begin{proof}
Let \( \Gamma \) be defined by formula
    \( \Gamma X = \begin{pmatrix} \Gamma_{11}(X) & \Gamma_{12}(X) \\
        \Gamma_{21}(X) & \Gamma_{22}(X)
    \end{pmatrix} \).
    Then
\[
    \mathcal{A} \Gamma X - (\Gamma X)\mathcal{A} = 
    \begin{pmatrix} 0 & N\Gamma_{12}(X) \\
        - N\Gamma_{21}(X) & 0
        \end{pmatrix}. \]
The equation \eqref{eq:kozlukovsv:gamma} then reduces to
\[
    X - \mathfrak{J} X =
    N \begin{pmatrix} 0 & \Gamma_{12}(X) \\
        - \Gamma_{21}(X) & 0
        \end{pmatrix}. \]
Thus the best we can do with \( \mathfrak{J} \)
    is to project every \( X \) unto its two diagonal blocks of the sizes
    of \( 1\times 1 \) and \( (N-1)\times(N-1) \) respectively.

So we define
\[
    \mathfrak{J} X = \begin{pmatrix} x_{11} & 0 \\ 0 & X_{22} \end{pmatrix}, \]
\[
    \Gamma X = \frac{1}{N}\begin{pmatrix} 0 & X_{12} \\ -X_{21} & 0 \end{pmatrix}, \]
for \( X =
    \begin{pmatrix}
    x_{11} & X_{12} \\
    X_{21} & X_{22}
    \end{pmatrix} \in \mathbb{C}^{N{\times}N} \).
\end{proof}

\begin{lem}
    The equation~\eqref{kozlukovsv:eq:similarity} is equivalent to
    \begin{equation}\label{kozlukovsv:eq:fixptn}
        X = \mathcal{B} \Gamma X + \mathcal{B} - (\Gamma X)(\mathfrak{J}(\mathcal{B} (E + \Gamma X))),\ X\in\mathbb{C}^{N{\times}N}.
    \end{equation}
\end{lem}
\begin{proof}
By expansion the equation~\eqref{kozlukovsv:eq:similarity} is reduced to
\begin{equation}\label{kozlukovsv:eq:fixptn-ini}
    X = \mathcal{B} \Gamma X + \mathcal{B} - (\Gamma X) \mathfrak{J} X.
\end{equation}
Suppose \( X \) satisfies~\eqref{kozlukovsv:eq:fixptn-ini}.
Then
    \begin{equation}\label{kozlukovsv:eq:jx}
        \mathfrak{J} X = \mathfrak{J}(\mathcal{B} (E + \Gamma X)).
    \end{equation}
Substituting this equality back to~\eqref{kozlukovsv:eq:fixptn-ini}
    we obtain~\eqref{kozlukovsv:eq:fixptn}.
Conversely applying \( \mathfrak{J} \) to~\eqref{kozlukovsv:eq:fixptn},
    yields~\eqref{kozlukovsv:eq:jx} and~\eqref{kozlukovsv:eq:fixptn-ini}.
\end{proof}

Denote the right hand side expression of~\eqref{kozlukovsv:eq:fixptn} as
\[
    \Phi(X) = \mathcal{B} \Gamma X + \mathcal{B} - (\Gamma X)(\mathfrak{J}(\mathcal{B} (E + \Gamma X))).
    \]
Let's show that under certain restrictions
    this nonlinear mapping \( \Phi:\mathbb{C}^{N{\times}N}{\to}\mathbb{C}^{N{\times}N} \)
    restricted to a~cetrain zero-centered ball \( \Omega \subset \mathbb{C}^{N{\times}N} \)
    is a contraction.

Suppose there is some submultiplicative norm \( \|\cdot\| \) chosen in~\( \mathbb{C}^{N{\times}N} \),
    i.e.\ such a norm that satisfies
    \( \| \mathcal{A}_1\mathcal{A}_2 \| \leq \|\mathcal{A}_1\|\|\mathcal{A}_2\| \)
    for every \( \mathcal{A}_1, \mathcal{A}_2 \in \mathbb{C}^{N{\times}N} \).
For example one may consider Frobenius norm
\[
    \|Z\|_{\mathrm{F}}^2 = \sum_{ij}|z_{ij}|^2,\ Z=(z_{ij})\in\mathbb{C}^{N{\times}N}.
    \]
If the norm \( \|\cdot\| \) is submultiplicative,
    then
\[
    \max_{\lambda\in\sigma(Z)}|\lambda| \leq \|Z\|.
    \]


We want to find such a radius \( r \geq 0 \),
    that the ball \( \{ X{\in}\mathbb{C}^{N{\times}N};\ \|X\|{\leq}r \} \)
    is an invariant set under \( \Phi \)
    and \( \Phi \) is a contraction on this set.
That is we seek for such an \( r \)
    that \( \|X\|,\|Y\| \leq r \) implies \( \|\Phi(X)\| \leq r \)
and~\( \|\Phi(X) - \Phi(Y)\| < q\|X-Y\| \), \( q\in(0,1) \).
Denote
\( \beta = \|\mathcal{B}\| \), \( \gamma = \sup_{\|X\|=1} \|\Gamma X\| \).

\begin{lem}
    Let \( \gamma\beta < \frac14\),
    then the ball
    \[
        \Omega = \left\{ X\in \mathbb{C}^{N{\times}N}; \|X\| \leq r_0 \right\}, \]
    \[  0 < r_0 = \frac{1 - 2\gamma\beta - \sqrt{1-4\gamma\beta}}{2\gamma^2\beta} < 4\beta, \]
    satisfies the condition \( \Phi(\Omega)\subset\Omega \).
\end{lem}
\begin{proof}
By definition of \( \Phi \) and properties of the norm \( \|\cdot\| \):
    \[ \| \Phi(X) \| \leq
     \beta \gamma^2 \|X\|^2 + 2\beta\gamma\|X\| + \beta. \]
Thus if \( r \) satisfies inequality
    \begin{equation}\label{kozlukovsv:ineq:invariance-radius}
        \beta \gamma^2 r^2 + (2\beta\gamma - 1)r + \beta \leq 0,
    \end{equation}
    then \( \|\Phi(X)\| \leq r \) for all \( \|X\| \leq r \).
If \( \gamma\beta \leq \frac14 \),
    then the discriminant \( \Delta = 1-4\gamma\beta \)
    of the corresponding equation is positive.
The coefficients of the polynomial imply the roots are real and positive.
Hence the smallest \( r \),
    that satisfies~\eqref{kozlukovsv:ineq:invariance-radius} is
    \[ r_0 = \frac{1 - 2\gamma\beta - \sqrt{1-4\gamma\beta}}{2\gamma^2\beta}. \]
Finally \( \gamma\beta<\frac14 \) implies \( r_0 < 4\beta \).
\end{proof}

In a similar manner we prove the following
\begin{lem}
    Let \(\gamma\beta<\frac14\),
    then \( \Phi \) is a contraction on \( \Omega \):
    \[ \| \Phi(X) - \Phi(Y) \| \leq q \|X - Y\|, \quad X,Y\in\Omega \]
    \[ q = (1+2\gamma r_0) \gamma\beta \leq (1+8\gamma\beta)\gamma\beta \leq \frac34. \]
\end{lem}
\begin{proof}
    \begin{align*} \| \Phi(X) - \Phi(Y) \| = \| \mathcal{B}\Gamma (X-Y) + (\Gamma X)(\mathcal{B}\Gamma X + \mathcal{B})
     - (\Gamma Y)(\mathcal{B} \Gamma Y + \mathcal{B}) \| \leq \\
        \leq
     \beta\gamma\|X-Y\| +
     \beta \gamma^2 \|X-Y\| \|X+Y\| \leq \\
        \leq
     \beta\gamma\|X-Y\| +
     2 r_0 \beta \gamma^2 \|X-Y\|.
    \end{align*}
Here we used the equality
\[ (\Gamma X) \mathfrak{J}(\mathcal{B}\Gamma X) - (\Gamma Y) \mathfrak{J}(\mathcal{B}\Gamma Y) =
    \frac12\left[
        \Gamma(X-Y) \mathfrak{J}(\mathcal{B}\Gamma(X+Y))
    +   \Gamma(X+Y) \mathfrak{J}(\mathcal{B}\Gamma(X-Y))
    \right]. \]
\end{proof}

These two lemmas and the Banach fix-point theorem imply
\begin{lem}
    The equation~\eqref{kozlukovsv:eq:fixptn} has a~unique solution~\( X^o \)
    in the ball
    \[
        \Omega = \left\{ X\in\mathbb{C}^{N{\times}N}; \quad \|X\| \leq r_0 \right\}.
        \]
    This solution is a limit of the sequence \( \{ \Phi^k(0); k\in\mathbb{N} \} \),
    where \( \Phi^k = \Phi\circ\Phi^{k-1} \) denotes a composition.
\end{lem}

\begin{crl}
The matrix \( \mathcal{A} - \mathcal{B} \) is similar
    to a block-diagonal matrix \( \mathcal{A} - \mathfrak{J} X^o \):
\[
    \mathcal{A} - \mathcal{B} \sim
    \begin{pmatrix}
    N - x_{11}^o & 0 \\
    0 & -X_{22}^o
    \end{pmatrix},
    \]
and the following conditions are satisfied:
\[
    \sigma\left(\mathcal{A} - \mathcal{B}\right) = \left\{N-x_{11}^o\right\}\cup \sigma\left(-X_{22}^o\right),
    \]
\[
    x_{11}^o\in\mathbb{R}, \lvert x_{11}^o \rvert < r_0 \leq 4\beta,
    \]
\[
    \sigma\left(-X_{22}^o\right) \subset \{ \mu\in\mathbb{C}; \lvert x \rvert < r_0 \leq 4\beta \}.
    \]
\end{crl}
\begin{proof}
    The matrix \( \mathcal{A} - \mathcal{B} \) is similar to the block-diagonal matrix \( \mathcal{A} - \mathfrak{J} X^o \)
        so they are isospectral.
    The spectrum of \( \mathcal{A} - \mathfrak{J} X^o \)
        is the union of the spectra of its diagonal blocks.
    Submultiplicativity of the norm implies
    \[
        \mathtt{spr}(X^o) = \max_{\lambda\in\sigma(X^o)}\lvert\lambda\rvert \leq \|X^o\| \leq r_0.
        \]
    The final note is that \( x_{11}^o \) is a real number for it is a limit of convergent real sequence.
\end{proof}
