Now back to the main theorem
\begin{proof}[Proof of the Theorem~\ref{kozlukovsv:thm:almost-all-ones}]
    Now consider the space \( \mathbb{C}^{N{\times}N} \)
        with the submultiplicative Frobenius norm defined above.
    We remember that  \( \|\mathcal{B}\|=\|\mathscr{B}_{MN}\| \) for any norm \( \|\cdot\| \).
    Now that \( \mathscr{B}_{MN}=(b_{ij}) \) is comprised of \( M \) ones and the rest entries are zero,
        its Frobenius norm is
    \[
        \beta = {\left\|B\right\|}_{\mathrm{F}} =
        {\left\|\mathscr{B}_{MN}\right\|}_{\mathrm{F}} =
        \sqrt{\sum_{ij} \lvert |b_{ij}|\rvert^2} =
        \sqrt{M}.
        \]
    Also note a trivial equality
    \[
        \gamma = \frac1N
                \sup_{{\left\|X\right\|}_{\mathrm{F}}=1}{\left\|\begin{pmatrix}0 & X_{12} \\ -X_{21} & 0\end{pmatrix}\right\|}_{\mathrm{F}}
                = \frac1N. \]
    Thus the condition
        \( \sqrt{M} < \frac{N}{4} \),
        allows us to apply the last lemma
        with 
        \(
            r_0{<}4\sqrt{M}.
            \)
    The result is then:
        \[ \sigma(\mathscr{A}_{MN}) = \sigma_1 \cup \sigma_2. \]
        where
        \[ \sigma_1{=}\{ \lambda_1 \}{\subset}\mathbb{R},\ \lvert \lambda_1{-}N \rvert{<}4\sqrt{M}, \]
        \[ \sigma_2{\subset}\{ \mu{\in}\mathbb{C};\ \lvert\mu\rvert{<}4\sqrt{M} \}, \]
        \[ \sigma_1{\cap}\sigma_2{=}\emptyset. \]
    \end{proof}
