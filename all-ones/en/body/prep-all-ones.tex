Matrices \( \mathcal{A}_1, \mathcal{A}_2 \in\mathbb{C}^{N\times N} \)
    are said to be similar, if
    there exists such an invertible matrix \( \mathcal{U}\in\mathbb{C}^{N\times N} \)
    that \( \mathcal{A}_1 \mathcal{U} = \mathcal{U} \mathcal{A}_2 \).
Similar matrices are isospectral, i.e. they have exact same spectra.

The proof of the Theorem~\ref{kozlukovsv:thm:almost-all-ones} consists of construction of a matrix
    \emph{similar} to \( \mathscr{A}_{MN} \) (and thus isospectral)
    yet having ``simpler structure''.
The problem is reduced to a nonlinear equation in the Banach algebra
    \( \mathbb{C}^{N\times N} \) of \( N\times N \) complex matrices.
The existence of the solution of this equation is proved
    and norm estimates are derived using the fix-point iteration~\cite{baskakov-harmonic}.

Let's approach the problem by analyzing the \( \mathcal{J}_N \) first.

\begin{lem}
    The all-ones matrix
    \( \mathcal{J}_N =
    \begin{pmatrix}
        1 & \cdots & 1 \\
        \vdots & \ddots & \vdots \\ 
    1 & \cdots & 1 \end{pmatrix} \),
    is similar to the matrix
    \[
        \mathcal{A} = \begin{pmatrix}
            N & 0 & \cdots & 0 \\
            0 & 0 & \cdots & 0 \\
            \vdots & \vdots & \ddots & \vdots \\
            0 & 0 & \cdots & 0 \end{pmatrix}. \]
    There exists an orthogonal matrix \( \mathcal{U} \),
    such that
    \( \mathcal{J}_N = \mathcal{U}\mathcal{A} \mathcal{U}^{-1} \).
\end{lem}
\begin{proof}
    The kernel of \( \mathcal{J}_N \)
        is a \( (N-1) \)-dimensional subspace
        with the basis
        \( f_1 = {\left(1,-1,0,\ldots,0\right)}, \ldots,
           f_{N-1} = {\left(0,\ldots,0,1,-1\right)} \).
    The eigenvalue \( N \) of the matrix \( \mathcal{J}_N \) 
        has a single corresponding eigenvector
        \( f_N = {\left(1,\ldots,1\right)} \).
    Applying the Gram-Schmidt process,
        we get vectors \( h_1, \ldots, h_N \)
        which form an orthonormal eigenbasis of the matrix \( \mathcal{J}_N \):
    \[
        h_k = \frac{1}{\sqrt{k(k+1)}}
            \left(\smash{\underbrace{1,~\ldots,~1,}_{k\ \text{times}}}~-k,~0,~\ldots,~0\right)
            \in \mathbb{R}^N, \quad k={1, \ldots, N-1}, \]
    \[
        h_N = \frac{1}{\sqrt{N}}{\left(1,~\ldots,~1\right)} \in \mathbb{R}^N, \]
    The matrix \( \mathcal{U} \) is then
    comprised of column-vectors \( h_N, h_1, \ldots, h_{N-1} \):
    \[ \mathcal{U} =
    \begin{pmatrix}
        \frac{1}{\sqrt N} &  \frac{1}{\sqrt2} &  \frac{1}{\sqrt{6}} & \cdots & \frac{1}{\sqrt{(N-2)(N-1)}} & \frac{1}{\sqrt{N(N-1)}} \\
        \frac{1}{\sqrt N} & -\frac{1}{\sqrt2} &  \frac{1}{\sqrt{6}} & \cdots & \frac{1}{\sqrt{(N-2)(N-1)}} & \frac{1}{\sqrt{N(N-1)}} \\
        \frac{1}{\sqrt N} & 0                 & -\frac{2}{\sqrt{6}} & \cdots & \frac{1}{\sqrt{(N-2)(N-1)}} & \frac{1}{\sqrt{N(N-1)}} \\
        \frac{1}{\sqrt N} & 0                 &  0                  & \cdots & \frac{1}{\sqrt{(N-2)(N-1)}} & \frac{1}{\sqrt{N(N-1)}} \\
        \vdots            & \vdots            &  \vdots             & \ddots & \vdots                      & \vdots   \\
        \frac{1}{\sqrt N} & 0                 &  0                  & \cdots & \frac{1}{\sqrt{(N-2)(N-1)}} & \frac{1}{\sqrt{N(N-1)}} \\
        \frac{1}{\sqrt N} & 0                 &  0                  & \cdots & \frac{2-N}{\sqrt{(N-2)(N-1)}} & \frac{1}{\sqrt{N(N-1)}} \\
        \frac{1}{\sqrt N} & 0                 &  0                  & \cdots & 0                  & \frac{1-N}{\sqrt{N(N-1)}}
    \end{pmatrix}.\]
\end{proof}

The matrix \( \mathcal{A}_{MN} \) is of actual interest
    and it is clearly similar to the matrix
    \( \mathcal{A} - \mathcal{B} \),
    where \( \mathcal{B} = \mathcal{U}^{-1} \mathscr{B}_{MN} \mathcal{U} \).
The matrix \( \mathcal{U} \) is orthogonal and thus \( \|\mathcal{B}\| = \|\mathscr{B}_{MN}\| \).
