Suppose
\begin{equation}\label{eq:kozlukovsv:amn}
    \mathscr{A}_{MN} =
    \begin{pmatrix}
        1 & \cdots & 1 \\
        \vdots & \ddots & \vdots \\
        1 & \cdots & 1
    \end{pmatrix} - \mathscr{B}_{MN}
\end{equation}
    is a~\( N\times N \) matrix composed of
    \( N^2 - M \) unities and \( M \) zeroes.
If considered as an adjacency matrix \( \mathscr{A}_{MN} \)
    defines a~complete digraph (with loops) on \( N \) vertices
    with some \( M \) out of \( N^2 \) total edges removed.
Some important properties of a~graph are determined by its spectrum.
For example Wang et al. \cite{epidemic} proposed a~discrete-time model
    of viral propagation in a~network.
In this work it is shown that the ratio
    of the cure and the infection probabilities
    being belove or above a certain treshold value
    determines whether the contamination will take an epidemic form
    or fade out.
Wang et al. have shown that treshold value
    to be the spectral radius of the adjacency matrix of the network,
    i.e. the largest absolute value of its eigenvalues.
A more comprehensive description of spectral graph theory
    and its application is given by Cv\`etkovic et al. \cite{cvet}.

This article analyzes spectral properties of matrices such as \eqref{eq:kozlukovsv:amn}.
First of all the matrix \( \mathscr{A}_{MN} \) can be represented in the form
    \( \mathscr{A}_{MN} = \mathcal{J}_N - \mathscr{B}_{MN} \),
    where \( \mathcal{J}_N \) is a~\( N\times N \) matrix
    whose all entries are ones
    and \( \mathcal{B}_{MN} \) has ones exactly at these \( M \)
    places where \( \mathscr{A}_{MN} \) has zeros.
The spectrum of \( \mathcal{J}_N \) can be easily computed:
    \( \mathcal{J}_N^2 = N \mathcal{J} \),
    so \( \lambda(\lambda - N) \) is the minimal
    annihilating polynomial of \( \mathcal{J}_N \)
    and thus the spectrum of \( \mathcal{J}_{N} \) is
    \( \sigma(\mathcal{J}_N) = \{ 0,N \} \).

Be \( M \) small enough
    the eigenvalues of \( \mathscr{A}_{MN} \) would be close to those of \( \mathcal{J}_N \).
Applying the method of similar operators (see \cite{baskakov-harmonic,baskakov-split})
    we prove the following theorem:
\begin{thm}\label{kozlukovsv:thm:almost-all-ones}
    Let \( M < \frac{N^2}{16} \),
    then the spectrum of \( \mathscr{A}_{MN} \)
    can be represented as a~disjoint union
    \( \sigma\left(\mathscr{A}_{MN}\right) = \sigma_1 \cup \sigma_2 \)
    of a~singletone \( \sigma_1=\{\lambda_1\} \)
    and the set \( \sigma_2 \) satisfying the following restrictions:
    \[
        \sigma_1 \subset \left\{
            \mu\in\mathbb{R};\ \lvert N - \mu \rvert < 4\sqrt{M}
            \right\},
        \]
    \[ \sigma_2 \subset \left\{ \mu\in\mathbb{C};\ \lvert \mu \rvert < 4\sqrt{M} \right\}. \]
\end{thm}

